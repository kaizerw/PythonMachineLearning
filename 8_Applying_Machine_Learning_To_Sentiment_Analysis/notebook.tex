
% Default to the notebook output style

    


% Inherit from the specified cell style.




    
\documentclass[11pt]{article}

    
    
    \usepackage[T1]{fontenc}
    % Nicer default font (+ math font) than Computer Modern for most use cases
    \usepackage{mathpazo}

    % Basic figure setup, for now with no caption control since it's done
    % automatically by Pandoc (which extracts ![](path) syntax from Markdown).
    \usepackage{graphicx}
    % We will generate all images so they have a width \maxwidth. This means
    % that they will get their normal width if they fit onto the page, but
    % are scaled down if they would overflow the margins.
    \makeatletter
    \def\maxwidth{\ifdim\Gin@nat@width>\linewidth\linewidth
    \else\Gin@nat@width\fi}
    \makeatother
    \let\Oldincludegraphics\includegraphics
    % Set max figure width to be 80% of text width, for now hardcoded.
    \renewcommand{\includegraphics}[1]{\Oldincludegraphics[width=.8\maxwidth]{#1}}
    % Ensure that by default, figures have no caption (until we provide a
    % proper Figure object with a Caption API and a way to capture that
    % in the conversion process - todo).
    \usepackage{caption}
    \DeclareCaptionLabelFormat{nolabel}{}
    \captionsetup{labelformat=nolabel}

    \usepackage{adjustbox} % Used to constrain images to a maximum size 
    \usepackage{xcolor} % Allow colors to be defined
    \usepackage{enumerate} % Needed for markdown enumerations to work
    \usepackage{geometry} % Used to adjust the document margins
    \usepackage{amsmath} % Equations
    \usepackage{amssymb} % Equations
    \usepackage{textcomp} % defines textquotesingle
    % Hack from http://tex.stackexchange.com/a/47451/13684:
    \AtBeginDocument{%
        \def\PYZsq{\textquotesingle}% Upright quotes in Pygmentized code
    }
    \usepackage{upquote} % Upright quotes for verbatim code
    \usepackage{eurosym} % defines \euro
    \usepackage[mathletters]{ucs} % Extended unicode (utf-8) support
    \usepackage[utf8x]{inputenc} % Allow utf-8 characters in the tex document
    \usepackage{fancyvrb} % verbatim replacement that allows latex
    \usepackage{grffile} % extends the file name processing of package graphics 
                         % to support a larger range 
    % The hyperref package gives us a pdf with properly built
    % internal navigation ('pdf bookmarks' for the table of contents,
    % internal cross-reference links, web links for URLs, etc.)
    \usepackage{hyperref}
    \usepackage{longtable} % longtable support required by pandoc >1.10
    \usepackage{booktabs}  % table support for pandoc > 1.12.2
    \usepackage[inline]{enumitem} % IRkernel/repr support (it uses the enumerate* environment)
    \usepackage[normalem]{ulem} % ulem is needed to support strikethroughs (\sout)
                                % normalem makes italics be italics, not underlines
    

    
    
    % Colors for the hyperref package
    \definecolor{urlcolor}{rgb}{0,.145,.698}
    \definecolor{linkcolor}{rgb}{.71,0.21,0.01}
    \definecolor{citecolor}{rgb}{.12,.54,.11}

    % ANSI colors
    \definecolor{ansi-black}{HTML}{3E424D}
    \definecolor{ansi-black-intense}{HTML}{282C36}
    \definecolor{ansi-red}{HTML}{E75C58}
    \definecolor{ansi-red-intense}{HTML}{B22B31}
    \definecolor{ansi-green}{HTML}{00A250}
    \definecolor{ansi-green-intense}{HTML}{007427}
    \definecolor{ansi-yellow}{HTML}{DDB62B}
    \definecolor{ansi-yellow-intense}{HTML}{B27D12}
    \definecolor{ansi-blue}{HTML}{208FFB}
    \definecolor{ansi-blue-intense}{HTML}{0065CA}
    \definecolor{ansi-magenta}{HTML}{D160C4}
    \definecolor{ansi-magenta-intense}{HTML}{A03196}
    \definecolor{ansi-cyan}{HTML}{60C6C8}
    \definecolor{ansi-cyan-intense}{HTML}{258F8F}
    \definecolor{ansi-white}{HTML}{C5C1B4}
    \definecolor{ansi-white-intense}{HTML}{A1A6B2}

    % commands and environments needed by pandoc snippets
    % extracted from the output of `pandoc -s`
    \providecommand{\tightlist}{%
      \setlength{\itemsep}{0pt}\setlength{\parskip}{0pt}}
    \DefineVerbatimEnvironment{Highlighting}{Verbatim}{commandchars=\\\{\}}
    % Add ',fontsize=\small' for more characters per line
    \newenvironment{Shaded}{}{}
    \newcommand{\KeywordTok}[1]{\textcolor[rgb]{0.00,0.44,0.13}{\textbf{{#1}}}}
    \newcommand{\DataTypeTok}[1]{\textcolor[rgb]{0.56,0.13,0.00}{{#1}}}
    \newcommand{\DecValTok}[1]{\textcolor[rgb]{0.25,0.63,0.44}{{#1}}}
    \newcommand{\BaseNTok}[1]{\textcolor[rgb]{0.25,0.63,0.44}{{#1}}}
    \newcommand{\FloatTok}[1]{\textcolor[rgb]{0.25,0.63,0.44}{{#1}}}
    \newcommand{\CharTok}[1]{\textcolor[rgb]{0.25,0.44,0.63}{{#1}}}
    \newcommand{\StringTok}[1]{\textcolor[rgb]{0.25,0.44,0.63}{{#1}}}
    \newcommand{\CommentTok}[1]{\textcolor[rgb]{0.38,0.63,0.69}{\textit{{#1}}}}
    \newcommand{\OtherTok}[1]{\textcolor[rgb]{0.00,0.44,0.13}{{#1}}}
    \newcommand{\AlertTok}[1]{\textcolor[rgb]{1.00,0.00,0.00}{\textbf{{#1}}}}
    \newcommand{\FunctionTok}[1]{\textcolor[rgb]{0.02,0.16,0.49}{{#1}}}
    \newcommand{\RegionMarkerTok}[1]{{#1}}
    \newcommand{\ErrorTok}[1]{\textcolor[rgb]{1.00,0.00,0.00}{\textbf{{#1}}}}
    \newcommand{\NormalTok}[1]{{#1}}
    
    % Additional commands for more recent versions of Pandoc
    \newcommand{\ConstantTok}[1]{\textcolor[rgb]{0.53,0.00,0.00}{{#1}}}
    \newcommand{\SpecialCharTok}[1]{\textcolor[rgb]{0.25,0.44,0.63}{{#1}}}
    \newcommand{\VerbatimStringTok}[1]{\textcolor[rgb]{0.25,0.44,0.63}{{#1}}}
    \newcommand{\SpecialStringTok}[1]{\textcolor[rgb]{0.73,0.40,0.53}{{#1}}}
    \newcommand{\ImportTok}[1]{{#1}}
    \newcommand{\DocumentationTok}[1]{\textcolor[rgb]{0.73,0.13,0.13}{\textit{{#1}}}}
    \newcommand{\AnnotationTok}[1]{\textcolor[rgb]{0.38,0.63,0.69}{\textbf{\textit{{#1}}}}}
    \newcommand{\CommentVarTok}[1]{\textcolor[rgb]{0.38,0.63,0.69}{\textbf{\textit{{#1}}}}}
    \newcommand{\VariableTok}[1]{\textcolor[rgb]{0.10,0.09,0.49}{{#1}}}
    \newcommand{\ControlFlowTok}[1]{\textcolor[rgb]{0.00,0.44,0.13}{\textbf{{#1}}}}
    \newcommand{\OperatorTok}[1]{\textcolor[rgb]{0.40,0.40,0.40}{{#1}}}
    \newcommand{\BuiltInTok}[1]{{#1}}
    \newcommand{\ExtensionTok}[1]{{#1}}
    \newcommand{\PreprocessorTok}[1]{\textcolor[rgb]{0.74,0.48,0.00}{{#1}}}
    \newcommand{\AttributeTok}[1]{\textcolor[rgb]{0.49,0.56,0.16}{{#1}}}
    \newcommand{\InformationTok}[1]{\textcolor[rgb]{0.38,0.63,0.69}{\textbf{\textit{{#1}}}}}
    \newcommand{\WarningTok}[1]{\textcolor[rgb]{0.38,0.63,0.69}{\textbf{\textit{{#1}}}}}
    
    
    % Define a nice break command that doesn't care if a line doesn't already
    % exist.
    \def\br{\hspace*{\fill} \\* }
    % Math Jax compatability definitions
    \def\gt{>}
    \def\lt{<}
    % Document parameters
    \title{8\_Applying\_Machine\_Learning\_To\_Sentiment\_Analysis}
    
    
    

    % Pygments definitions
    
\makeatletter
\def\PY@reset{\let\PY@it=\relax \let\PY@bf=\relax%
    \let\PY@ul=\relax \let\PY@tc=\relax%
    \let\PY@bc=\relax \let\PY@ff=\relax}
\def\PY@tok#1{\csname PY@tok@#1\endcsname}
\def\PY@toks#1+{\ifx\relax#1\empty\else%
    \PY@tok{#1}\expandafter\PY@toks\fi}
\def\PY@do#1{\PY@bc{\PY@tc{\PY@ul{%
    \PY@it{\PY@bf{\PY@ff{#1}}}}}}}
\def\PY#1#2{\PY@reset\PY@toks#1+\relax+\PY@do{#2}}

\expandafter\def\csname PY@tok@w\endcsname{\def\PY@tc##1{\textcolor[rgb]{0.73,0.73,0.73}{##1}}}
\expandafter\def\csname PY@tok@c\endcsname{\let\PY@it=\textit\def\PY@tc##1{\textcolor[rgb]{0.25,0.50,0.50}{##1}}}
\expandafter\def\csname PY@tok@cp\endcsname{\def\PY@tc##1{\textcolor[rgb]{0.74,0.48,0.00}{##1}}}
\expandafter\def\csname PY@tok@k\endcsname{\let\PY@bf=\textbf\def\PY@tc##1{\textcolor[rgb]{0.00,0.50,0.00}{##1}}}
\expandafter\def\csname PY@tok@kp\endcsname{\def\PY@tc##1{\textcolor[rgb]{0.00,0.50,0.00}{##1}}}
\expandafter\def\csname PY@tok@kt\endcsname{\def\PY@tc##1{\textcolor[rgb]{0.69,0.00,0.25}{##1}}}
\expandafter\def\csname PY@tok@o\endcsname{\def\PY@tc##1{\textcolor[rgb]{0.40,0.40,0.40}{##1}}}
\expandafter\def\csname PY@tok@ow\endcsname{\let\PY@bf=\textbf\def\PY@tc##1{\textcolor[rgb]{0.67,0.13,1.00}{##1}}}
\expandafter\def\csname PY@tok@nb\endcsname{\def\PY@tc##1{\textcolor[rgb]{0.00,0.50,0.00}{##1}}}
\expandafter\def\csname PY@tok@nf\endcsname{\def\PY@tc##1{\textcolor[rgb]{0.00,0.00,1.00}{##1}}}
\expandafter\def\csname PY@tok@nc\endcsname{\let\PY@bf=\textbf\def\PY@tc##1{\textcolor[rgb]{0.00,0.00,1.00}{##1}}}
\expandafter\def\csname PY@tok@nn\endcsname{\let\PY@bf=\textbf\def\PY@tc##1{\textcolor[rgb]{0.00,0.00,1.00}{##1}}}
\expandafter\def\csname PY@tok@ne\endcsname{\let\PY@bf=\textbf\def\PY@tc##1{\textcolor[rgb]{0.82,0.25,0.23}{##1}}}
\expandafter\def\csname PY@tok@nv\endcsname{\def\PY@tc##1{\textcolor[rgb]{0.10,0.09,0.49}{##1}}}
\expandafter\def\csname PY@tok@no\endcsname{\def\PY@tc##1{\textcolor[rgb]{0.53,0.00,0.00}{##1}}}
\expandafter\def\csname PY@tok@nl\endcsname{\def\PY@tc##1{\textcolor[rgb]{0.63,0.63,0.00}{##1}}}
\expandafter\def\csname PY@tok@ni\endcsname{\let\PY@bf=\textbf\def\PY@tc##1{\textcolor[rgb]{0.60,0.60,0.60}{##1}}}
\expandafter\def\csname PY@tok@na\endcsname{\def\PY@tc##1{\textcolor[rgb]{0.49,0.56,0.16}{##1}}}
\expandafter\def\csname PY@tok@nt\endcsname{\let\PY@bf=\textbf\def\PY@tc##1{\textcolor[rgb]{0.00,0.50,0.00}{##1}}}
\expandafter\def\csname PY@tok@nd\endcsname{\def\PY@tc##1{\textcolor[rgb]{0.67,0.13,1.00}{##1}}}
\expandafter\def\csname PY@tok@s\endcsname{\def\PY@tc##1{\textcolor[rgb]{0.73,0.13,0.13}{##1}}}
\expandafter\def\csname PY@tok@sd\endcsname{\let\PY@it=\textit\def\PY@tc##1{\textcolor[rgb]{0.73,0.13,0.13}{##1}}}
\expandafter\def\csname PY@tok@si\endcsname{\let\PY@bf=\textbf\def\PY@tc##1{\textcolor[rgb]{0.73,0.40,0.53}{##1}}}
\expandafter\def\csname PY@tok@se\endcsname{\let\PY@bf=\textbf\def\PY@tc##1{\textcolor[rgb]{0.73,0.40,0.13}{##1}}}
\expandafter\def\csname PY@tok@sr\endcsname{\def\PY@tc##1{\textcolor[rgb]{0.73,0.40,0.53}{##1}}}
\expandafter\def\csname PY@tok@ss\endcsname{\def\PY@tc##1{\textcolor[rgb]{0.10,0.09,0.49}{##1}}}
\expandafter\def\csname PY@tok@sx\endcsname{\def\PY@tc##1{\textcolor[rgb]{0.00,0.50,0.00}{##1}}}
\expandafter\def\csname PY@tok@m\endcsname{\def\PY@tc##1{\textcolor[rgb]{0.40,0.40,0.40}{##1}}}
\expandafter\def\csname PY@tok@gh\endcsname{\let\PY@bf=\textbf\def\PY@tc##1{\textcolor[rgb]{0.00,0.00,0.50}{##1}}}
\expandafter\def\csname PY@tok@gu\endcsname{\let\PY@bf=\textbf\def\PY@tc##1{\textcolor[rgb]{0.50,0.00,0.50}{##1}}}
\expandafter\def\csname PY@tok@gd\endcsname{\def\PY@tc##1{\textcolor[rgb]{0.63,0.00,0.00}{##1}}}
\expandafter\def\csname PY@tok@gi\endcsname{\def\PY@tc##1{\textcolor[rgb]{0.00,0.63,0.00}{##1}}}
\expandafter\def\csname PY@tok@gr\endcsname{\def\PY@tc##1{\textcolor[rgb]{1.00,0.00,0.00}{##1}}}
\expandafter\def\csname PY@tok@ge\endcsname{\let\PY@it=\textit}
\expandafter\def\csname PY@tok@gs\endcsname{\let\PY@bf=\textbf}
\expandafter\def\csname PY@tok@gp\endcsname{\let\PY@bf=\textbf\def\PY@tc##1{\textcolor[rgb]{0.00,0.00,0.50}{##1}}}
\expandafter\def\csname PY@tok@go\endcsname{\def\PY@tc##1{\textcolor[rgb]{0.53,0.53,0.53}{##1}}}
\expandafter\def\csname PY@tok@gt\endcsname{\def\PY@tc##1{\textcolor[rgb]{0.00,0.27,0.87}{##1}}}
\expandafter\def\csname PY@tok@err\endcsname{\def\PY@bc##1{\setlength{\fboxsep}{0pt}\fcolorbox[rgb]{1.00,0.00,0.00}{1,1,1}{\strut ##1}}}
\expandafter\def\csname PY@tok@kc\endcsname{\let\PY@bf=\textbf\def\PY@tc##1{\textcolor[rgb]{0.00,0.50,0.00}{##1}}}
\expandafter\def\csname PY@tok@kd\endcsname{\let\PY@bf=\textbf\def\PY@tc##1{\textcolor[rgb]{0.00,0.50,0.00}{##1}}}
\expandafter\def\csname PY@tok@kn\endcsname{\let\PY@bf=\textbf\def\PY@tc##1{\textcolor[rgb]{0.00,0.50,0.00}{##1}}}
\expandafter\def\csname PY@tok@kr\endcsname{\let\PY@bf=\textbf\def\PY@tc##1{\textcolor[rgb]{0.00,0.50,0.00}{##1}}}
\expandafter\def\csname PY@tok@bp\endcsname{\def\PY@tc##1{\textcolor[rgb]{0.00,0.50,0.00}{##1}}}
\expandafter\def\csname PY@tok@fm\endcsname{\def\PY@tc##1{\textcolor[rgb]{0.00,0.00,1.00}{##1}}}
\expandafter\def\csname PY@tok@vc\endcsname{\def\PY@tc##1{\textcolor[rgb]{0.10,0.09,0.49}{##1}}}
\expandafter\def\csname PY@tok@vg\endcsname{\def\PY@tc##1{\textcolor[rgb]{0.10,0.09,0.49}{##1}}}
\expandafter\def\csname PY@tok@vi\endcsname{\def\PY@tc##1{\textcolor[rgb]{0.10,0.09,0.49}{##1}}}
\expandafter\def\csname PY@tok@vm\endcsname{\def\PY@tc##1{\textcolor[rgb]{0.10,0.09,0.49}{##1}}}
\expandafter\def\csname PY@tok@sa\endcsname{\def\PY@tc##1{\textcolor[rgb]{0.73,0.13,0.13}{##1}}}
\expandafter\def\csname PY@tok@sb\endcsname{\def\PY@tc##1{\textcolor[rgb]{0.73,0.13,0.13}{##1}}}
\expandafter\def\csname PY@tok@sc\endcsname{\def\PY@tc##1{\textcolor[rgb]{0.73,0.13,0.13}{##1}}}
\expandafter\def\csname PY@tok@dl\endcsname{\def\PY@tc##1{\textcolor[rgb]{0.73,0.13,0.13}{##1}}}
\expandafter\def\csname PY@tok@s2\endcsname{\def\PY@tc##1{\textcolor[rgb]{0.73,0.13,0.13}{##1}}}
\expandafter\def\csname PY@tok@sh\endcsname{\def\PY@tc##1{\textcolor[rgb]{0.73,0.13,0.13}{##1}}}
\expandafter\def\csname PY@tok@s1\endcsname{\def\PY@tc##1{\textcolor[rgb]{0.73,0.13,0.13}{##1}}}
\expandafter\def\csname PY@tok@mb\endcsname{\def\PY@tc##1{\textcolor[rgb]{0.40,0.40,0.40}{##1}}}
\expandafter\def\csname PY@tok@mf\endcsname{\def\PY@tc##1{\textcolor[rgb]{0.40,0.40,0.40}{##1}}}
\expandafter\def\csname PY@tok@mh\endcsname{\def\PY@tc##1{\textcolor[rgb]{0.40,0.40,0.40}{##1}}}
\expandafter\def\csname PY@tok@mi\endcsname{\def\PY@tc##1{\textcolor[rgb]{0.40,0.40,0.40}{##1}}}
\expandafter\def\csname PY@tok@il\endcsname{\def\PY@tc##1{\textcolor[rgb]{0.40,0.40,0.40}{##1}}}
\expandafter\def\csname PY@tok@mo\endcsname{\def\PY@tc##1{\textcolor[rgb]{0.40,0.40,0.40}{##1}}}
\expandafter\def\csname PY@tok@ch\endcsname{\let\PY@it=\textit\def\PY@tc##1{\textcolor[rgb]{0.25,0.50,0.50}{##1}}}
\expandafter\def\csname PY@tok@cm\endcsname{\let\PY@it=\textit\def\PY@tc##1{\textcolor[rgb]{0.25,0.50,0.50}{##1}}}
\expandafter\def\csname PY@tok@cpf\endcsname{\let\PY@it=\textit\def\PY@tc##1{\textcolor[rgb]{0.25,0.50,0.50}{##1}}}
\expandafter\def\csname PY@tok@c1\endcsname{\let\PY@it=\textit\def\PY@tc##1{\textcolor[rgb]{0.25,0.50,0.50}{##1}}}
\expandafter\def\csname PY@tok@cs\endcsname{\let\PY@it=\textit\def\PY@tc##1{\textcolor[rgb]{0.25,0.50,0.50}{##1}}}

\def\PYZbs{\char`\\}
\def\PYZus{\char`\_}
\def\PYZob{\char`\{}
\def\PYZcb{\char`\}}
\def\PYZca{\char`\^}
\def\PYZam{\char`\&}
\def\PYZlt{\char`\<}
\def\PYZgt{\char`\>}
\def\PYZsh{\char`\#}
\def\PYZpc{\char`\%}
\def\PYZdl{\char`\$}
\def\PYZhy{\char`\-}
\def\PYZsq{\char`\'}
\def\PYZdq{\char`\"}
\def\PYZti{\char`\~}
% for compatibility with earlier versions
\def\PYZat{@}
\def\PYZlb{[}
\def\PYZrb{]}
\makeatother


    % Exact colors from NB
    \definecolor{incolor}{rgb}{0.0, 0.0, 0.5}
    \definecolor{outcolor}{rgb}{0.545, 0.0, 0.0}



    
    % Prevent overflowing lines due to hard-to-break entities
    \sloppy 
    % Setup hyperref package
    \hypersetup{
      breaklinks=true,  % so long urls are correctly broken across lines
      colorlinks=true,
      urlcolor=urlcolor,
      linkcolor=linkcolor,
      citecolor=citecolor,
      }
    % Slightly bigger margins than the latex defaults
    
    \geometry{verbose,tmargin=1in,bmargin=1in,lmargin=1in,rmargin=1in}
    
    

    \begin{document}
    
    
    \maketitle
    
    

    
    \section{Applying Machine Learning to Sentiment
Analysis}\label{applying-machine-learning-to-sentiment-analysis}

    In this internet and social media age, people's opinions, reviews, and
recommendations have become a valuable resource of political science and
business. Thanks to modern technologies, we are now able to collect and
analyze such data most efficiently. In this chapter, we will delve into
a subfield of \textbf{Natural Language Processing (NLP)} called
\textbf{sentiment analysis} and learn how to use machine learning
algorithms to classify documents based on their polarity: the attitude
of the writer. In particular, we are going to work with a a dataset of
50.000 movie reviews from the \textbf{Internet Movie Database (IMDb)}
and build a predict that can distinguish between positive and negative
reviews.

The topics that we will cover in the following sections include the
following:

\begin{itemize}
\tightlist
\item
  Cleaning and preparing text data
\item
  Building feature vectors from text documents
\item
  Training a machine learning model to classify positive and negative
  movie reviews
\item
  Working with large text datasets using out-of-core learning
\item
  Inferring topics from document collections for categorization
\end{itemize}

    \subsection{Preparing the IMDb movie review data for text
processing}\label{preparing-the-imdb-movie-review-data-for-text-processing}

    Sentiment analysis, sometimes also called \textbf{opinion mining}, is a
popular subdiscipline of the broader field of NLP; it is concerned with
analyzing the polarity of documents. A popular task in sentiment
analysis is the classification of documents based on the expressed
opinions or emotions of the authors with regard to a particualr topic.

In this chapter, we will be working with a large dataset of movie
reviews from the IMDb that has been collected by Maas and others. The
movie review datasets consists of 50.000 polar movie reviews that are
labeled as either positive or negative; here, positive means that a
movie was rated with more than six stars on IMBDb, and negative means
that a movie was rated with fewer than five stars on IMDb. In the
following sections, we will download the dataset, preprocess it into a
useable format for machine learning tools, and extract meaningful
information from a subset of these movie reviews to build a machine
learning model that can predict whether a certain reviewer liked or
disliked a movie.

    \subsection{Obtaining the movie review
dataset}\label{obtaining-the-movie-review-dataset}

    A compressed archive of the movie review dataset (84.1 MB) can be
downloaded from
http://ai.stanford.edu/\textasciitilde{}amaas/data/sentiment/ as a
Gzip-compressed tarball archive. You can unpack the Gzip-compressed
tarball archive directly in Python as follows:

    \begin{Verbatim}[commandchars=\\\{\}]
{\color{incolor}In [{\color{incolor}1}]:} \PY{k+kn}{import} \PY{n+nn}{tarfile}
        
        \PY{k}{with} \PY{n}{tarfile}\PY{o}{.}\PY{n}{open}\PY{p}{(}\PY{l+s+s1}{\PYZsq{}}\PY{l+s+s1}{aclImdb\PYZus{}v1.tar.gz}\PY{l+s+s1}{\PYZsq{}}\PY{p}{,} \PY{l+s+s1}{\PYZsq{}}\PY{l+s+s1}{r:gz}\PY{l+s+s1}{\PYZsq{}}\PY{p}{)} \PY{k}{as} \PY{n}{tar}\PY{p}{:}
            \PY{n}{tar}\PY{o}{.}\PY{n}{extractall}\PY{p}{(}\PY{p}{)}
\end{Verbatim}


    \subsection{Preprocessing the movie dataset into more convenient
format}\label{preprocessing-the-movie-dataset-into-more-convenient-format}

    Having successfully extracted the dataset, we will now assemble the
individual text documents from the decompressed download archive into a
single CSV file. In the following code section, we will be reading the
movie reviews into a pandas \emph{DataFrame} object, which can take up
to 10 minutes on a standard desktop computer. To visualize the progress
and estimated time until completion, we will use the \textbf{Python
Progress Indicator (PyPrind)} package. PyPrind can be installed by
executing the \emph{pip install pyprind} command.

    \begin{Verbatim}[commandchars=\\\{\}]
{\color{incolor}In [{\color{incolor}2}]:} \PY{k+kn}{import} \PY{n+nn}{pyprind}
        \PY{k+kn}{import} \PY{n+nn}{pandas} \PY{k}{as} \PY{n+nn}{pd}
        \PY{k+kn}{import} \PY{n+nn}{os}
        
        \PY{n}{basepath} \PY{o}{=} \PY{l+s+s1}{\PYZsq{}}\PY{l+s+s1}{aclImdb}\PY{l+s+s1}{\PYZsq{}}
        
        \PY{n}{labels} \PY{o}{=} \PY{p}{\PYZob{}}\PY{l+s+s1}{\PYZsq{}}\PY{l+s+s1}{pos}\PY{l+s+s1}{\PYZsq{}}\PY{p}{:} \PY{l+m+mi}{1}\PY{p}{,} \PY{l+s+s1}{\PYZsq{}}\PY{l+s+s1}{neg}\PY{l+s+s1}{\PYZsq{}}\PY{p}{:} \PY{l+m+mi}{0}\PY{p}{\PYZcb{}}
        \PY{n}{pbar} \PY{o}{=} \PY{n}{pyprind}\PY{o}{.}\PY{n}{ProgBar}\PY{p}{(}\PY{l+m+mi}{50000}\PY{p}{)}
        \PY{n}{df} \PY{o}{=} \PY{n}{pd}\PY{o}{.}\PY{n}{DataFrame}\PY{p}{(}\PY{p}{)}
        \PY{k}{for} \PY{n}{s} \PY{o+ow}{in} \PY{p}{(}\PY{l+s+s1}{\PYZsq{}}\PY{l+s+s1}{test}\PY{l+s+s1}{\PYZsq{}}\PY{p}{,} \PY{l+s+s1}{\PYZsq{}}\PY{l+s+s1}{train}\PY{l+s+s1}{\PYZsq{}}\PY{p}{)}\PY{p}{:}
            \PY{k}{for} \PY{n}{l} \PY{o+ow}{in} \PY{p}{(}\PY{l+s+s1}{\PYZsq{}}\PY{l+s+s1}{pos}\PY{l+s+s1}{\PYZsq{}}\PY{p}{,} \PY{l+s+s1}{\PYZsq{}}\PY{l+s+s1}{neg}\PY{l+s+s1}{\PYZsq{}}\PY{p}{)}\PY{p}{:}
                \PY{n}{path} \PY{o}{=} \PY{n}{os}\PY{o}{.}\PY{n}{path}\PY{o}{.}\PY{n}{join}\PY{p}{(}\PY{n}{basepath}\PY{p}{,} \PY{n}{s}\PY{p}{,} \PY{n}{l}\PY{p}{)}
                \PY{k}{for} \PY{n}{file} \PY{o+ow}{in} \PY{n}{os}\PY{o}{.}\PY{n}{listdir}\PY{p}{(}\PY{n}{path}\PY{p}{)}\PY{p}{:}
                    \PY{k}{with} \PY{n+nb}{open}\PY{p}{(}\PY{n}{os}\PY{o}{.}\PY{n}{path}\PY{o}{.}\PY{n}{join}\PY{p}{(}\PY{n}{path}\PY{p}{,} \PY{n}{file}\PY{p}{)}\PY{p}{,} \PY{l+s+s1}{\PYZsq{}}\PY{l+s+s1}{r}\PY{l+s+s1}{\PYZsq{}}\PY{p}{,} 
                              \PY{n}{encoding}\PY{o}{=}\PY{l+s+s1}{\PYZsq{}}\PY{l+s+s1}{utf\PYZhy{}8}\PY{l+s+s1}{\PYZsq{}}\PY{p}{)} \PY{k}{as} \PY{n}{infile}\PY{p}{:}
                        \PY{n}{txt} \PY{o}{=} \PY{n}{infile}\PY{o}{.}\PY{n}{read}\PY{p}{(}\PY{p}{)}
                    \PY{n}{df} \PY{o}{=} \PY{n}{df}\PY{o}{.}\PY{n}{append}\PY{p}{(}\PY{p}{[}\PY{p}{[}\PY{n}{txt}\PY{p}{,} \PY{n}{labels}\PY{p}{[}\PY{n}{l}\PY{p}{]}\PY{p}{]}\PY{p}{]}\PY{p}{,} 
                                   \PY{n}{ignore\PYZus{}index}\PY{o}{=}\PY{k+kc}{True}\PY{p}{)}
                    \PY{n}{pbar}\PY{o}{.}\PY{n}{update}\PY{p}{(}\PY{p}{)}
        \PY{n}{df}\PY{o}{.}\PY{n}{columns} \PY{o}{=} \PY{p}{[}\PY{l+s+s1}{\PYZsq{}}\PY{l+s+s1}{review}\PY{l+s+s1}{\PYZsq{}}\PY{p}{,} \PY{l+s+s1}{\PYZsq{}}\PY{l+s+s1}{sentiment}\PY{l+s+s1}{\PYZsq{}}\PY{p}{]}
\end{Verbatim}


    \begin{Verbatim}[commandchars=\\\{\}]
0\% [\#\#\#\#\#\#\#\#\#\#\#\#\#\#\#\#\#\#\#\#\#\#\#\#\#\#\#\#\#\#] 100\% | ETA: 00:00:00
Total time elapsed: 00:01:20

    \end{Verbatim}

    In the preceding code, we first initialized a new progress bar object
\emph{pbar} with 50.000 iterations, which is the number of documents we
were going to read in. Using the nested \emph{for} loops, we iterated
over the \emph{train} and \emph{test} subdirectories in the main
\emph{aclImdb} directory and read the individual text files from the
\emph{pos} and \emph{neg} subdirectories that we eventually appended to
the \emph{df} pandas \emph{DataFrame}, together with an integer class
label (1 = positive and 0 = negative).

Since the class labels in the assembled dataset are sorted, we will now
shuffle \emph{DataFrame} using the \emph{permutation} function from the
\emph{np.random} submodule - this will be useful to split the dataset
into training and test sets in later sections when we will stream the
data from our local drive directly. For our own convenience, we will
also store the assembled and shuffled movie review dataset as a CSV
file:

    \begin{Verbatim}[commandchars=\\\{\}]
{\color{incolor}In [{\color{incolor}3}]:} \PY{k+kn}{import} \PY{n+nn}{numpy} \PY{k}{as} \PY{n+nn}{np}
        
        \PY{n}{np}\PY{o}{.}\PY{n}{random}\PY{o}{.}\PY{n}{seed}\PY{p}{(}\PY{l+m+mi}{0}\PY{p}{)}
        \PY{n}{df} \PY{o}{=} \PY{n}{df}\PY{o}{.}\PY{n}{reindex}\PY{p}{(}\PY{n}{np}\PY{o}{.}\PY{n}{random}\PY{o}{.}\PY{n}{permutation}\PY{p}{(}\PY{n}{df}\PY{o}{.}\PY{n}{index}\PY{p}{)}\PY{p}{)}
        \PY{n}{df}\PY{o}{.}\PY{n}{to\PYZus{}csv}\PY{p}{(}\PY{l+s+s1}{\PYZsq{}}\PY{l+s+s1}{movie\PYZus{}data.csv}\PY{l+s+s1}{\PYZsq{}}\PY{p}{,} \PY{n}{index}\PY{o}{=}\PY{k+kc}{False}\PY{p}{,} \PY{n}{encoding}\PY{o}{=}\PY{l+s+s1}{\PYZsq{}}\PY{l+s+s1}{utf\PYZhy{}8}\PY{l+s+s1}{\PYZsq{}}\PY{p}{)}
\end{Verbatim}


    Since we are going to use this dataset later in this chapter, let's
quickly confirm that we have successfully saved the data in the right
format by reading in the CSV and printing an excerpt of the first three
samples:

    \begin{Verbatim}[commandchars=\\\{\}]
{\color{incolor}In [{\color{incolor}4}]:} \PY{n}{df} \PY{o}{=} \PY{n}{pd}\PY{o}{.}\PY{n}{read\PYZus{}csv}\PY{p}{(}\PY{l+s+s1}{\PYZsq{}}\PY{l+s+s1}{movie\PYZus{}data.csv}\PY{l+s+s1}{\PYZsq{}}\PY{p}{,} \PY{n}{encoding}\PY{o}{=}\PY{l+s+s1}{\PYZsq{}}\PY{l+s+s1}{utf\PYZhy{}8}\PY{l+s+s1}{\PYZsq{}}\PY{p}{)}
        \PY{n}{df}\PY{o}{.}\PY{n}{head}\PY{p}{(}\PY{l+m+mi}{3}\PY{p}{)}
\end{Verbatim}


\begin{Verbatim}[commandchars=\\\{\}]
{\color{outcolor}Out[{\color{outcolor}4}]:}                                               review  sentiment
        0  This movie is the absolutely perfect way to ex{\ldots}          1
        1  I don't usually like TV movies, I reckon that {\ldots}          0
        2  First of all, I am not a huge fan of contempor{\ldots}          0
\end{Verbatim}
            
    \subsection{Introducing the bag-of-words
model}\label{introducing-the-bag-of-words-model}

    You may remember that we have to convert categorical data, such as text
or words, into a numerical form before we can pass it on to a machine
learning algorithm. In this section, we will introduce the
\textbf{bag-of-words}, which allows us to represent text as numerical
feature vectors. The idea behind the bag-of-words model is quite simple
and can be summarized as follows:

\begin{enumerate}
\def\labelenumi{\arabic{enumi}.}
\tightlist
\item
  We create a vocabulary of unique tokens - for example, words - from
  the entire set of documents.
\item
  We construct a feature vector from each document that contains the
  counts of how often each word occurs in the particular document.
\end{enumerate}

Since the unique words in each document represent only a small subset of
all the words in the bag-of-words vocabulary, the feature vectors will
mosly consist of zeros, which is why we call them \textbf{sparse}. Do
not worry if this sounds too abstract; in the following subsections, we
will walk through the process of creating a simple bag-of-words model
step-by-step.

    \subsection{Transforming words into feature
vectors}\label{transforming-words-into-feature-vectors}

    To construct a bag-of-words model based on the word counts in the
respective documents, we can use the \emph{CountVectorizer} class
implemented in scikit-learn. As we will see in the following code
section, \emph{CountVectorizer} takes an array of text data, which can
be documents or sentences, and constructs the bag-of-words model for us:

    \begin{Verbatim}[commandchars=\\\{\}]
{\color{incolor}In [{\color{incolor}5}]:} \PY{k+kn}{import} \PY{n+nn}{numpy} \PY{k}{as} \PY{n+nn}{np}
        \PY{k+kn}{from} \PY{n+nn}{sklearn}\PY{n+nn}{.}\PY{n+nn}{feature\PYZus{}extraction}\PY{n+nn}{.}\PY{n+nn}{text} \PY{k}{import} \PY{n}{CountVectorizer}
        
        \PY{n}{count} \PY{o}{=} \PY{n}{CountVectorizer}\PY{p}{(}\PY{p}{)}
        \PY{n}{docs} \PY{o}{=} \PY{n}{np}\PY{o}{.}\PY{n}{array}\PY{p}{(}\PY{p}{[}\PY{l+s+s1}{\PYZsq{}}\PY{l+s+s1}{The sun is shining}\PY{l+s+s1}{\PYZsq{}}\PY{p}{,} 
                         \PY{l+s+s1}{\PYZsq{}}\PY{l+s+s1}{The weather is sweet}\PY{l+s+s1}{\PYZsq{}}\PY{p}{,} 
                         \PY{l+s+s1}{\PYZsq{}}\PY{l+s+s1}{The sun is shining and the weather is sweet}\PY{l+s+s1}{\PYZsq{}}\PY{p}{]}\PY{p}{)}
        \PY{n}{bag} \PY{o}{=} \PY{n}{count}\PY{o}{.}\PY{n}{fit\PYZus{}transform}\PY{p}{(}\PY{n}{docs}\PY{p}{)}
\end{Verbatim}


    By calling the \emph{fit\_transform} method on \emph{CountVectorizer},
we constructed the vocabulary of the bag-of-words model and transformed
the following three sentences into sparse feature vectors:

\begin{itemize}
\tightlist
\item
  'The sun is shining'
\item
  'The weather is sweet'
\item
  'The sun is shining and the weather is sweet'
\end{itemize}

Now let's print the contents of the vocabulary to get a better
understanding of the underlying concepts:

    \begin{Verbatim}[commandchars=\\\{\}]
{\color{incolor}In [{\color{incolor}6}]:} \PY{n+nb}{print}\PY{p}{(}\PY{n}{count}\PY{o}{.}\PY{n}{vocabulary\PYZus{}}\PY{p}{)}
\end{Verbatim}


    \begin{Verbatim}[commandchars=\\\{\}]
\{'the': 5, 'sun': 3, 'is': 1, 'shining': 2, 'weather': 6, 'sweet': 4, 'and': 0\}

    \end{Verbatim}

    As we can see from executing the preceding command, the vocabulary is
stored in a Python dictionary that maps the unique words to integer
indices. Next, let's print the feature vectors that we just created:

    \begin{Verbatim}[commandchars=\\\{\}]
{\color{incolor}In [{\color{incolor}7}]:} \PY{n+nb}{print}\PY{p}{(}\PY{n}{bag}\PY{o}{.}\PY{n}{toarray}\PY{p}{(}\PY{p}{)}\PY{p}{)}
\end{Verbatim}


    \begin{Verbatim}[commandchars=\\\{\}]
[[0 1 1 1 0 1 0]
 [0 1 0 0 1 1 1]
 [1 2 1 1 1 2 1]]

    \end{Verbatim}

    Each index position in the feature vectors shown here corresponds to the
integer values that are stored as dictionary items in the
\emph{CountVectorizer} vocabulary. For example, the first feature at
index position 0 resembles the count of the word \emph{'and'}, which
only occurs in the last document, and the word \emph{'is'}, at index
position 1 (the second feature in the document vectors), occurs in all
three sentences. These values in the feature vectors are also called the
\textbf{raw term frequencies (tf(t, d))} - the number of times a term
\emph{t} occurs in a document \emph{d}.

The sequence of items in the bag-of-words model that we just created is
also called the \textbf{1-gram} or \textbf{unigram} model - each item or
token in the vocabulary represents a single word. More generally, the
contiguous sequences of item in NLP - words, letters, or symbols - are
also called \textbf{n-grams}. The choice of the number \emph{n} in the
n-gram model depends on the particular application; for example, a study
by Kanaris and other revealedthat n-grams of size 3 or 4 yield good
performance in anti-spam filtering of email messages. To summarize the
concept of the n-gram representation, the 1-gram and 2-gram
representations of our first document 'the sun is shining' would be
constructed as follows:

\begin{itemize}
\tightlist
\item
  1-gram: 'the', 'sun', 'is', 'shining'
\item
  2-gram: 'the sun', 'sun is', 'is shining'
\end{itemize}

The \emph{CountVectorizer} class in scikit-learn allows us to use
different n-gram models via its \emph{ngram\_range} parameter. While a
1-gram representation is used by default, we could switch to a 2-gram
representation by initializing a new \emph{CountVectorizer} instance
with \emph{ngram\_range=(2,2)}.

    \subsection{Assessing word relevancy via term frequency-inverse document
frequency}\label{assessing-word-relevancy-via-term-frequency-inverse-document-frequency}

    When we are analyzing text data, we often encounter words that occur
across multiple documents from both classes. These frequently occurring
words typically do not contain useful or discriminatory information. In
this subsection, we will learn about a useful technique called
\textbf{term frequency-inverse document frequency (tf-idf)} that can be
used to downweight these frequently occurring words in the feature
vectors. The tf-idf can be defined as the product of the term frequency
and the inverse document frequency:

\[\text{tf-idf}(t,d) = tf(t,d) \times idf(t,d)\]

Here, the \(tf(t,d)\) is the term frequency that we introduced in the
previous section, and \(idf(t,d)\) is the inverse document frequency and
can be calculated as follows:

\[idf(t,d) = log\frac{n_d}{1+df(d,t)}\]

Here, \(n_d\) is the total number of documents, and \(df(d,t)\) is the
number of documents \(d\) that contain the term \(t\). Note that adding
the constant 1 to the denominator is optional and serves the purpose of
assigning a non-zero value to terms that occur in all training samples;
the \emph{log} is used to ensure that low document frequencies are not
given too much weight.

The scikit-learn library implements yet another transformer, the
\emph{TfidfTransformer} class, that takes the raw term frequencies from
the \emph{CountVectorizer} class as input and transforms them into
tf-idfs:

    \begin{Verbatim}[commandchars=\\\{\}]
{\color{incolor}In [{\color{incolor}8}]:} \PY{k+kn}{from} \PY{n+nn}{sklearn}\PY{n+nn}{.}\PY{n+nn}{feature\PYZus{}extraction}\PY{n+nn}{.}\PY{n+nn}{text} \PY{k}{import} \PY{n}{TfidfTransformer}
        
        \PY{n}{tfidf} \PY{o}{=} \PY{n}{TfidfTransformer}\PY{p}{(}\PY{n}{use\PYZus{}idf}\PY{o}{=}\PY{k+kc}{True}\PY{p}{,} \PY{n}{norm}\PY{o}{=}\PY{l+s+s1}{\PYZsq{}}\PY{l+s+s1}{l2}\PY{l+s+s1}{\PYZsq{}}\PY{p}{,} 
                                 \PY{n}{smooth\PYZus{}idf}\PY{o}{=}\PY{k+kc}{True}\PY{p}{)}
        \PY{n}{np}\PY{o}{.}\PY{n}{set\PYZus{}printoptions}\PY{p}{(}\PY{n}{precision}\PY{o}{=}\PY{l+m+mi}{2}\PY{p}{)}
        \PY{n+nb}{print}\PY{p}{(}\PY{n}{tfidf}\PY{o}{.}\PY{n}{fit\PYZus{}transform}\PY{p}{(}\PY{n}{count}\PY{o}{.}\PY{n}{fit\PYZus{}transform}\PY{p}{(}\PY{n}{docs}\PY{p}{)}\PY{p}{)}\PY{o}{.}\PY{n}{toarray}\PY{p}{(}\PY{p}{)}\PY{p}{)}
\end{Verbatim}


    \begin{Verbatim}[commandchars=\\\{\}]
[[0.   0.43 0.56 0.56 0.   0.43 0.  ]
 [0.   0.43 0.   0.   0.56 0.43 0.56]
 [0.4  0.48 0.31 0.31 0.31 0.48 0.31]]

    \end{Verbatim}

    As we saw in the previous subsection, the word \emph{'is'} had the
largest term frequency in the third document, being the most frequently
occurring word. However, after transforming the same feature vector into
tf-idfs, we see that the work \emph{'is'} s now associated with a
relatively small tf-idf (0.48) in the third document, since it is also
present in the first and second document and thus is unlikely to contain
any useful discriminatory information.

However, if we would manually calculated the tf-idfs of the individual
terms in our feature vector, we would notice that
\emph{TfidfTransformer} calculates the tf-idfs slightly different
compared to the standard textbook equations that we defined previously.

While it is also more typical to normalize the raw terms frequencies
before calculating the tf-idfs, \emph{TfidfTranformer} class normalizes
the tf-idfs directly. By default (norm = 'l2'), scikit-learn's
\emph{TfidfTransformer} applies the L2-normalization, which returns a
vector of length 1 by dividing an un-normalized feature vector \emph{v}
by its L2-norm.

    \subsection{Cleaning text data}\label{cleaning-text-data}

    In the previous subsections, we learned about the bag-of-words model,
term frequencies, and tf-idfs. However, the first important step -
before we build our bag-of-words model - is to clean the text data by
stripping it of all unwanted characters. To illustrate why this is
important, let's display the last 50 characters from the first document
in the reshuffled movie review dataset:

    \begin{Verbatim}[commandchars=\\\{\}]
{\color{incolor}In [{\color{incolor}9}]:} \PY{n}{df}\PY{o}{.}\PY{n}{loc}\PY{p}{[}\PY{l+m+mi}{0}\PY{p}{,} \PY{l+s+s1}{\PYZsq{}}\PY{l+s+s1}{review}\PY{l+s+s1}{\PYZsq{}}\PY{p}{]}\PY{p}{[}\PY{o}{\PYZhy{}}\PY{l+m+mi}{50}\PY{p}{:}\PY{p}{:}\PY{p}{]}
\end{Verbatim}


\begin{Verbatim}[commandchars=\\\{\}]
{\color{outcolor}Out[{\color{outcolor}9}]:} 'your hearts forever :-).<br /><br />Regards Klavs.'
\end{Verbatim}
            
    As we can see here, the text contains HTML markup as well as punctuation
and other non-letter characters. While HTML markup does not contain much
useful semantics, punctuation marks can represent useful, additional
information in certai NLP context. However, for simplicity, we will now
remove all punctuation marks except for emoticon characters such as :)
since those are certainly useful for sentiment analysis. To accomplish
this task, we will use Python's \textbf{regular expression (regex)}
library, \emph{re}, as shown here:

    \begin{Verbatim}[commandchars=\\\{\}]
{\color{incolor}In [{\color{incolor}10}]:} \PY{k+kn}{import} \PY{n+nn}{re}
         
         \PY{k}{def} \PY{n+nf}{preprocessor}\PY{p}{(}\PY{n}{text}\PY{p}{)}\PY{p}{:}
             \PY{n}{text} \PY{o}{=} \PY{n}{re}\PY{o}{.}\PY{n}{sub}\PY{p}{(}\PY{l+s+s1}{\PYZsq{}}\PY{l+s+s1}{\PYZlt{}[\PYZca{}\PYZgt{}]*\PYZgt{}}\PY{l+s+s1}{\PYZsq{}}\PY{p}{,} \PY{l+s+s1}{\PYZsq{}}\PY{l+s+s1}{\PYZsq{}}\PY{p}{,} \PY{n}{text}\PY{p}{)}
             \PY{n}{emoticons} \PY{o}{=} \PY{n}{re}\PY{o}{.}\PY{n}{findall}\PY{p}{(}\PY{l+s+s1}{\PYZsq{}}\PY{l+s+s1}{(?::|;|=)(?:\PYZhy{})?(?:}\PY{l+s+s1}{\PYZbs{}}\PY{l+s+s1}{)|}\PY{l+s+s1}{\PYZbs{}}\PY{l+s+s1}{(|D|P)}\PY{l+s+s1}{\PYZsq{}}\PY{p}{,} \PY{n}{text}\PY{p}{)}
             \PY{n}{text} \PY{o}{=} \PY{p}{(}\PY{n}{re}\PY{o}{.}\PY{n}{sub}\PY{p}{(}\PY{l+s+s1}{\PYZsq{}}\PY{l+s+s1}{[}\PY{l+s+s1}{\PYZbs{}}\PY{l+s+s1}{W]+}\PY{l+s+s1}{\PYZsq{}}\PY{p}{,} \PY{l+s+s1}{\PYZsq{}}\PY{l+s+s1}{ }\PY{l+s+s1}{\PYZsq{}}\PY{p}{,} \PY{n}{text}\PY{o}{.}\PY{n}{lower}\PY{p}{(}\PY{p}{)}\PY{p}{)} \PY{o}{+} \PY{l+s+s1}{\PYZsq{}}\PY{l+s+s1}{ }\PY{l+s+s1}{\PYZsq{}}\PY{o}{.}\PY{n}{join}\PY{p}{(}\PY{n}{emoticons}\PY{p}{)}\PY{o}{.}\PY{n}{replace}\PY{p}{(}\PY{l+s+s1}{\PYZsq{}}\PY{l+s+s1}{\PYZhy{}}\PY{l+s+s1}{\PYZsq{}}\PY{p}{,} \PY{l+s+s1}{\PYZsq{}}\PY{l+s+s1}{\PYZsq{}}\PY{p}{)}\PY{p}{)}
             \PY{k}{return} \PY{n}{text}
\end{Verbatim}


    Via the first regex in the preceding code section, we tried to remove
all of the HTML markup from the movie reviews. Although many programmers
generally advice against the use of regex to parse HTML, this regex
should be sufficient to \emph{clean} this particular dataset. After we
removed the HTML markup, we used a slightly more complex regex to find
emoticons, which we temporarily stored as \emph{emoticons}. Next, we
removed all non-word characters from the text via the last regex and
converted the text into lowercase characters.

Eventually, we added the temporarily stored \emph{emoticons} to the end
of the processed document string. Additionally, we remove the
\emph{nose} character (-) from the emoticons for consistency.

Although the addition of the emoticon characters to the end of the
cleaned document string may not look like the most elegant approach, we
shall note that the order of the words does not matter in our
bag-of-words model if our vocabulary consists of only one-word tokens.
But before we talk more about the splitting of documents into individual
term, words, or tokens, let's confirm that our preprocessor works
correctly:

    \begin{Verbatim}[commandchars=\\\{\}]
{\color{incolor}In [{\color{incolor}11}]:} \PY{n}{preprocessor}\PY{p}{(}\PY{n}{df}\PY{o}{.}\PY{n}{loc}\PY{p}{[}\PY{l+m+mi}{0}\PY{p}{,} \PY{l+s+s1}{\PYZsq{}}\PY{l+s+s1}{review}\PY{l+s+s1}{\PYZsq{}}\PY{p}{]}\PY{p}{[}\PY{o}{\PYZhy{}}\PY{l+m+mi}{50}\PY{p}{:}\PY{p}{]}\PY{p}{)}
\end{Verbatim}


\begin{Verbatim}[commandchars=\\\{\}]
{\color{outcolor}Out[{\color{outcolor}11}]:} 'your hearts forever regards klavs :)'
\end{Verbatim}
            
    Lastly, since we will make use of the cleaned text data over and over
again during the next sections, let us now apply our \emph{preprocessor}
function to all the movie reviews in our \emph{DataFrame}:

    \begin{Verbatim}[commandchars=\\\{\}]
{\color{incolor}In [{\color{incolor}12}]:} \PY{n}{df}\PY{p}{[}\PY{l+s+s1}{\PYZsq{}}\PY{l+s+s1}{review}\PY{l+s+s1}{\PYZsq{}}\PY{p}{]} \PY{o}{=} \PY{n}{df}\PY{p}{[}\PY{l+s+s1}{\PYZsq{}}\PY{l+s+s1}{review}\PY{l+s+s1}{\PYZsq{}}\PY{p}{]}\PY{o}{.}\PY{n}{apply}\PY{p}{(}\PY{n}{preprocessor}\PY{p}{)}
\end{Verbatim}


    \subsection{Preprocessing documents into
tokens}\label{preprocessing-documents-into-tokens}

    After successfully preparing the movie review dataset, we now need to
think about how to split the text corpora into individual elements. One
way to \emph{tokenize} documents is to split them into individual words
by splitting the cleaned documents at its whitespace characters:

    \begin{Verbatim}[commandchars=\\\{\}]
{\color{incolor}In [{\color{incolor}13}]:} \PY{k}{def} \PY{n+nf}{tokenizer}\PY{p}{(}\PY{n}{text}\PY{p}{)}\PY{p}{:}
             \PY{k}{return} \PY{n}{text}\PY{o}{.}\PY{n}{split}\PY{p}{(}\PY{p}{)}
         \PY{n}{tokenizer}\PY{p}{(}\PY{l+s+s1}{\PYZsq{}}\PY{l+s+s1}{runners like running and thus they run}\PY{l+s+s1}{\PYZsq{}}\PY{p}{)}
\end{Verbatim}


\begin{Verbatim}[commandchars=\\\{\}]
{\color{outcolor}Out[{\color{outcolor}13}]:} ['runners', 'like', 'running', 'and', 'thus', 'they', 'run']
\end{Verbatim}
            
    In the context of tokenization, another useful tecnique is \textbf{word
stemming}, which is the process of transforming a work into its root
form. It allows us to map related words to the same steam. The original
stemming algorithm was developed by Martin F. Porter in 1979 and is
hence known as the \textbf{Porter stemmer} algorithm. The
\textbf{Natural Language Toolkit (NLTK)} for Python implements the
Porter stemming algorithm, which we will use in the following code
section. In order to install the NLTK, you can simply execute \emph{pip
install nltk}.

The following code shows how to use the Porter stemming algorithm:

    \begin{Verbatim}[commandchars=\\\{\}]
{\color{incolor}In [{\color{incolor}14}]:} \PY{k+kn}{from} \PY{n+nn}{nltk}\PY{n+nn}{.}\PY{n+nn}{stem}\PY{n+nn}{.}\PY{n+nn}{porter} \PY{k}{import} \PY{n}{PorterStemmer}
         
         \PY{n}{porter} \PY{o}{=} \PY{n}{PorterStemmer}\PY{p}{(}\PY{p}{)}
         \PY{k}{def} \PY{n+nf}{tokenizer\PYZus{}porter}\PY{p}{(}\PY{n}{text}\PY{p}{)}\PY{p}{:}
             \PY{k}{return} \PY{p}{[}\PY{n}{porter}\PY{o}{.}\PY{n}{stem}\PY{p}{(}\PY{n}{word}\PY{p}{)} \PY{k}{for} \PY{n}{word} \PY{o+ow}{in} \PY{n}{text}\PY{o}{.}\PY{n}{split}\PY{p}{(}\PY{p}{)}\PY{p}{]}
         \PY{n}{tokenizer\PYZus{}porter}\PY{p}{(}\PY{l+s+s1}{\PYZsq{}}\PY{l+s+s1}{runners like to running and thus they run}\PY{l+s+s1}{\PYZsq{}}\PY{p}{)}
\end{Verbatim}


\begin{Verbatim}[commandchars=\\\{\}]
{\color{outcolor}Out[{\color{outcolor}14}]:} ['runner', 'like', 'to', 'run', 'and', 'thu', 'they', 'run']
\end{Verbatim}
            
    Using the \emph{PorterStemmer} from the \emph{nltk} package, we modified
our \emph{tokenizer} function to reduce words to their root form, which
was illustrated by the simple preceding example where the word
\emph{'running'} was \emph{stemmed} to its root form \emph{'run'}.

The Porter stemming algorithm is probably the oldest and simplest
stemming algorithm. Other popular stemming algorithms include the newer
\textbf{Snowball stemmer} and the \textbf{Lancaster stemmer}, which is
fater but also more aggressive then the Porter stemmer.

While stemming can create non-real words, such as \emph{'thu'} (from
\emph{'thus'}), as shown in the previous example, a technique called
\textbf{lemmatization} aims to obtain the canonical (grammatically
correct) forms of individual words - the so-called \textbf{lemmas}.
However, lemmatization is computationally more difficulty and expensive
compared to stemming and, in practice, it has been observed that
stemming and lemmatization have little impact on the performance of text
classification.

Before we jump into the next section, where we will train a machine
learning model using the bag-of-words model, let's briefly talk about
another useful topic called \textbf{stop-word removal}. Stop-words are
simply those words that are extremely common in all sorts of texts and
probably bear no (or only little) useful information that can be used to
distinguish between different classes of documents. Examples of
stop-words are \emph{is}, \emph{and}, \emph{has}, and \emph{like}.
Removing stop-words can be useful if we are working with raw or
normalized term frequencies rather than tf-idfs, which are already
downweighting frequently occurring words.

In order to remove stop-words from the movie reviews, we will use the
set of 127 English stop-words that is available from the NLTK library,
which can be obtained by calling the \emph{nltk.download} function:

    \begin{Verbatim}[commandchars=\\\{\}]
{\color{incolor}In [{\color{incolor}15}]:} \PY{k+kn}{import} \PY{n+nn}{nltk}
         
         \PY{n}{nltk}\PY{o}{.}\PY{n}{download}\PY{p}{(}\PY{l+s+s1}{\PYZsq{}}\PY{l+s+s1}{stopwords}\PY{l+s+s1}{\PYZsq{}}\PY{p}{)}
\end{Verbatim}


    \begin{Verbatim}[commandchars=\\\{\}]
[nltk\_data] Downloading package stopwords to /home/wesley/nltk\_data{\ldots}
[nltk\_data]   Package stopwords is already up-to-date!

    \end{Verbatim}

\begin{Verbatim}[commandchars=\\\{\}]
{\color{outcolor}Out[{\color{outcolor}15}]:} True
\end{Verbatim}
            
    After we download the stop-words set, we can load and apply the English
stop-word set as follows:

    \begin{Verbatim}[commandchars=\\\{\}]
{\color{incolor}In [{\color{incolor}16}]:} \PY{k+kn}{from} \PY{n+nn}{nltk}\PY{n+nn}{.}\PY{n+nn}{corpus} \PY{k}{import} \PY{n}{stopwords}
         
         \PY{n}{stop} \PY{o}{=} \PY{n}{stopwords}\PY{o}{.}\PY{n}{words}\PY{p}{(}\PY{l+s+s1}{\PYZsq{}}\PY{l+s+s1}{english}\PY{l+s+s1}{\PYZsq{}}\PY{p}{)}
         \PY{p}{[}\PY{n}{w} \PY{k}{for} \PY{n}{w} \PY{o+ow}{in} \PY{n}{tokenizer\PYZus{}porter}\PY{p}{(}\PY{l+s+s1}{\PYZsq{}}\PY{l+s+s1}{a runner likes running and runs a lot}\PY{l+s+s1}{\PYZsq{}}\PY{p}{)}\PY{p}{[}\PY{o}{\PYZhy{}}\PY{l+m+mi}{10}\PY{p}{:}\PY{p}{]}  \PY{k}{if} \PY{n}{w} \PY{o+ow}{not} \PY{o+ow}{in} \PY{n}{stop}\PY{p}{]}
\end{Verbatim}


\begin{Verbatim}[commandchars=\\\{\}]
{\color{outcolor}Out[{\color{outcolor}16}]:} ['runner', 'like', 'run', 'run', 'lot']
\end{Verbatim}
            
    \subsection{Training a logistic regression model for document
classification}\label{training-a-logistic-regression-model-for-document-classification}

    In this section, we will train a logistic regression model to classify
the movie reviews into positive and negative reviews. First, we will
divide the \emph{DataFrame} of cleaned text documents into 25.000
documents for training and 25.000 documents for testing:

    \begin{Verbatim}[commandchars=\\\{\}]
{\color{incolor}In [{\color{incolor}17}]:} \PY{n}{X\PYZus{}train} \PY{o}{=} \PY{n}{df}\PY{o}{.}\PY{n}{loc}\PY{p}{[}\PY{p}{:}\PY{l+m+mi}{25000}\PY{p}{,} \PY{l+s+s1}{\PYZsq{}}\PY{l+s+s1}{review}\PY{l+s+s1}{\PYZsq{}}\PY{p}{]}\PY{o}{.}\PY{n}{values}
         \PY{n}{y\PYZus{}train} \PY{o}{=} \PY{n}{df}\PY{o}{.}\PY{n}{loc}\PY{p}{[}\PY{p}{:}\PY{l+m+mi}{25000}\PY{p}{,} \PY{l+s+s1}{\PYZsq{}}\PY{l+s+s1}{sentiment}\PY{l+s+s1}{\PYZsq{}}\PY{p}{]}\PY{o}{.}\PY{n}{values}
         \PY{n}{X\PYZus{}test} \PY{o}{=} \PY{n}{df}\PY{o}{.}\PY{n}{loc}\PY{p}{[}\PY{l+m+mi}{25000}\PY{p}{:}\PY{p}{,} \PY{l+s+s1}{\PYZsq{}}\PY{l+s+s1}{review}\PY{l+s+s1}{\PYZsq{}}\PY{p}{]}\PY{o}{.}\PY{n}{values}
         \PY{n}{y\PYZus{}test} \PY{o}{=} \PY{n}{df}\PY{o}{.}\PY{n}{loc}\PY{p}{[}\PY{l+m+mi}{25000}\PY{p}{:}\PY{p}{,} \PY{l+s+s1}{\PYZsq{}}\PY{l+s+s1}{sentiment}\PY{l+s+s1}{\PYZsq{}}\PY{p}{]}\PY{o}{.}\PY{n}{values}
\end{Verbatim}


    Next, we will use a \emph{GridSearchCV} object to find the optimal set
of parameters for our logistic regression model using 5-fold stratified
cross-validation:

    \begin{Verbatim}[commandchars=\\\{\}]
{\color{incolor}In [{\color{incolor}18}]:} \PY{k+kn}{from} \PY{n+nn}{sklearn}\PY{n+nn}{.}\PY{n+nn}{model\PYZus{}selection} \PY{k}{import} \PY{n}{GridSearchCV}
         \PY{k+kn}{from} \PY{n+nn}{sklearn}\PY{n+nn}{.}\PY{n+nn}{pipeline} \PY{k}{import} \PY{n}{Pipeline}
         \PY{k+kn}{from} \PY{n+nn}{sklearn}\PY{n+nn}{.}\PY{n+nn}{linear\PYZus{}model} \PY{k}{import} \PY{n}{LogisticRegression}
         \PY{k+kn}{from} \PY{n+nn}{sklearn}\PY{n+nn}{.}\PY{n+nn}{feature\PYZus{}extraction}\PY{n+nn}{.}\PY{n+nn}{text} \PY{k}{import} \PY{n}{TfidfVectorizer}
         
         \PY{n}{tfidf} \PY{o}{=} \PY{n}{TfidfVectorizer}\PY{p}{(}\PY{n}{strip\PYZus{}accents}\PY{o}{=}\PY{k+kc}{None}\PY{p}{,} \PY{n}{lowercase}\PY{o}{=}\PY{k+kc}{False}\PY{p}{,} 
                                 \PY{n}{preprocessor}\PY{o}{=}\PY{k+kc}{None}\PY{p}{)}
         \PY{n}{param\PYZus{}grid} \PY{o}{=} \PY{p}{[}\PY{p}{\PYZob{}}\PY{l+s+s1}{\PYZsq{}}\PY{l+s+s1}{vect\PYZus{}\PYZus{}ngram\PYZus{}range}\PY{l+s+s1}{\PYZsq{}}\PY{p}{:} \PY{p}{[}\PY{p}{(}\PY{l+m+mi}{1}\PY{p}{,} \PY{l+m+mi}{1}\PY{p}{)}\PY{p}{]}\PY{p}{,} 
                        \PY{l+s+s1}{\PYZsq{}}\PY{l+s+s1}{vect\PYZus{}\PYZus{}stop\PYZus{}words}\PY{l+s+s1}{\PYZsq{}}\PY{p}{:} \PY{p}{[}\PY{n}{stop}\PY{p}{,} \PY{k+kc}{None}\PY{p}{]}\PY{p}{,} 
                        \PY{l+s+s1}{\PYZsq{}}\PY{l+s+s1}{vect\PYZus{}\PYZus{}tokenizer}\PY{l+s+s1}{\PYZsq{}}\PY{p}{:} \PY{p}{[}\PY{n}{tokenizer}\PY{p}{,} \PY{n}{tokenizer\PYZus{}porter}\PY{p}{]}\PY{p}{,} 
                        \PY{l+s+s1}{\PYZsq{}}\PY{l+s+s1}{clf\PYZus{}\PYZus{}penalty}\PY{l+s+s1}{\PYZsq{}}\PY{p}{:} \PY{p}{[}\PY{l+s+s1}{\PYZsq{}}\PY{l+s+s1}{l1}\PY{l+s+s1}{\PYZsq{}}\PY{p}{,} \PY{l+s+s1}{\PYZsq{}}\PY{l+s+s1}{l2}\PY{l+s+s1}{\PYZsq{}}\PY{p}{]}\PY{p}{,} 
                        \PY{l+s+s1}{\PYZsq{}}\PY{l+s+s1}{clf\PYZus{}\PYZus{}C}\PY{l+s+s1}{\PYZsq{}}\PY{p}{:} \PY{p}{[}\PY{l+m+mf}{1.0}\PY{p}{,} \PY{l+m+mf}{10.0}\PY{p}{,} \PY{l+m+mf}{100.0}\PY{p}{]}\PY{p}{\PYZcb{}}\PY{p}{,} 
                       \PY{p}{\PYZob{}}\PY{l+s+s1}{\PYZsq{}}\PY{l+s+s1}{vect\PYZus{}\PYZus{}ngram\PYZus{}range}\PY{l+s+s1}{\PYZsq{}}\PY{p}{:} \PY{p}{[}\PY{p}{(}\PY{l+m+mi}{1}\PY{p}{,} \PY{l+m+mi}{1}\PY{p}{)}\PY{p}{]}\PY{p}{,} 
                        \PY{l+s+s1}{\PYZsq{}}\PY{l+s+s1}{vect\PYZus{}\PYZus{}stop\PYZus{}words}\PY{l+s+s1}{\PYZsq{}}\PY{p}{:} \PY{p}{[}\PY{n}{stop}\PY{p}{,} \PY{k+kc}{None}\PY{p}{]}\PY{p}{,} 
                        \PY{l+s+s1}{\PYZsq{}}\PY{l+s+s1}{vect\PYZus{}\PYZus{}tokenizer}\PY{l+s+s1}{\PYZsq{}}\PY{p}{:} \PY{p}{[}\PY{n}{tokenizer}\PY{p}{,} \PY{n}{tokenizer\PYZus{}porter}\PY{p}{]}\PY{p}{,} 
                        \PY{l+s+s1}{\PYZsq{}}\PY{l+s+s1}{vect\PYZus{}\PYZus{}use\PYZus{}idf}\PY{l+s+s1}{\PYZsq{}}\PY{p}{:} \PY{p}{[}\PY{k+kc}{False}\PY{p}{]}\PY{p}{,} 
                        \PY{l+s+s1}{\PYZsq{}}\PY{l+s+s1}{vect\PYZus{}\PYZus{}norm}\PY{l+s+s1}{\PYZsq{}}\PY{p}{:} \PY{p}{[}\PY{k+kc}{None}\PY{p}{]}\PY{p}{,} 
                        \PY{l+s+s1}{\PYZsq{}}\PY{l+s+s1}{clf\PYZus{}\PYZus{}penalty}\PY{l+s+s1}{\PYZsq{}}\PY{p}{:} \PY{p}{[}\PY{l+s+s1}{\PYZsq{}}\PY{l+s+s1}{l1}\PY{l+s+s1}{\PYZsq{}}\PY{p}{,} \PY{l+s+s1}{\PYZsq{}}\PY{l+s+s1}{l2}\PY{l+s+s1}{\PYZsq{}}\PY{p}{]}\PY{p}{,} 
                        \PY{l+s+s1}{\PYZsq{}}\PY{l+s+s1}{clf\PYZus{}\PYZus{}C}\PY{l+s+s1}{\PYZsq{}}\PY{p}{:} \PY{p}{[}\PY{l+m+mf}{1.0}\PY{p}{,} \PY{l+m+mf}{10.0}\PY{p}{,} \PY{l+m+mf}{100.0}\PY{p}{]}\PY{p}{\PYZcb{}}\PY{p}{]}
         \PY{n}{lr\PYZus{}tfidf} \PY{o}{=} \PY{n}{Pipeline}\PY{p}{(}\PY{p}{[}\PY{p}{(}\PY{l+s+s1}{\PYZsq{}}\PY{l+s+s1}{vect}\PY{l+s+s1}{\PYZsq{}}\PY{p}{,} \PY{n}{tfidf}\PY{p}{)}\PY{p}{,} 
                              \PY{p}{(}\PY{l+s+s1}{\PYZsq{}}\PY{l+s+s1}{clf}\PY{l+s+s1}{\PYZsq{}}\PY{p}{,} \PY{n}{LogisticRegression}\PY{p}{(}\PY{n}{random\PYZus{}state}\PY{o}{=}\PY{l+m+mi}{0}\PY{p}{)}\PY{p}{)}\PY{p}{]}\PY{p}{)}
         \PY{n}{gs\PYZus{}lr\PYZus{}tfidf} \PY{o}{=} \PY{n}{GridSearchCV}\PY{p}{(}\PY{n}{lr\PYZus{}tfidf}\PY{p}{,} \PY{n}{param\PYZus{}grid}\PY{p}{,} 
                                    \PY{n}{scoring}\PY{o}{=}\PY{l+s+s1}{\PYZsq{}}\PY{l+s+s1}{accuracy}\PY{l+s+s1}{\PYZsq{}}\PY{p}{,} \PY{n}{cv}\PY{o}{=}\PY{l+m+mi}{5}\PY{p}{,} \PY{n}{verbose}\PY{o}{=}\PY{l+m+mi}{1}\PY{p}{,} 
                                    \PY{n}{n\PYZus{}jobs}\PY{o}{=}\PY{o}{\PYZhy{}}\PY{l+m+mi}{1}\PY{p}{)}
         \PY{n}{gs\PYZus{}lr\PYZus{}tfidf}\PY{o}{.}\PY{n}{fit}\PY{p}{(}\PY{n}{X\PYZus{}train}\PY{p}{,} \PY{n}{y\PYZus{}train}\PY{p}{)}
\end{Verbatim}


    \begin{Verbatim}[commandchars=\\\{\}]
Fitting 5 folds for each of 48 candidates, totalling 240 fits

    \end{Verbatim}

    \begin{Verbatim}[commandchars=\\\{\}]
[Parallel(n\_jobs=-1)]: Done  34 tasks      | elapsed: 13.1min
[Parallel(n\_jobs=-1)]: Done 184 tasks      | elapsed: 73.0min
[Parallel(n\_jobs=-1)]: Done 240 out of 240 | elapsed: 98.9min finished

    \end{Verbatim}

\begin{Verbatim}[commandchars=\\\{\}]
{\color{outcolor}Out[{\color{outcolor}18}]:} GridSearchCV(cv=5, error\_score='raise',
                estimator=Pipeline(memory=None,
              steps=[('vect', TfidfVectorizer(analyzer='word', binary=False, decode\_error='strict',
                 dtype=<class 'numpy.int64'>, encoding='utf-8', input='content',
                 lowercase=False, max\_df=1.0, max\_features=None, min\_df=1,
                 ngram\_range=(1, 1), norm='l2', preprocessor=None, smooth\_idf=True,
          {\ldots}nalty='l2', random\_state=0, solver='liblinear', tol=0.0001,
                   verbose=0, warm\_start=False))]),
                fit\_params=None, iid=True, n\_jobs=-1,
                param\_grid=[\{'vect\_\_ngram\_range': [(1, 1)], 'vect\_\_stop\_words': [['i', 'me', 'my', 'myself', 'we', 'our', 'ours', 'ourselves', 'you', "you're", "you've", "you'll", "you'd", 'your', 'yours', 'yourself', 'yourselves', 'he', 'him', 'his', 'himself', 'she', "she's", 'her', 'hers', 'herself', 'it', "it's{\ldots}se\_idf': [False], 'vect\_\_norm': [None], 'clf\_\_penalty': ['l1', 'l2'], 'clf\_\_C': [1.0, 10.0, 100.0]\}],
                pre\_dispatch='2*n\_jobs', refit=True, return\_train\_score='warn',
                scoring='accuracy', verbose=1)
\end{Verbatim}
            
    When we initialized the \emph{GridSearchCV} object and its parameter
grid using the preceding code, we restricted ourselves to a limited
number of parameter combinations, since the number of feature vectors,
as well as the large vocabulary, can make the grid search
computationally quite expensive. Using a standard desktop computer, our
grid search may take up to 40 minutes to complete.

In the previous code example, we replaced \emph{CountVectorizer} and
\emph{TfidfTransformer} from the previous subsection with
\emph{TfidfVectorizer}, which combines the latter transformer objects.
Our \emph{param\_grid} consisted of two parameter dictionaries. In the
first dictionary, we used the \emph{TfidfVectorizer} with its default
settings (\emph{use\_idf=True}, \emph{smooth\_idf=True}, and
\emph{norm='l2'}) to calculate the tf-idfs; in the second dictionary, we
set those parameters to \emph{use\_idf=False}, \emph{smooth\_idf=False},
and \emph{norm=None} in order to train a model based on raw term
frequencies. Furthermore, for the logistic regression classifier itself,
we trained models using L2 and L1 regularization via the penalty
parameter and compared different regularization strengths by defining a
range of values for the inverse-regularization parameter C.

After the grid search has finished, we can print the best parameters
set:

    \begin{Verbatim}[commandchars=\\\{\}]
{\color{incolor}In [{\color{incolor}20}]:} \PY{n+nb}{print}\PY{p}{(}\PY{l+s+s1}{\PYZsq{}}\PY{l+s+s1}{Best parameter set: }\PY{l+s+si}{\PYZpc{}s}\PY{l+s+s1}{\PYZsq{}} \PY{o}{\PYZpc{}}\PY{k}{gs\PYZus{}lr\PYZus{}tfidf}.best\PYZus{}params\PYZus{})
\end{Verbatim}


    \begin{Verbatim}[commandchars=\\\{\}]
Best parameter set: \{'clf\_\_C': 10.0, 'clf\_\_penalty': 'l2', 'vect\_\_ngram\_range': (1, 1), 'vect\_\_stop\_words': None, 'vect\_\_tokenizer': <function tokenizer at 0x7efc0c05a6a8>\}

    \end{Verbatim}

    As we can see in the preceding output, we obtained the best grid search
results using the regular \emph{tokenizer} without Porter stemming, no
stop-word library, and tf-idfs in combination with a logistic regression
that uses L2-regularization with the regularization strength of 10.0.

Using the best model from this grid search, let's print the average
5-fold cross-validation accuracy scores on the training set and the
classification accuracy on the test dataset:

    \begin{Verbatim}[commandchars=\\\{\}]
{\color{incolor}In [{\color{incolor}21}]:} \PY{n+nb}{print}\PY{p}{(}\PY{l+s+s1}{\PYZsq{}}\PY{l+s+s1}{CV Accuracy: }\PY{l+s+si}{\PYZpc{}.3f}\PY{l+s+s1}{\PYZsq{}} \PY{o}{\PYZpc{}} \PY{p}{(}\PY{n}{gs\PYZus{}lr\PYZus{}tfidf}\PY{o}{.}\PY{n}{best\PYZus{}score\PYZus{}}\PY{p}{)}\PY{p}{)}
         \PY{n}{clf} \PY{o}{=} \PY{n}{gs\PYZus{}lr\PYZus{}tfidf}\PY{o}{.}\PY{n}{best\PYZus{}estimator\PYZus{}}
         \PY{n+nb}{print}\PY{p}{(}\PY{l+s+s1}{\PYZsq{}}\PY{l+s+s1}{Test Accuracy: }\PY{l+s+si}{\PYZpc{}.3f}\PY{l+s+s1}{\PYZsq{}} \PY{o}{\PYZpc{}} \PY{p}{(}\PY{n}{clf}\PY{o}{.}\PY{n}{score}\PY{p}{(}\PY{n}{X\PYZus{}test}\PY{p}{,} \PY{n}{y\PYZus{}test}\PY{p}{)}\PY{p}{)}\PY{p}{)}
\end{Verbatim}


    \begin{Verbatim}[commandchars=\\\{\}]
CV Accuracy: 0.897
Test Accuracy: 0.896

    \end{Verbatim}

    The results reveal that our machine learning model can predict whether a
movie review is positive or negative with 90 percent accuracy.

A still very popular classifier for text classification is the Naive
Bayes classifier, which gained popularity in applications of email spam
filtering. Naive Bayes classifiers are easy to implement,
computationally efficient, and tend to perform particularly well on
relatively small datasets compared to other algorithms.

    \section{Working with bigger data - online algorithms and out-of-core
learning}\label{working-with-bigger-data---online-algorithms-and-out-of-core-learning}

    If you executed the code examples in the previous section, you may have
noticed that it could be computationally quite expensive to construct
the feature vectors for the 50.000 movie review dataset during grid
search. In many real-world applications, it is not common to work with
even larger datasets that can exceed our computer's memory. Since not
everyone has access to supercomputer facilities, we will now apply a
techinique called \textbf{out-of-core learning}, which allows us to work
with such large datasets by fitting the classifier incrementally on
smaller batches of the dataset.

We introduced earlier the concept of \textbf{stochastic gradient
descent}, which is an optimization algorithm that updates the model's
weights using one sample at a time. In this section, we will make use of
the \emph{partial\_fit} function of \emph{SGDClassifier} in scikit-learn
to stream the documents directly from our local drive, and train a
logistic regression model using small mini-batches of documents.

First, we define a \emph{tokenizer} function that cleans the unprocessed
text data from the \emph{movie\_data.csv} file that we constructed at
the beginning of this chapter and separate it into token words while
removing stop words:

    \begin{Verbatim}[commandchars=\\\{\}]
{\color{incolor}In [{\color{incolor}22}]:} \PY{k+kn}{import} \PY{n+nn}{numpy} \PY{k}{as} \PY{n+nn}{np}
         \PY{k+kn}{import} \PY{n+nn}{re}
         \PY{k+kn}{from} \PY{n+nn}{nltk}\PY{n+nn}{.}\PY{n+nn}{corpus} \PY{k}{import} \PY{n}{stopwords}
         
         \PY{n}{stop} \PY{o}{=} \PY{n}{stopwords}\PY{o}{.}\PY{n}{words}\PY{p}{(}\PY{l+s+s1}{\PYZsq{}}\PY{l+s+s1}{english}\PY{l+s+s1}{\PYZsq{}}\PY{p}{)}
         
         \PY{k}{def} \PY{n+nf}{tokenizer}\PY{p}{(}\PY{n}{text}\PY{p}{)}\PY{p}{:}
             \PY{n}{text} \PY{o}{=} \PY{n}{re}\PY{o}{.}\PY{n}{sub}\PY{p}{(}\PY{l+s+s1}{\PYZsq{}}\PY{l+s+s1}{\PYZlt{}[\PYZca{}\PYZgt{}]*\PYZgt{}}\PY{l+s+s1}{\PYZsq{}}\PY{p}{,} \PY{l+s+s1}{\PYZsq{}}\PY{l+s+s1}{\PYZsq{}}\PY{p}{,} \PY{n}{text}\PY{p}{)}
             \PY{n}{emoticons} \PY{o}{=} \PY{n}{re}\PY{o}{.}\PY{n}{findall}\PY{p}{(}\PY{l+s+s1}{\PYZsq{}}\PY{l+s+s1}{(?::|;|=)(?:\PYZhy{})?(?:}\PY{l+s+s1}{\PYZbs{}}\PY{l+s+s1}{)|}\PY{l+s+s1}{\PYZbs{}}\PY{l+s+s1}{(|D|P)}\PY{l+s+s1}{\PYZsq{}}\PY{p}{,} \PY{n}{text}\PY{o}{.}\PY{n}{lower}\PY{p}{(}\PY{p}{)}\PY{p}{)}
             \PY{n}{text} \PY{o}{=} \PY{p}{(}\PY{n}{re}\PY{o}{.}\PY{n}{sub}\PY{p}{(}\PY{l+s+s1}{\PYZsq{}}\PY{l+s+s1}{[}\PY{l+s+s1}{\PYZbs{}}\PY{l+s+s1}{W]+}\PY{l+s+s1}{\PYZsq{}}\PY{p}{,} \PY{l+s+s1}{\PYZsq{}}\PY{l+s+s1}{ }\PY{l+s+s1}{\PYZsq{}}\PY{p}{,} \PY{n}{text}\PY{o}{.}\PY{n}{lower}\PY{p}{(}\PY{p}{)}\PY{p}{)} \PY{o}{+} \PY{l+s+s1}{\PYZsq{}}\PY{l+s+s1}{ }\PY{l+s+s1}{\PYZsq{}}\PY{o}{.}\PY{n}{join}\PY{p}{(}\PY{n}{emoticons}\PY{p}{)}\PY{o}{.}\PY{n}{replace}\PY{p}{(}\PY{l+s+s1}{\PYZsq{}}\PY{l+s+s1}{\PYZhy{}}\PY{l+s+s1}{\PYZsq{}}\PY{p}{,} \PY{l+s+s1}{\PYZsq{}}\PY{l+s+s1}{\PYZsq{}}\PY{p}{)}\PY{p}{)}
             \PY{n}{tokenized} \PY{o}{=} \PY{p}{[}\PY{n}{w} \PY{k}{for} \PY{n}{w} \PY{o+ow}{in} \PY{n}{text}\PY{o}{.}\PY{n}{split}\PY{p}{(}\PY{p}{)} \PY{k}{if} \PY{n}{w} \PY{o+ow}{not} \PY{o+ow}{in} \PY{n}{stop}\PY{p}{]}
             \PY{k}{return} \PY{n}{tokenized}
\end{Verbatim}


    Next, we define a generator function \emph{stream\_docs} that reads in
and return one document at a time:

    \begin{Verbatim}[commandchars=\\\{\}]
{\color{incolor}In [{\color{incolor}23}]:} \PY{k}{def} \PY{n+nf}{stream\PYZus{}docs}\PY{p}{(}\PY{n}{path}\PY{p}{)}\PY{p}{:}
             \PY{k}{with} \PY{n+nb}{open}\PY{p}{(}\PY{n}{path}\PY{p}{,} \PY{l+s+s1}{\PYZsq{}}\PY{l+s+s1}{r}\PY{l+s+s1}{\PYZsq{}}\PY{p}{,} \PY{n}{encoding}\PY{o}{=}\PY{l+s+s1}{\PYZsq{}}\PY{l+s+s1}{utf\PYZhy{}8}\PY{l+s+s1}{\PYZsq{}}\PY{p}{)} \PY{k}{as} \PY{n}{csv}\PY{p}{:}
                 \PY{n+nb}{next}\PY{p}{(}\PY{n}{csv}\PY{p}{)} \PY{c+c1}{\PYZsh{} skip header}
                 \PY{k}{for} \PY{n}{line} \PY{o+ow}{in} \PY{n}{csv}\PY{p}{:}
                     \PY{n}{text}\PY{p}{,} \PY{n}{label} \PY{o}{=} \PY{n}{line}\PY{p}{[}\PY{p}{:}\PY{o}{\PYZhy{}}\PY{l+m+mi}{3}\PY{p}{]}\PY{p}{,} \PY{n+nb}{int}\PY{p}{(}\PY{n}{line}\PY{p}{[}\PY{o}{\PYZhy{}}\PY{l+m+mi}{2}\PY{p}{]}\PY{p}{)}
                     \PY{k}{yield} \PY{n}{text}\PY{p}{,} \PY{n}{label}
\end{Verbatim}


    To verify that our \emph{stream\_docs} functions works correctly, let's
read in the first document from the \emph{'movie\_data.csv'} file, which
should return a tuple consisting of the review text as well as the
corresponding class label:

    \begin{Verbatim}[commandchars=\\\{\}]
{\color{incolor}In [{\color{incolor}24}]:} \PY{n+nb}{next}\PY{p}{(}\PY{n}{stream\PYZus{}docs}\PY{p}{(}\PY{n}{path}\PY{o}{=}\PY{l+s+s1}{\PYZsq{}}\PY{l+s+s1}{movie\PYZus{}data.csv}\PY{l+s+s1}{\PYZsq{}}\PY{p}{)}\PY{p}{)}
\end{Verbatim}


\begin{Verbatim}[commandchars=\\\{\}]
{\color{outcolor}Out[{\color{outcolor}24}]:} ('"This movie is the absolutely perfect way to explain what a good movie is. It is a movie both for children and parents, and it is ""timeless"". I saw this movie before Ocean\textbackslash{}'s Eleven and must admit that the actors in OE has class in advance, but the play of Klatretøsen is much more down-to-earth and moving. Why i say Ocean\textbackslash{}'s Eleven, is the fact that these 2 movies has many things in common and Klatretøsen is THAT good ! that it can bear comparison with OE. The 3 youngsters play their part in a perfect way not to believe, compared to what we are used to in Denmark. The different cases of impossible acts, is handled in a way that you wouldn\textbackslash{}'t belive from many young actors (and sometimes not even from the ""old boys""). The story ? again it has things in common with Ocean\textbackslash{}'s Eleven except for the reason of stealing the money and in Klatretøsen they need to climb up to the strongroom. The part in which the young people has to take care of Ida\textbackslash{}'s younger brother, gives a fine element in the movie. Try to imagine how to rob a bank with a 2 year child with a nappy, on your arm :-).<br /><br />This movie can bear comparison with most ""grown up"" movie like MI2 and Ocean\textbackslash{}'s Eleven. Yes i do believe it is THAT good. It is best to see it in a theater, as all the children in the room comments the movie and these comments will stay in your hearts forever :-).<br /><br />Regards Klavs."',
          1)
\end{Verbatim}
            
    We will now define a function, \emph{get\_minibatch}, that will take a
document stream from the \emph{stream\_docs} function and return a
particular number of documents specified by the \emph{size} parameter:

    \begin{Verbatim}[commandchars=\\\{\}]
{\color{incolor}In [{\color{incolor}25}]:} \PY{k}{def} \PY{n+nf}{get\PYZus{}minibatch}\PY{p}{(}\PY{n}{doc\PYZus{}stream}\PY{p}{,} \PY{n}{size}\PY{p}{)}\PY{p}{:}
             \PY{n}{docs}\PY{p}{,} \PY{n}{y} \PY{o}{=} \PY{p}{[}\PY{p}{]}\PY{p}{,} \PY{p}{[}\PY{p}{]}
             \PY{k}{try}\PY{p}{:} 
                 \PY{k}{for} \PY{n}{\PYZus{}} \PY{o+ow}{in} \PY{n+nb}{range}\PY{p}{(}\PY{n}{size}\PY{p}{)}\PY{p}{:}
                     \PY{n}{text}\PY{p}{,} \PY{n}{label} \PY{o}{=} \PY{n+nb}{next}\PY{p}{(}\PY{n}{doc\PYZus{}stream}\PY{p}{)}
                     \PY{n}{docs}\PY{o}{.}\PY{n}{append}\PY{p}{(}\PY{n}{text}\PY{p}{)}
                     \PY{n}{y}\PY{o}{.}\PY{n}{append}\PY{p}{(}\PY{n}{label}\PY{p}{)}
             \PY{k}{except} \PY{n+ne}{StopIteration}\PY{p}{:}
                 \PY{k}{return} \PY{k+kc}{None}\PY{p}{,} \PY{k+kc}{None}
             \PY{k}{return} \PY{n}{docs}\PY{p}{,} \PY{n}{y}
\end{Verbatim}


    Unfortunately, we cannot use \emph{CountVectorizer} for out-of-core
learning since it requires holding the complete vocabulary in memory.
Also, \emph{TfidfVectorizer} needs to keep all the feature vectors of
the training dataset in memory to calculate the inverse documents
frequencies. However, another useful vectorizer for text processing
implemented in scikit-learn is \emph{HashingVectorizer}.
\emph{HashingVectorizer} is data-independent and makes use of the
hashing trick via the 32-bit \emph{MurmurHash3} function.

    \begin{Verbatim}[commandchars=\\\{\}]
{\color{incolor}In [{\color{incolor}26}]:} \PY{k+kn}{from} \PY{n+nn}{sklearn}\PY{n+nn}{.}\PY{n+nn}{feature\PYZus{}extraction}\PY{n+nn}{.}\PY{n+nn}{text} \PY{k}{import} \PY{n}{HashingVectorizer}
         \PY{k+kn}{from} \PY{n+nn}{sklearn}\PY{n+nn}{.}\PY{n+nn}{linear\PYZus{}model} \PY{k}{import} \PY{n}{SGDClassifier}
         
         \PY{n}{vect} \PY{o}{=} \PY{n}{HashingVectorizer}\PY{p}{(}\PY{n}{decode\PYZus{}error}\PY{o}{=}\PY{l+s+s1}{\PYZsq{}}\PY{l+s+s1}{ignore}\PY{l+s+s1}{\PYZsq{}}\PY{p}{,} 
                                  \PY{n}{n\PYZus{}features}\PY{o}{=}\PY{l+m+mi}{2}\PY{o}{*}\PY{o}{*}\PY{l+m+mi}{21}\PY{p}{,} 
                                  \PY{n}{preprocessor}\PY{o}{=}\PY{k+kc}{None}\PY{p}{,} 
                                  \PY{n}{tokenizer}\PY{o}{=}\PY{n}{tokenizer}\PY{p}{)}
         \PY{n}{clf} \PY{o}{=} \PY{n}{SGDClassifier}\PY{p}{(}\PY{n}{loss}\PY{o}{=}\PY{l+s+s1}{\PYZsq{}}\PY{l+s+s1}{log}\PY{l+s+s1}{\PYZsq{}}\PY{p}{,} \PY{n}{random\PYZus{}state}\PY{o}{=}\PY{l+m+mi}{1}\PY{p}{,} \PY{n}{n\PYZus{}iter}\PY{o}{=}\PY{l+m+mi}{1}\PY{p}{)}
         \PY{n}{doc\PYZus{}stream} \PY{o}{=} \PY{n}{stream\PYZus{}docs}\PY{p}{(}\PY{n}{path}\PY{o}{=}\PY{l+s+s1}{\PYZsq{}}\PY{l+s+s1}{movie\PYZus{}data.csv}\PY{l+s+s1}{\PYZsq{}}\PY{p}{)}
\end{Verbatim}


    Using the preceding code, we initialized \emph{HashingVectorizer} with
our tokenizer function and set the number of features to \emph{2\^{}21}.
Furthermore, we reinitialized a logistic regression classifier by
setting the \emph{loss} parameter of the \emph{SGDClassifier} to
\emph{'log'} - note that by choosing a large number of features in the
\emph{HashingVectorizer}, we reduce the chance of causing hash
collisions, but we also increase the number of coefficients in our
logistic regression model. Now comes the really interesting part. Having
set up all the complementary functions, we can now start the out-of-core
learning using the following code:

    \begin{Verbatim}[commandchars=\\\{\}]
{\color{incolor}In [{\color{incolor}28}]:} \PY{k+kn}{import} \PY{n+nn}{pyprind} 
         
         \PY{n}{pbar} \PY{o}{=} \PY{n}{pyprind}\PY{o}{.}\PY{n}{ProgBar}\PY{p}{(}\PY{l+m+mi}{45}\PY{p}{)}
         \PY{n}{classes} \PY{o}{=} \PY{n}{np}\PY{o}{.}\PY{n}{array}\PY{p}{(}\PY{p}{[}\PY{l+m+mi}{0}\PY{p}{,} \PY{l+m+mi}{1}\PY{p}{]}\PY{p}{)}
         \PY{k}{for} \PY{n}{\PYZus{}} \PY{o+ow}{in} \PY{n+nb}{range}\PY{p}{(}\PY{l+m+mi}{45}\PY{p}{)}\PY{p}{:}
             \PY{n}{X\PYZus{}train}\PY{p}{,} \PY{n}{y\PYZus{}train} \PY{o}{=} \PY{n}{get\PYZus{}minibatch}\PY{p}{(}\PY{n}{doc\PYZus{}stream}\PY{p}{,} \PY{n}{size}\PY{o}{=}\PY{l+m+mi}{1000}\PY{p}{)}
             \PY{k}{if} \PY{o+ow}{not} \PY{n}{X\PYZus{}train}\PY{p}{:}
                 \PY{k}{break}
             \PY{n}{X\PYZus{}train} \PY{o}{=} \PY{n}{vect}\PY{o}{.}\PY{n}{transform}\PY{p}{(}\PY{n}{X\PYZus{}train}\PY{p}{)}
             \PY{n}{clf}\PY{o}{.}\PY{n}{partial\PYZus{}fit}\PY{p}{(}\PY{n}{X\PYZus{}train}\PY{p}{,} \PY{n}{y\PYZus{}train}\PY{p}{,} \PY{n}{classes}\PY{o}{=}\PY{n}{classes}\PY{p}{)}
             \PY{n}{pbar}\PY{o}{.}\PY{n}{update}\PY{p}{(}\PY{p}{)}
\end{Verbatim}


    \begin{Verbatim}[commandchars=\\\{\}]
/usr/lib/python3.6/site-packages/sklearn/linear\_model/stochastic\_gradient.py:117: DeprecationWarning: n\_iter parameter is deprecated in 0.19 and will be removed in 0.21. Use max\_iter and tol instead.
  DeprecationWarning)
/usr/lib/python3.6/site-packages/sklearn/linear\_model/stochastic\_gradient.py:117: DeprecationWarning: n\_iter parameter is deprecated in 0.19 and will be removed in 0.21. Use max\_iter and tol instead.
  DeprecationWarning)
0\% [\#                             ] 100\% | ETA: 00:00:23/usr/lib/python3.6/site-packages/sklearn/linear\_model/stochastic\_gradient.py:117: DeprecationWarning: n\_iter parameter is deprecated in 0.19 and will be removed in 0.21. Use max\_iter and tol instead.
  DeprecationWarning)
0\% [\#\#                            ] 100\% | ETA: 00:00:23/usr/lib/python3.6/site-packages/sklearn/linear\_model/stochastic\_gradient.py:117: DeprecationWarning: n\_iter parameter is deprecated in 0.19 and will be removed in 0.21. Use max\_iter and tol instead.
  DeprecationWarning)
/usr/lib/python3.6/site-packages/sklearn/linear\_model/stochastic\_gradient.py:117: DeprecationWarning: n\_iter parameter is deprecated in 0.19 and will be removed in 0.21. Use max\_iter and tol instead.
  DeprecationWarning)
0\% [\#\#\#                           ] 100\% | ETA: 00:00:22/usr/lib/python3.6/site-packages/sklearn/linear\_model/stochastic\_gradient.py:117: DeprecationWarning: n\_iter parameter is deprecated in 0.19 and will be removed in 0.21. Use max\_iter and tol instead.
  DeprecationWarning)
/usr/lib/python3.6/site-packages/sklearn/linear\_model/stochastic\_gradient.py:117: DeprecationWarning: n\_iter parameter is deprecated in 0.19 and will be removed in 0.21. Use max\_iter and tol instead.
  DeprecationWarning)
0\% [\#\#\#\#                          ] 100\% | ETA: 00:00:20/usr/lib/python3.6/site-packages/sklearn/linear\_model/stochastic\_gradient.py:117: DeprecationWarning: n\_iter parameter is deprecated in 0.19 and will be removed in 0.21. Use max\_iter and tol instead.
  DeprecationWarning)
0\% [\#\#\#\#\#                         ] 100\% | ETA: 00:00:20/usr/lib/python3.6/site-packages/sklearn/linear\_model/stochastic\_gradient.py:117: DeprecationWarning: n\_iter parameter is deprecated in 0.19 and will be removed in 0.21. Use max\_iter and tol instead.
  DeprecationWarning)
0\% [\#\#\#\#\#\#                        ] 100\% | ETA: 00:00:19/usr/lib/python3.6/site-packages/sklearn/linear\_model/stochastic\_gradient.py:117: DeprecationWarning: n\_iter parameter is deprecated in 0.19 and will be removed in 0.21. Use max\_iter and tol instead.
  DeprecationWarning)
/usr/lib/python3.6/site-packages/sklearn/linear\_model/stochastic\_gradient.py:117: DeprecationWarning: n\_iter parameter is deprecated in 0.19 and will be removed in 0.21. Use max\_iter and tol instead.
  DeprecationWarning)
0\% [\#\#\#\#\#\#\#                       ] 100\% | ETA: 00:00:18/usr/lib/python3.6/site-packages/sklearn/linear\_model/stochastic\_gradient.py:117: DeprecationWarning: n\_iter parameter is deprecated in 0.19 and will be removed in 0.21. Use max\_iter and tol instead.
  DeprecationWarning)
0\% [\#\#\#\#\#\#\#\#                      ] 100\% | ETA: 00:00:18/usr/lib/python3.6/site-packages/sklearn/linear\_model/stochastic\_gradient.py:117: DeprecationWarning: n\_iter parameter is deprecated in 0.19 and will be removed in 0.21. Use max\_iter and tol instead.
  DeprecationWarning)
/usr/lib/python3.6/site-packages/sklearn/linear\_model/stochastic\_gradient.py:117: DeprecationWarning: n\_iter parameter is deprecated in 0.19 and will be removed in 0.21. Use max\_iter and tol instead.
  DeprecationWarning)
0\% [\#\#\#\#\#\#\#\#\#                     ] 100\% | ETA: 00:00:17/usr/lib/python3.6/site-packages/sklearn/linear\_model/stochastic\_gradient.py:117: DeprecationWarning: n\_iter parameter is deprecated in 0.19 and will be removed in 0.21. Use max\_iter and tol instead.
  DeprecationWarning)
/usr/lib/python3.6/site-packages/sklearn/linear\_model/stochastic\_gradient.py:117: DeprecationWarning: n\_iter parameter is deprecated in 0.19 and will be removed in 0.21. Use max\_iter and tol instead.
  DeprecationWarning)
0\% [\#\#\#\#\#\#\#\#\#\#                    ] 100\% | ETA: 00:00:15/usr/lib/python3.6/site-packages/sklearn/linear\_model/stochastic\_gradient.py:117: DeprecationWarning: n\_iter parameter is deprecated in 0.19 and will be removed in 0.21. Use max\_iter and tol instead.
  DeprecationWarning)
0\% [\#\#\#\#\#\#\#\#\#\#\#                   ] 100\% | ETA: 00:00:15/usr/lib/python3.6/site-packages/sklearn/linear\_model/stochastic\_gradient.py:117: DeprecationWarning: n\_iter parameter is deprecated in 0.19 and will be removed in 0.21. Use max\_iter and tol instead.
  DeprecationWarning)
0\% [\#\#\#\#\#\#\#\#\#\#\#\#                  ] 100\% | ETA: 00:00:14/usr/lib/python3.6/site-packages/sklearn/linear\_model/stochastic\_gradient.py:117: DeprecationWarning: n\_iter parameter is deprecated in 0.19 and will be removed in 0.21. Use max\_iter and tol instead.
  DeprecationWarning)
/usr/lib/python3.6/site-packages/sklearn/linear\_model/stochastic\_gradient.py:117: DeprecationWarning: n\_iter parameter is deprecated in 0.19 and will be removed in 0.21. Use max\_iter and tol instead.
  DeprecationWarning)
0\% [\#\#\#\#\#\#\#\#\#\#\#\#\#                 ] 100\% | ETA: 00:00:13/usr/lib/python3.6/site-packages/sklearn/linear\_model/stochastic\_gradient.py:117: DeprecationWarning: n\_iter parameter is deprecated in 0.19 and will be removed in 0.21. Use max\_iter and tol instead.
  DeprecationWarning)
0\% [\#\#\#\#\#\#\#\#\#\#\#\#\#\#                ] 100\% | ETA: 00:00:13/usr/lib/python3.6/site-packages/sklearn/linear\_model/stochastic\_gradient.py:117: DeprecationWarning: n\_iter parameter is deprecated in 0.19 and will be removed in 0.21. Use max\_iter and tol instead.
  DeprecationWarning)
/usr/lib/python3.6/site-packages/sklearn/linear\_model/stochastic\_gradient.py:117: DeprecationWarning: n\_iter parameter is deprecated in 0.19 and will be removed in 0.21. Use max\_iter and tol instead.
  DeprecationWarning)
0\% [\#\#\#\#\#\#\#\#\#\#\#\#\#\#\#               ] 100\% | ETA: 00:00:11/usr/lib/python3.6/site-packages/sklearn/linear\_model/stochastic\_gradient.py:117: DeprecationWarning: n\_iter parameter is deprecated in 0.19 and will be removed in 0.21. Use max\_iter and tol instead.
  DeprecationWarning)
/usr/lib/python3.6/site-packages/sklearn/linear\_model/stochastic\_gradient.py:117: DeprecationWarning: n\_iter parameter is deprecated in 0.19 and will be removed in 0.21. Use max\_iter and tol instead.
  DeprecationWarning)
0\% [\#\#\#\#\#\#\#\#\#\#\#\#\#\#\#\#              ] 100\% | ETA: 00:00:10/usr/lib/python3.6/site-packages/sklearn/linear\_model/stochastic\_gradient.py:117: DeprecationWarning: n\_iter parameter is deprecated in 0.19 and will be removed in 0.21. Use max\_iter and tol instead.
  DeprecationWarning)
0\% [\#\#\#\#\#\#\#\#\#\#\#\#\#\#\#\#\#             ] 100\% | ETA: 00:00:10/usr/lib/python3.6/site-packages/sklearn/linear\_model/stochastic\_gradient.py:117: DeprecationWarning: n\_iter parameter is deprecated in 0.19 and will be removed in 0.21. Use max\_iter and tol instead.
  DeprecationWarning)
0\% [\#\#\#\#\#\#\#\#\#\#\#\#\#\#\#\#\#\#            ] 100\% | ETA: 00:00:09/usr/lib/python3.6/site-packages/sklearn/linear\_model/stochastic\_gradient.py:117: DeprecationWarning: n\_iter parameter is deprecated in 0.19 and will be removed in 0.21. Use max\_iter and tol instead.
  DeprecationWarning)
/usr/lib/python3.6/site-packages/sklearn/linear\_model/stochastic\_gradient.py:117: DeprecationWarning: n\_iter parameter is deprecated in 0.19 and will be removed in 0.21. Use max\_iter and tol instead.
  DeprecationWarning)
0\% [\#\#\#\#\#\#\#\#\#\#\#\#\#\#\#\#\#\#\#           ] 100\% | ETA: 00:00:08/usr/lib/python3.6/site-packages/sklearn/linear\_model/stochastic\_gradient.py:117: DeprecationWarning: n\_iter parameter is deprecated in 0.19 and will be removed in 0.21. Use max\_iter and tol instead.
  DeprecationWarning)
0\% [\#\#\#\#\#\#\#\#\#\#\#\#\#\#\#\#\#\#\#\#          ] 100\% | ETA: 00:00:08/usr/lib/python3.6/site-packages/sklearn/linear\_model/stochastic\_gradient.py:117: DeprecationWarning: n\_iter parameter is deprecated in 0.19 and will be removed in 0.21. Use max\_iter and tol instead.
  DeprecationWarning)
/usr/lib/python3.6/site-packages/sklearn/linear\_model/stochastic\_gradient.py:117: DeprecationWarning: n\_iter parameter is deprecated in 0.19 and will be removed in 0.21. Use max\_iter and tol instead.
  DeprecationWarning)
0\% [\#\#\#\#\#\#\#\#\#\#\#\#\#\#\#\#\#\#\#\#\#         ] 100\% | ETA: 00:00:07/usr/lib/python3.6/site-packages/sklearn/linear\_model/stochastic\_gradient.py:117: DeprecationWarning: n\_iter parameter is deprecated in 0.19 and will be removed in 0.21. Use max\_iter and tol instead.
  DeprecationWarning)
/usr/lib/python3.6/site-packages/sklearn/linear\_model/stochastic\_gradient.py:117: DeprecationWarning: n\_iter parameter is deprecated in 0.19 and will be removed in 0.21. Use max\_iter and tol instead.
  DeprecationWarning)
0\% [\#\#\#\#\#\#\#\#\#\#\#\#\#\#\#\#\#\#\#\#\#\#        ] 100\% | ETA: 00:00:05/usr/lib/python3.6/site-packages/sklearn/linear\_model/stochastic\_gradient.py:117: DeprecationWarning: n\_iter parameter is deprecated in 0.19 and will be removed in 0.21. Use max\_iter and tol instead.
  DeprecationWarning)
0\% [\#\#\#\#\#\#\#\#\#\#\#\#\#\#\#\#\#\#\#\#\#\#\#       ] 100\% | ETA: 00:00:05/usr/lib/python3.6/site-packages/sklearn/linear\_model/stochastic\_gradient.py:117: DeprecationWarning: n\_iter parameter is deprecated in 0.19 and will be removed in 0.21. Use max\_iter and tol instead.
  DeprecationWarning)
0\% [\#\#\#\#\#\#\#\#\#\#\#\#\#\#\#\#\#\#\#\#\#\#\#\#      ] 100\% | ETA: 00:00:04/usr/lib/python3.6/site-packages/sklearn/linear\_model/stochastic\_gradient.py:117: DeprecationWarning: n\_iter parameter is deprecated in 0.19 and will be removed in 0.21. Use max\_iter and tol instead.
  DeprecationWarning)
/usr/lib/python3.6/site-packages/sklearn/linear\_model/stochastic\_gradient.py:117: DeprecationWarning: n\_iter parameter is deprecated in 0.19 and will be removed in 0.21. Use max\_iter and tol instead.
  DeprecationWarning)
0\% [\#\#\#\#\#\#\#\#\#\#\#\#\#\#\#\#\#\#\#\#\#\#\#\#\#     ] 100\% | ETA: 00:00:03/usr/lib/python3.6/site-packages/sklearn/linear\_model/stochastic\_gradient.py:117: DeprecationWarning: n\_iter parameter is deprecated in 0.19 and will be removed in 0.21. Use max\_iter and tol instead.
  DeprecationWarning)
0\% [\#\#\#\#\#\#\#\#\#\#\#\#\#\#\#\#\#\#\#\#\#\#\#\#\#\#    ] 100\% | ETA: 00:00:03/usr/lib/python3.6/site-packages/sklearn/linear\_model/stochastic\_gradient.py:117: DeprecationWarning: n\_iter parameter is deprecated in 0.19 and will be removed in 0.21. Use max\_iter and tol instead.
  DeprecationWarning)
/usr/lib/python3.6/site-packages/sklearn/linear\_model/stochastic\_gradient.py:117: DeprecationWarning: n\_iter parameter is deprecated in 0.19 and will be removed in 0.21. Use max\_iter and tol instead.
  DeprecationWarning)
0\% [\#\#\#\#\#\#\#\#\#\#\#\#\#\#\#\#\#\#\#\#\#\#\#\#\#\#\#   ] 100\% | ETA: 00:00:02/usr/lib/python3.6/site-packages/sklearn/linear\_model/stochastic\_gradient.py:117: DeprecationWarning: n\_iter parameter is deprecated in 0.19 and will be removed in 0.21. Use max\_iter and tol instead.
  DeprecationWarning)
/usr/lib/python3.6/site-packages/sklearn/linear\_model/stochastic\_gradient.py:117: DeprecationWarning: n\_iter parameter is deprecated in 0.19 and will be removed in 0.21. Use max\_iter and tol instead.
  DeprecationWarning)
0\% [\#\#\#\#\#\#\#\#\#\#\#\#\#\#\#\#\#\#\#\#\#\#\#\#\#\#\#\#  ] 100\% | ETA: 00:00:01/usr/lib/python3.6/site-packages/sklearn/linear\_model/stochastic\_gradient.py:117: DeprecationWarning: n\_iter parameter is deprecated in 0.19 and will be removed in 0.21. Use max\_iter and tol instead.
  DeprecationWarning)
0\% [\#\#\#\#\#\#\#\#\#\#\#\#\#\#\#\#\#\#\#\#\#\#\#\#\#\#\#\#\# ] 100\% | ETA: 00:00:00/usr/lib/python3.6/site-packages/sklearn/linear\_model/stochastic\_gradient.py:117: DeprecationWarning: n\_iter parameter is deprecated in 0.19 and will be removed in 0.21. Use max\_iter and tol instead.
  DeprecationWarning)
0\% [\#\#\#\#\#\#\#\#\#\#\#\#\#\#\#\#\#\#\#\#\#\#\#\#\#\#\#\#\#\#] 100\% | ETA: 00:00:00
Total time elapsed: 00:00:24

    \end{Verbatim}

    Again, we made use of PyPrind package in order to estimate the progress
of our learning algorithm. We initialized the progress bar object with
45 iterations and, in the following \emph{for} loop, we iterated over 45
mini-batches of documents were each mini-batch consists of 1.000
documents. Having completed the incremental learning process, we will
use the last 5.000 documents to evaluate the performance of our model:

    \begin{Verbatim}[commandchars=\\\{\}]
{\color{incolor}In [{\color{incolor}29}]:} \PY{n}{X\PYZus{}test}\PY{p}{,} \PY{n}{y\PYZus{}test} \PY{o}{=} \PY{n}{get\PYZus{}minibatch}\PY{p}{(}\PY{n}{doc\PYZus{}stream}\PY{p}{,} \PY{n}{size}\PY{o}{=}\PY{l+m+mi}{5000}\PY{p}{)}
         \PY{n}{X\PYZus{}test} \PY{o}{=} \PY{n}{vect}\PY{o}{.}\PY{n}{transform}\PY{p}{(}\PY{n}{X\PYZus{}test}\PY{p}{)}
         \PY{n+nb}{print}\PY{p}{(}\PY{l+s+s1}{\PYZsq{}}\PY{l+s+s1}{Accuracy: }\PY{l+s+si}{\PYZpc{}.3f}\PY{l+s+s1}{\PYZsq{}} \PY{o}{\PYZpc{}} \PY{n}{clf}\PY{o}{.}\PY{n}{score}\PY{p}{(}\PY{n}{X\PYZus{}test}\PY{p}{,} \PY{n}{y\PYZus{}test}\PY{p}{)}\PY{p}{)}
\end{Verbatim}


    \begin{Verbatim}[commandchars=\\\{\}]
Accuracy: 0.872

    \end{Verbatim}

    As we can see, the accuracy of the model is approximately 88 percent,
slight below the accuracy that we achieved in the previous section using
the grid search for hyperparameter tuning. However, out-of-core learning
is very memory efficient and took less than a minute to complete.
Finally, we can use the last 5.000 documents to update our model:

    \begin{Verbatim}[commandchars=\\\{\}]
{\color{incolor}In [{\color{incolor}30}]:} \PY{n}{clf} \PY{o}{=} \PY{n}{clf}\PY{o}{.}\PY{n}{partial\PYZus{}fit}\PY{p}{(}\PY{n}{X\PYZus{}test}\PY{p}{,} \PY{n}{y\PYZus{}test}\PY{p}{)}
\end{Verbatim}


    \begin{Verbatim}[commandchars=\\\{\}]
/usr/lib/python3.6/site-packages/sklearn/linear\_model/stochastic\_gradient.py:117: DeprecationWarning: n\_iter parameter is deprecated in 0.19 and will be removed in 0.21. Use max\_iter and tol instead.
  DeprecationWarning)

    \end{Verbatim}

    If you are planning to continue directly to the next chapter, I
recommend you keep the current Python session open. In the next chapter,
we will use the model that we just trained to learn how to save it to
the disk for later use and embed it into a web application.

    \section{Topic modeling with Latent Dirichlet
Allocation}\label{topic-modeling-with-latent-dirichlet-allocation}

    Topic modeling describes the broad task of assigning topis to unlabelled
text documents. For example, a typical application would be the
categorization of documents in a large text corpus of newspaper articles
where we do not know on which specific page or category they appear in.
In applications of topic modeling, we then aim to assign category labels
to those articles - for example, sports, finance, world news, politics,
local news, and so forth. Thus, in the context of the broad categories
of machine learning, we can consider topic modeling as a clustering
task, a subcategory of unsupervised learning.

In this section, we will introduce a popular technique for topic
modeling called \textbf{Latent Dirichlet Allocation (LDA)}. However,
note that while Latent Dirichlet Allocation if often abbreviated as LDA,
it is not to be confused with Linear Discriminant Analysis, a supervised
dimensionality reduction technique.

    \subsection{Decomposing text documents with
LDA}\label{decomposing-text-documents-with-lda}

    Since the mathematics behind LDA is quite involved and requires
knowledge about Bayesian inference, we will approach this topic from a
practitioner's perspective and interpret LDA using layman's terms.

LDA is a generative probabilistic model that tries to find groups of
words that appear frequently together across different documents. These
frequently appearing words represent our topics, assuming that each
document is a mixture of different words. The input to a LDA is the
bag-of-words model we discusse earlier in this chapter. Given a
bag-of-words matrix as input, LDA decomposes it into two new matrices:

\begin{itemize}
\tightlist
\item
  A document of topic matrix
\item
  A word to topic matrix
\end{itemize}

LDA decomposes the bag-of-words matrix in such a way that if we multiply
those two matrices together, we would be able to reproduce the input,
the bag-of-words matrix, with the lowest possible error. In practice, we
are interested in those topics that LDA found in the bag-of-words
matrix. The only downside may be that we must define the number of
topics beforehand - the number of topics is a hyperparameter of LDA that
has to be specified manually.

    \subsection{LDA with scikit-learn}\label{lda-with-scikit-learn}

    In this subsection, we will use the \emph{LatentDirichAllocation} class
implemented in scikit-learn to decompose the movie review dataset and
categorize it into different topics. In the following example, we
restrict the analysis to 10 different topics, but readers are encouraged
to experiment with the hyperparameters of the algorithm to explore the
topics that can be found in this dataset further.

First, we are going to load the dataset into a pandas \emph{DataFrame}
using the local \emph{movie\_data.csv} file of the movie reviews that we
have created at the beginning of this chapter:

    \begin{Verbatim}[commandchars=\\\{\}]
{\color{incolor}In [{\color{incolor}31}]:} \PY{k+kn}{import} \PY{n+nn}{pandas} \PY{k}{as} \PY{n+nn}{pd}
         
         \PY{n}{df} \PY{o}{=} \PY{n}{pd}\PY{o}{.}\PY{n}{read\PYZus{}csv}\PY{p}{(}\PY{l+s+s1}{\PYZsq{}}\PY{l+s+s1}{movie\PYZus{}data.csv}\PY{l+s+s1}{\PYZsq{}}\PY{p}{,} \PY{n}{encoding}\PY{o}{=}\PY{l+s+s1}{\PYZsq{}}\PY{l+s+s1}{utf\PYZhy{}8}\PY{l+s+s1}{\PYZsq{}}\PY{p}{)}
\end{Verbatim}


    Next, we are going to use the already familiar \emph{CountVectorizer} to
create the bag-of-words matrix as input to the LDA. For convenience, we
will use scikit-learn's built-in English stop word library via
\emph{stop\_words='english'}:

    \begin{Verbatim}[commandchars=\\\{\}]
{\color{incolor}In [{\color{incolor}32}]:} \PY{k+kn}{from} \PY{n+nn}{sklearn}\PY{n+nn}{.}\PY{n+nn}{feature\PYZus{}extraction}\PY{n+nn}{.}\PY{n+nn}{text} \PY{k}{import} \PY{n}{CountVectorizer}
         
         \PY{n}{count} \PY{o}{=} \PY{n}{CountVectorizer}\PY{p}{(}\PY{n}{stop\PYZus{}words}\PY{o}{=}\PY{l+s+s1}{\PYZsq{}}\PY{l+s+s1}{english}\PY{l+s+s1}{\PYZsq{}}\PY{p}{,} 
                                 \PY{n}{max\PYZus{}df}\PY{o}{=}\PY{o}{.}\PY{l+m+mi}{1}\PY{p}{,} 
                                 \PY{n}{max\PYZus{}features}\PY{o}{=}\PY{l+m+mi}{5000}\PY{p}{)}
         \PY{n}{X} \PY{o}{=} \PY{n}{count}\PY{o}{.}\PY{n}{fit\PYZus{}transform}\PY{p}{(}\PY{n}{df}\PY{p}{[}\PY{l+s+s1}{\PYZsq{}}\PY{l+s+s1}{review}\PY{l+s+s1}{\PYZsq{}}\PY{p}{]}\PY{o}{.}\PY{n}{values}\PY{p}{)}
\end{Verbatim}


    Notice that we set the maximum document frequency of words to be
considered to 10 percent (\emph{max\_df=.1}) to exclude words that occur
too frequently across documents. The rationale behind the removal of
frequently occuring words is that these might be common words appearing
across all documents and are therefore less likely associated with a
specific topic category of a given document. Also, we limited the number
of words to be considered to the most frequently occuring 5.000 words
(\emph{max\_features=5000}), to limit the dimensionality of this dataset
so that it improves the inference performed by LDA. However, both
\emph{max\_df-.1} and \emph{max\_features=5000} are hyperparameter
values that I chose arbitrarily, and readers are encouraged to tune them
while comparing the results.

The following code example demonstrates how to fit a
\emph{LatentDirichletAllocation} estimator to the bag-of-words matrix
and infer the 10 different topics from the documents (note that the
model fitting can take up to five minutes or more on a laptop or
standard desktop computer):

    \begin{Verbatim}[commandchars=\\\{\}]
{\color{incolor}In [{\color{incolor}33}]:} \PY{k+kn}{from} \PY{n+nn}{sklearn}\PY{n+nn}{.}\PY{n+nn}{decomposition} \PY{k}{import} \PY{n}{LatentDirichletAllocation}
         
         \PY{n}{lda} \PY{o}{=} \PY{n}{LatentDirichletAllocation}\PY{p}{(}\PY{n}{n\PYZus{}topics}\PY{o}{=}\PY{l+m+mi}{10}\PY{p}{,} 
                                         \PY{n}{random\PYZus{}state}\PY{o}{=}\PY{l+m+mi}{123}\PY{p}{,} 
                                         \PY{n}{learning\PYZus{}method}\PY{o}{=}\PY{l+s+s1}{\PYZsq{}}\PY{l+s+s1}{batch}\PY{l+s+s1}{\PYZsq{}}\PY{p}{)}
         \PY{n}{X\PYZus{}topics} \PY{o}{=} \PY{n}{lda}\PY{o}{.}\PY{n}{fit\PYZus{}transform}\PY{p}{(}\PY{n}{X}\PY{p}{)}
\end{Verbatim}


    \begin{Verbatim}[commandchars=\\\{\}]
/usr/lib/python3.6/site-packages/sklearn/decomposition/online\_lda.py:294: DeprecationWarning: n\_topics has been renamed to n\_components in version 0.19 and will be removed in 0.21
  DeprecationWarning)

    \end{Verbatim}

    By setting \emph{learning\_method='batch'}, we let the \emph{lda}
estimator do its estimation based on all available training data (the
bag-of-words matrix) in one iteration, which is slower than the
alternative \emph{'online'} learning method but can lead to more
accurate results (setting \emph{learning\_method='online'} is analogous
to online and mini-batch learning that we discussed previously).

    After fitting the LDA, we now have access to the \emph{components\_}
attribute of the \emph{lda} instance, which stores a matrix containing
the word importance (here, 5000) for each of the 10 topics in increasing
order:

    \begin{Verbatim}[commandchars=\\\{\}]
{\color{incolor}In [{\color{incolor}34}]:} \PY{n}{lda}\PY{o}{.}\PY{n}{components\PYZus{}}\PY{o}{.}\PY{n}{shape}
\end{Verbatim}


\begin{Verbatim}[commandchars=\\\{\}]
{\color{outcolor}Out[{\color{outcolor}34}]:} (10, 5000)
\end{Verbatim}
            
    To analyse the results, let's print the five most important words for
each of the 10 topics. Note that the word importance are ranked in
increasing order. Thus, to print the five top words, we need to sort the
\emph{topic} array in reverse order:

    \begin{Verbatim}[commandchars=\\\{\}]
{\color{incolor}In [{\color{incolor}35}]:} \PY{n}{n\PYZus{}top\PYZus{}words} \PY{o}{=} \PY{l+m+mi}{5}
         \PY{n}{feature\PYZus{}names} \PY{o}{=} \PY{n}{count}\PY{o}{.}\PY{n}{get\PYZus{}feature\PYZus{}names}\PY{p}{(}\PY{p}{)}
         \PY{k}{for} \PY{n}{topic\PYZus{}idx}\PY{p}{,} \PY{n}{topic} \PY{o+ow}{in} \PY{n+nb}{enumerate}\PY{p}{(}\PY{n}{lda}\PY{o}{.}\PY{n}{components\PYZus{}}\PY{p}{)}\PY{p}{:}
             \PY{n+nb}{print}\PY{p}{(}\PY{l+s+s1}{\PYZsq{}}\PY{l+s+s1}{Topic }\PY{l+s+si}{\PYZpc{}d}\PY{l+s+s1}{:}\PY{l+s+s1}{\PYZsq{}} \PY{o}{\PYZpc{}} \PY{p}{(}\PY{n}{topic\PYZus{}idx} \PY{o}{+} \PY{l+m+mi}{1}\PY{p}{)}\PY{p}{)}
             \PY{n+nb}{print}\PY{p}{(}\PY{l+s+s1}{\PYZsq{}}\PY{l+s+s1}{ }\PY{l+s+s1}{\PYZsq{}}\PY{o}{.}\PY{n}{join}\PY{p}{(}\PY{p}{[}\PY{n}{feature\PYZus{}names}\PY{p}{[}\PY{n}{i}\PY{p}{]} \PY{k}{for} \PY{n}{i} \PY{o+ow}{in} \PY{n}{topic}\PY{o}{.}\PY{n}{argsort}\PY{p}{(}\PY{p}{)}\PY{p}{[}\PY{p}{:}\PY{o}{\PYZhy{}}\PY{n}{n\PYZus{}top\PYZus{}words}\PY{o}{\PYZhy{}}\PY{l+m+mi}{1}\PY{p}{:}\PY{o}{\PYZhy{}}\PY{l+m+mi}{1}\PY{p}{]}\PY{p}{]}\PY{p}{)}\PY{p}{)}
\end{Verbatim}


    \begin{Verbatim}[commandchars=\\\{\}]
Topic 1:
worst minutes script awful stupid
Topic 2:
family mother father children girl
Topic 3:
american war dvd music tv
Topic 4:
human audience cinema art sense
Topic 5:
police dead guy car murder
Topic 6:
horror house sex girl woman
Topic 7:
role performance comedy actor performances
Topic 8:
series episode war episodes season
Topic 9:
book version original effects read
Topic 10:
action fight guy guys cool

    \end{Verbatim}

    Based on reading the five most important words for each topic, we may
guess that the LDA identified the following topics:

\begin{enumerate}
\def\labelenumi{\arabic{enumi}.}
\tightlist
\item
  Generally bad movies (not really a topic category)
\item
  Movies about families
\item
  War movies
\item
  Art movies
\item
  Crime movies
\item
  Horror movies
\item
  Comedy movies
\item
  Movies somehow related to TV shows
\item
  Movies based on books
\item
  Action movies
\end{enumerate}

To confirm that the categories make sense based on the reviews, let's
plot three movies from the horror movie category (horror movies belong
to category 6 at index position 5):

    \begin{Verbatim}[commandchars=\\\{\}]
{\color{incolor}In [{\color{incolor}37}]:} \PY{n}{horror} \PY{o}{=} \PY{n}{X\PYZus{}topics}\PY{p}{[}\PY{p}{:}\PY{p}{,} \PY{l+m+mi}{5}\PY{p}{]}\PY{o}{.}\PY{n}{argsort}\PY{p}{(}\PY{p}{)}\PY{p}{[}\PY{p}{:}\PY{p}{:}\PY{o}{\PYZhy{}}\PY{l+m+mi}{1}\PY{p}{]}
         \PY{k}{for} \PY{n}{iter\PYZus{}idx}\PY{p}{,} \PY{n}{movie\PYZus{}idx} \PY{o+ow}{in} \PY{n+nb}{enumerate}\PY{p}{(}\PY{n}{horror}\PY{p}{[}\PY{p}{:}\PY{l+m+mi}{3}\PY{p}{]}\PY{p}{)}\PY{p}{:}
             \PY{n+nb}{print}\PY{p}{(}\PY{l+s+s1}{\PYZsq{}}\PY{l+s+se}{\PYZbs{}n}\PY{l+s+s1}{Horror movie \PYZsh{}}\PY{l+s+si}{\PYZpc{}d}\PY{l+s+s1}{:}\PY{l+s+s1}{\PYZsq{}} \PY{o}{\PYZpc{}} \PY{p}{(}\PY{n}{iter\PYZus{}idx} \PY{o}{+} \PY{l+m+mi}{1}\PY{p}{)}\PY{p}{)}
             \PY{n+nb}{print}\PY{p}{(}\PY{n}{df}\PY{p}{[}\PY{l+s+s1}{\PYZsq{}}\PY{l+s+s1}{review}\PY{l+s+s1}{\PYZsq{}}\PY{p}{]}\PY{p}{[}\PY{n}{movie\PYZus{}idx}\PY{p}{]}\PY{p}{[}\PY{p}{:}\PY{l+m+mi}{300}\PY{p}{]}\PY{p}{,} \PY{l+s+s1}{\PYZsq{}}\PY{l+s+s1}{...}\PY{l+s+s1}{\PYZsq{}}\PY{p}{)}
\end{Verbatim}


    \begin{Verbatim}[commandchars=\\\{\}]

Horror movie \#1:
Once upon a time in a castle{\ldots} Two little girls are playing in the garden's castle. They are sisters. A blonde little girl (Kitty) and a brunette one (Evelyn). Evelyn steals Kitty's doll. Kitty pursues Evelyn. Running through long corridors, they reach the room where their grandfather, sitting o {\ldots}

Horror movie \#2:
House of Dracula works from the same basic premise as House of Frankenstein from the year before; namely that Universal's three most famous monsters; Dracula, Frankenstein's Monster and The Wolf Man are appearing in the movie together. Naturally, the film is rather messy therefore, but the fact that {\ldots}

Horror movie \#3:
<br /><br />Horror movie time, Japanese style. Uzumaki/Spiral was a total freakfest from start to finish. A fun freakfest at that, but at times it was a tad too reliant on kitsch rather than the horror. The story is difficult to summarize succinctly: a carefree, normal teenage girl starts coming fac {\ldots}

    \end{Verbatim}

    Using the preceding code example, we printed the first 300 characters
from the top three horror movies, and we can see that the reviews - even
though we do not know which exact movie they belong to - sound like
reviews of horror movies (however, one might argue that \emph{Horror
movie \#2} could also be a good fit for topic category 1:
\emph{Generally bad movies}).

    \section{Summary}\label{summary}

    In this chapter, we learned how to use machine learning algorithms to
classify text documents based on their polarity, which is a basic task
in sentiment analysis in the field of NLP. Not only did we learn how to
encode a document as a feature vector using the bag-of-words model, but
we also learned how to weight the term frequency by relevance using
tf-idf.

Working with text data can be computationally quite expensive due to the
large feature vectors that are created during this process; in the last
section, we learned how to utilize out-of-core or incremental learning
to train a machine learning algorith without loading the whole dataset
into a computer's memory.

Lastly, we introduced the concept of topic modeling using LDA to
categorize the movie reviews into different categories in unsupervised
fashion.

In the next chapter, we will use our document classifier and learn how
to embed it into a web application.

    \section{Serializing fitted scikit-learn
estimators}\label{serializing-fitted-scikit-learn-estimators}

    \begin{Verbatim}[commandchars=\\\{\}]
{\color{incolor}In [{\color{incolor}40}]:} \PY{k+kn}{import} \PY{n+nn}{pickle} 
         \PY{k+kn}{import} \PY{n+nn}{os}
         
         \PY{n}{dest} \PY{o}{=} \PY{n}{os}\PY{o}{.}\PY{n}{path}\PY{o}{.}\PY{n}{join}\PY{p}{(}\PY{l+s+s1}{\PYZsq{}}\PY{l+s+s1}{movieclassifier}\PY{l+s+s1}{\PYZsq{}}\PY{p}{,} \PY{l+s+s1}{\PYZsq{}}\PY{l+s+s1}{pkl\PYZus{}objects}\PY{l+s+s1}{\PYZsq{}}\PY{p}{)}
         \PY{k}{if} \PY{o+ow}{not} \PY{n}{os}\PY{o}{.}\PY{n}{path}\PY{o}{.}\PY{n}{exists}\PY{p}{(}\PY{n}{dest}\PY{p}{)}\PY{p}{:}
             \PY{n}{os}\PY{o}{.}\PY{n}{makedirs}\PY{p}{(}\PY{n}{dest}\PY{p}{)}
             
         \PY{n}{pickle}\PY{o}{.}\PY{n}{dump}\PY{p}{(}\PY{n}{stop}\PY{p}{,} \PY{n+nb}{open}\PY{p}{(}\PY{n}{os}\PY{o}{.}\PY{n}{path}\PY{o}{.}\PY{n}{join}\PY{p}{(}\PY{n}{dest}\PY{p}{,} \PY{l+s+s1}{\PYZsq{}}\PY{l+s+s1}{stopwords.pkl}\PY{l+s+s1}{\PYZsq{}}\PY{p}{)}\PY{p}{,} \PY{l+s+s1}{\PYZsq{}}\PY{l+s+s1}{wb}\PY{l+s+s1}{\PYZsq{}}\PY{p}{)}\PY{p}{,} \PY{n}{protocol}\PY{o}{=}\PY{l+m+mi}{4}\PY{p}{)}
         \PY{n}{pickle}\PY{o}{.}\PY{n}{dump}\PY{p}{(}\PY{n}{clf}\PY{p}{,} \PY{n+nb}{open}\PY{p}{(}\PY{n}{os}\PY{o}{.}\PY{n}{path}\PY{o}{.}\PY{n}{join}\PY{p}{(}\PY{n}{dest}\PY{p}{,} \PY{l+s+s1}{\PYZsq{}}\PY{l+s+s1}{classifier.pkl}\PY{l+s+s1}{\PYZsq{}}\PY{p}{)}\PY{p}{,} \PY{l+s+s1}{\PYZsq{}}\PY{l+s+s1}{wb}\PY{l+s+s1}{\PYZsq{}}\PY{p}{)}\PY{p}{,} \PY{n}{protocol}\PY{o}{=}\PY{l+m+mi}{4}\PY{p}{)}
\end{Verbatim}



    % Add a bibliography block to the postdoc
    
    
    
    \end{document}
