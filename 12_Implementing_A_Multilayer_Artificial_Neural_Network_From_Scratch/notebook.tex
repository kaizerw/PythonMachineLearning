
% Default to the notebook output style

    


% Inherit from the specified cell style.




    
\documentclass[11pt]{article}

    
    
    \usepackage[T1]{fontenc}
    % Nicer default font (+ math font) than Computer Modern for most use cases
    \usepackage{mathpazo}

    % Basic figure setup, for now with no caption control since it's done
    % automatically by Pandoc (which extracts ![](path) syntax from Markdown).
    \usepackage{graphicx}
    % We will generate all images so they have a width \maxwidth. This means
    % that they will get their normal width if they fit onto the page, but
    % are scaled down if they would overflow the margins.
    \makeatletter
    \def\maxwidth{\ifdim\Gin@nat@width>\linewidth\linewidth
    \else\Gin@nat@width\fi}
    \makeatother
    \let\Oldincludegraphics\includegraphics
    % Set max figure width to be 80% of text width, for now hardcoded.
    \renewcommand{\includegraphics}[1]{\Oldincludegraphics[width=.8\maxwidth]{#1}}
    % Ensure that by default, figures have no caption (until we provide a
    % proper Figure object with a Caption API and a way to capture that
    % in the conversion process - todo).
    \usepackage{caption}
    \DeclareCaptionLabelFormat{nolabel}{}
    \captionsetup{labelformat=nolabel}

    \usepackage{adjustbox} % Used to constrain images to a maximum size 
    \usepackage{xcolor} % Allow colors to be defined
    \usepackage{enumerate} % Needed for markdown enumerations to work
    \usepackage{geometry} % Used to adjust the document margins
    \usepackage{amsmath} % Equations
    \usepackage{amssymb} % Equations
    \usepackage{textcomp} % defines textquotesingle
    % Hack from http://tex.stackexchange.com/a/47451/13684:
    \AtBeginDocument{%
        \def\PYZsq{\textquotesingle}% Upright quotes in Pygmentized code
    }
    \usepackage{upquote} % Upright quotes for verbatim code
    \usepackage{eurosym} % defines \euro
    \usepackage[mathletters]{ucs} % Extended unicode (utf-8) support
    \usepackage[utf8x]{inputenc} % Allow utf-8 characters in the tex document
    \usepackage{fancyvrb} % verbatim replacement that allows latex
    \usepackage{grffile} % extends the file name processing of package graphics 
                         % to support a larger range 
    % The hyperref package gives us a pdf with properly built
    % internal navigation ('pdf bookmarks' for the table of contents,
    % internal cross-reference links, web links for URLs, etc.)
    \usepackage{hyperref}
    \usepackage{longtable} % longtable support required by pandoc >1.10
    \usepackage{booktabs}  % table support for pandoc > 1.12.2
    \usepackage[inline]{enumitem} % IRkernel/repr support (it uses the enumerate* environment)
    \usepackage[normalem]{ulem} % ulem is needed to support strikethroughs (\sout)
                                % normalem makes italics be italics, not underlines
    

    
    
    % Colors for the hyperref package
    \definecolor{urlcolor}{rgb}{0,.145,.698}
    \definecolor{linkcolor}{rgb}{.71,0.21,0.01}
    \definecolor{citecolor}{rgb}{.12,.54,.11}

    % ANSI colors
    \definecolor{ansi-black}{HTML}{3E424D}
    \definecolor{ansi-black-intense}{HTML}{282C36}
    \definecolor{ansi-red}{HTML}{E75C58}
    \definecolor{ansi-red-intense}{HTML}{B22B31}
    \definecolor{ansi-green}{HTML}{00A250}
    \definecolor{ansi-green-intense}{HTML}{007427}
    \definecolor{ansi-yellow}{HTML}{DDB62B}
    \definecolor{ansi-yellow-intense}{HTML}{B27D12}
    \definecolor{ansi-blue}{HTML}{208FFB}
    \definecolor{ansi-blue-intense}{HTML}{0065CA}
    \definecolor{ansi-magenta}{HTML}{D160C4}
    \definecolor{ansi-magenta-intense}{HTML}{A03196}
    \definecolor{ansi-cyan}{HTML}{60C6C8}
    \definecolor{ansi-cyan-intense}{HTML}{258F8F}
    \definecolor{ansi-white}{HTML}{C5C1B4}
    \definecolor{ansi-white-intense}{HTML}{A1A6B2}

    % commands and environments needed by pandoc snippets
    % extracted from the output of `pandoc -s`
    \providecommand{\tightlist}{%
      \setlength{\itemsep}{0pt}\setlength{\parskip}{0pt}}
    \DefineVerbatimEnvironment{Highlighting}{Verbatim}{commandchars=\\\{\}}
    % Add ',fontsize=\small' for more characters per line
    \newenvironment{Shaded}{}{}
    \newcommand{\KeywordTok}[1]{\textcolor[rgb]{0.00,0.44,0.13}{\textbf{{#1}}}}
    \newcommand{\DataTypeTok}[1]{\textcolor[rgb]{0.56,0.13,0.00}{{#1}}}
    \newcommand{\DecValTok}[1]{\textcolor[rgb]{0.25,0.63,0.44}{{#1}}}
    \newcommand{\BaseNTok}[1]{\textcolor[rgb]{0.25,0.63,0.44}{{#1}}}
    \newcommand{\FloatTok}[1]{\textcolor[rgb]{0.25,0.63,0.44}{{#1}}}
    \newcommand{\CharTok}[1]{\textcolor[rgb]{0.25,0.44,0.63}{{#1}}}
    \newcommand{\StringTok}[1]{\textcolor[rgb]{0.25,0.44,0.63}{{#1}}}
    \newcommand{\CommentTok}[1]{\textcolor[rgb]{0.38,0.63,0.69}{\textit{{#1}}}}
    \newcommand{\OtherTok}[1]{\textcolor[rgb]{0.00,0.44,0.13}{{#1}}}
    \newcommand{\AlertTok}[1]{\textcolor[rgb]{1.00,0.00,0.00}{\textbf{{#1}}}}
    \newcommand{\FunctionTok}[1]{\textcolor[rgb]{0.02,0.16,0.49}{{#1}}}
    \newcommand{\RegionMarkerTok}[1]{{#1}}
    \newcommand{\ErrorTok}[1]{\textcolor[rgb]{1.00,0.00,0.00}{\textbf{{#1}}}}
    \newcommand{\NormalTok}[1]{{#1}}
    
    % Additional commands for more recent versions of Pandoc
    \newcommand{\ConstantTok}[1]{\textcolor[rgb]{0.53,0.00,0.00}{{#1}}}
    \newcommand{\SpecialCharTok}[1]{\textcolor[rgb]{0.25,0.44,0.63}{{#1}}}
    \newcommand{\VerbatimStringTok}[1]{\textcolor[rgb]{0.25,0.44,0.63}{{#1}}}
    \newcommand{\SpecialStringTok}[1]{\textcolor[rgb]{0.73,0.40,0.53}{{#1}}}
    \newcommand{\ImportTok}[1]{{#1}}
    \newcommand{\DocumentationTok}[1]{\textcolor[rgb]{0.73,0.13,0.13}{\textit{{#1}}}}
    \newcommand{\AnnotationTok}[1]{\textcolor[rgb]{0.38,0.63,0.69}{\textbf{\textit{{#1}}}}}
    \newcommand{\CommentVarTok}[1]{\textcolor[rgb]{0.38,0.63,0.69}{\textbf{\textit{{#1}}}}}
    \newcommand{\VariableTok}[1]{\textcolor[rgb]{0.10,0.09,0.49}{{#1}}}
    \newcommand{\ControlFlowTok}[1]{\textcolor[rgb]{0.00,0.44,0.13}{\textbf{{#1}}}}
    \newcommand{\OperatorTok}[1]{\textcolor[rgb]{0.40,0.40,0.40}{{#1}}}
    \newcommand{\BuiltInTok}[1]{{#1}}
    \newcommand{\ExtensionTok}[1]{{#1}}
    \newcommand{\PreprocessorTok}[1]{\textcolor[rgb]{0.74,0.48,0.00}{{#1}}}
    \newcommand{\AttributeTok}[1]{\textcolor[rgb]{0.49,0.56,0.16}{{#1}}}
    \newcommand{\InformationTok}[1]{\textcolor[rgb]{0.38,0.63,0.69}{\textbf{\textit{{#1}}}}}
    \newcommand{\WarningTok}[1]{\textcolor[rgb]{0.38,0.63,0.69}{\textbf{\textit{{#1}}}}}
    
    
    % Define a nice break command that doesn't care if a line doesn't already
    % exist.
    \def\br{\hspace*{\fill} \\* }
    % Math Jax compatability definitions
    \def\gt{>}
    \def\lt{<}
    % Document parameters
    \title{12\_Implementing\_A\_Multilayer\_Artificial\_Neural\_Network\_From\_Scratch}
    
    
    

    % Pygments definitions
    
\makeatletter
\def\PY@reset{\let\PY@it=\relax \let\PY@bf=\relax%
    \let\PY@ul=\relax \let\PY@tc=\relax%
    \let\PY@bc=\relax \let\PY@ff=\relax}
\def\PY@tok#1{\csname PY@tok@#1\endcsname}
\def\PY@toks#1+{\ifx\relax#1\empty\else%
    \PY@tok{#1}\expandafter\PY@toks\fi}
\def\PY@do#1{\PY@bc{\PY@tc{\PY@ul{%
    \PY@it{\PY@bf{\PY@ff{#1}}}}}}}
\def\PY#1#2{\PY@reset\PY@toks#1+\relax+\PY@do{#2}}

\expandafter\def\csname PY@tok@w\endcsname{\def\PY@tc##1{\textcolor[rgb]{0.73,0.73,0.73}{##1}}}
\expandafter\def\csname PY@tok@c\endcsname{\let\PY@it=\textit\def\PY@tc##1{\textcolor[rgb]{0.25,0.50,0.50}{##1}}}
\expandafter\def\csname PY@tok@cp\endcsname{\def\PY@tc##1{\textcolor[rgb]{0.74,0.48,0.00}{##1}}}
\expandafter\def\csname PY@tok@k\endcsname{\let\PY@bf=\textbf\def\PY@tc##1{\textcolor[rgb]{0.00,0.50,0.00}{##1}}}
\expandafter\def\csname PY@tok@kp\endcsname{\def\PY@tc##1{\textcolor[rgb]{0.00,0.50,0.00}{##1}}}
\expandafter\def\csname PY@tok@kt\endcsname{\def\PY@tc##1{\textcolor[rgb]{0.69,0.00,0.25}{##1}}}
\expandafter\def\csname PY@tok@o\endcsname{\def\PY@tc##1{\textcolor[rgb]{0.40,0.40,0.40}{##1}}}
\expandafter\def\csname PY@tok@ow\endcsname{\let\PY@bf=\textbf\def\PY@tc##1{\textcolor[rgb]{0.67,0.13,1.00}{##1}}}
\expandafter\def\csname PY@tok@nb\endcsname{\def\PY@tc##1{\textcolor[rgb]{0.00,0.50,0.00}{##1}}}
\expandafter\def\csname PY@tok@nf\endcsname{\def\PY@tc##1{\textcolor[rgb]{0.00,0.00,1.00}{##1}}}
\expandafter\def\csname PY@tok@nc\endcsname{\let\PY@bf=\textbf\def\PY@tc##1{\textcolor[rgb]{0.00,0.00,1.00}{##1}}}
\expandafter\def\csname PY@tok@nn\endcsname{\let\PY@bf=\textbf\def\PY@tc##1{\textcolor[rgb]{0.00,0.00,1.00}{##1}}}
\expandafter\def\csname PY@tok@ne\endcsname{\let\PY@bf=\textbf\def\PY@tc##1{\textcolor[rgb]{0.82,0.25,0.23}{##1}}}
\expandafter\def\csname PY@tok@nv\endcsname{\def\PY@tc##1{\textcolor[rgb]{0.10,0.09,0.49}{##1}}}
\expandafter\def\csname PY@tok@no\endcsname{\def\PY@tc##1{\textcolor[rgb]{0.53,0.00,0.00}{##1}}}
\expandafter\def\csname PY@tok@nl\endcsname{\def\PY@tc##1{\textcolor[rgb]{0.63,0.63,0.00}{##1}}}
\expandafter\def\csname PY@tok@ni\endcsname{\let\PY@bf=\textbf\def\PY@tc##1{\textcolor[rgb]{0.60,0.60,0.60}{##1}}}
\expandafter\def\csname PY@tok@na\endcsname{\def\PY@tc##1{\textcolor[rgb]{0.49,0.56,0.16}{##1}}}
\expandafter\def\csname PY@tok@nt\endcsname{\let\PY@bf=\textbf\def\PY@tc##1{\textcolor[rgb]{0.00,0.50,0.00}{##1}}}
\expandafter\def\csname PY@tok@nd\endcsname{\def\PY@tc##1{\textcolor[rgb]{0.67,0.13,1.00}{##1}}}
\expandafter\def\csname PY@tok@s\endcsname{\def\PY@tc##1{\textcolor[rgb]{0.73,0.13,0.13}{##1}}}
\expandafter\def\csname PY@tok@sd\endcsname{\let\PY@it=\textit\def\PY@tc##1{\textcolor[rgb]{0.73,0.13,0.13}{##1}}}
\expandafter\def\csname PY@tok@si\endcsname{\let\PY@bf=\textbf\def\PY@tc##1{\textcolor[rgb]{0.73,0.40,0.53}{##1}}}
\expandafter\def\csname PY@tok@se\endcsname{\let\PY@bf=\textbf\def\PY@tc##1{\textcolor[rgb]{0.73,0.40,0.13}{##1}}}
\expandafter\def\csname PY@tok@sr\endcsname{\def\PY@tc##1{\textcolor[rgb]{0.73,0.40,0.53}{##1}}}
\expandafter\def\csname PY@tok@ss\endcsname{\def\PY@tc##1{\textcolor[rgb]{0.10,0.09,0.49}{##1}}}
\expandafter\def\csname PY@tok@sx\endcsname{\def\PY@tc##1{\textcolor[rgb]{0.00,0.50,0.00}{##1}}}
\expandafter\def\csname PY@tok@m\endcsname{\def\PY@tc##1{\textcolor[rgb]{0.40,0.40,0.40}{##1}}}
\expandafter\def\csname PY@tok@gh\endcsname{\let\PY@bf=\textbf\def\PY@tc##1{\textcolor[rgb]{0.00,0.00,0.50}{##1}}}
\expandafter\def\csname PY@tok@gu\endcsname{\let\PY@bf=\textbf\def\PY@tc##1{\textcolor[rgb]{0.50,0.00,0.50}{##1}}}
\expandafter\def\csname PY@tok@gd\endcsname{\def\PY@tc##1{\textcolor[rgb]{0.63,0.00,0.00}{##1}}}
\expandafter\def\csname PY@tok@gi\endcsname{\def\PY@tc##1{\textcolor[rgb]{0.00,0.63,0.00}{##1}}}
\expandafter\def\csname PY@tok@gr\endcsname{\def\PY@tc##1{\textcolor[rgb]{1.00,0.00,0.00}{##1}}}
\expandafter\def\csname PY@tok@ge\endcsname{\let\PY@it=\textit}
\expandafter\def\csname PY@tok@gs\endcsname{\let\PY@bf=\textbf}
\expandafter\def\csname PY@tok@gp\endcsname{\let\PY@bf=\textbf\def\PY@tc##1{\textcolor[rgb]{0.00,0.00,0.50}{##1}}}
\expandafter\def\csname PY@tok@go\endcsname{\def\PY@tc##1{\textcolor[rgb]{0.53,0.53,0.53}{##1}}}
\expandafter\def\csname PY@tok@gt\endcsname{\def\PY@tc##1{\textcolor[rgb]{0.00,0.27,0.87}{##1}}}
\expandafter\def\csname PY@tok@err\endcsname{\def\PY@bc##1{\setlength{\fboxsep}{0pt}\fcolorbox[rgb]{1.00,0.00,0.00}{1,1,1}{\strut ##1}}}
\expandafter\def\csname PY@tok@kc\endcsname{\let\PY@bf=\textbf\def\PY@tc##1{\textcolor[rgb]{0.00,0.50,0.00}{##1}}}
\expandafter\def\csname PY@tok@kd\endcsname{\let\PY@bf=\textbf\def\PY@tc##1{\textcolor[rgb]{0.00,0.50,0.00}{##1}}}
\expandafter\def\csname PY@tok@kn\endcsname{\let\PY@bf=\textbf\def\PY@tc##1{\textcolor[rgb]{0.00,0.50,0.00}{##1}}}
\expandafter\def\csname PY@tok@kr\endcsname{\let\PY@bf=\textbf\def\PY@tc##1{\textcolor[rgb]{0.00,0.50,0.00}{##1}}}
\expandafter\def\csname PY@tok@bp\endcsname{\def\PY@tc##1{\textcolor[rgb]{0.00,0.50,0.00}{##1}}}
\expandafter\def\csname PY@tok@fm\endcsname{\def\PY@tc##1{\textcolor[rgb]{0.00,0.00,1.00}{##1}}}
\expandafter\def\csname PY@tok@vc\endcsname{\def\PY@tc##1{\textcolor[rgb]{0.10,0.09,0.49}{##1}}}
\expandafter\def\csname PY@tok@vg\endcsname{\def\PY@tc##1{\textcolor[rgb]{0.10,0.09,0.49}{##1}}}
\expandafter\def\csname PY@tok@vi\endcsname{\def\PY@tc##1{\textcolor[rgb]{0.10,0.09,0.49}{##1}}}
\expandafter\def\csname PY@tok@vm\endcsname{\def\PY@tc##1{\textcolor[rgb]{0.10,0.09,0.49}{##1}}}
\expandafter\def\csname PY@tok@sa\endcsname{\def\PY@tc##1{\textcolor[rgb]{0.73,0.13,0.13}{##1}}}
\expandafter\def\csname PY@tok@sb\endcsname{\def\PY@tc##1{\textcolor[rgb]{0.73,0.13,0.13}{##1}}}
\expandafter\def\csname PY@tok@sc\endcsname{\def\PY@tc##1{\textcolor[rgb]{0.73,0.13,0.13}{##1}}}
\expandafter\def\csname PY@tok@dl\endcsname{\def\PY@tc##1{\textcolor[rgb]{0.73,0.13,0.13}{##1}}}
\expandafter\def\csname PY@tok@s2\endcsname{\def\PY@tc##1{\textcolor[rgb]{0.73,0.13,0.13}{##1}}}
\expandafter\def\csname PY@tok@sh\endcsname{\def\PY@tc##1{\textcolor[rgb]{0.73,0.13,0.13}{##1}}}
\expandafter\def\csname PY@tok@s1\endcsname{\def\PY@tc##1{\textcolor[rgb]{0.73,0.13,0.13}{##1}}}
\expandafter\def\csname PY@tok@mb\endcsname{\def\PY@tc##1{\textcolor[rgb]{0.40,0.40,0.40}{##1}}}
\expandafter\def\csname PY@tok@mf\endcsname{\def\PY@tc##1{\textcolor[rgb]{0.40,0.40,0.40}{##1}}}
\expandafter\def\csname PY@tok@mh\endcsname{\def\PY@tc##1{\textcolor[rgb]{0.40,0.40,0.40}{##1}}}
\expandafter\def\csname PY@tok@mi\endcsname{\def\PY@tc##1{\textcolor[rgb]{0.40,0.40,0.40}{##1}}}
\expandafter\def\csname PY@tok@il\endcsname{\def\PY@tc##1{\textcolor[rgb]{0.40,0.40,0.40}{##1}}}
\expandafter\def\csname PY@tok@mo\endcsname{\def\PY@tc##1{\textcolor[rgb]{0.40,0.40,0.40}{##1}}}
\expandafter\def\csname PY@tok@ch\endcsname{\let\PY@it=\textit\def\PY@tc##1{\textcolor[rgb]{0.25,0.50,0.50}{##1}}}
\expandafter\def\csname PY@tok@cm\endcsname{\let\PY@it=\textit\def\PY@tc##1{\textcolor[rgb]{0.25,0.50,0.50}{##1}}}
\expandafter\def\csname PY@tok@cpf\endcsname{\let\PY@it=\textit\def\PY@tc##1{\textcolor[rgb]{0.25,0.50,0.50}{##1}}}
\expandafter\def\csname PY@tok@c1\endcsname{\let\PY@it=\textit\def\PY@tc##1{\textcolor[rgb]{0.25,0.50,0.50}{##1}}}
\expandafter\def\csname PY@tok@cs\endcsname{\let\PY@it=\textit\def\PY@tc##1{\textcolor[rgb]{0.25,0.50,0.50}{##1}}}

\def\PYZbs{\char`\\}
\def\PYZus{\char`\_}
\def\PYZob{\char`\{}
\def\PYZcb{\char`\}}
\def\PYZca{\char`\^}
\def\PYZam{\char`\&}
\def\PYZlt{\char`\<}
\def\PYZgt{\char`\>}
\def\PYZsh{\char`\#}
\def\PYZpc{\char`\%}
\def\PYZdl{\char`\$}
\def\PYZhy{\char`\-}
\def\PYZsq{\char`\'}
\def\PYZdq{\char`\"}
\def\PYZti{\char`\~}
% for compatibility with earlier versions
\def\PYZat{@}
\def\PYZlb{[}
\def\PYZrb{]}
\makeatother


    % Exact colors from NB
    \definecolor{incolor}{rgb}{0.0, 0.0, 0.5}
    \definecolor{outcolor}{rgb}{0.545, 0.0, 0.0}



    
    % Prevent overflowing lines due to hard-to-break entities
    \sloppy 
    % Setup hyperref package
    \hypersetup{
      breaklinks=true,  % so long urls are correctly broken across lines
      colorlinks=true,
      urlcolor=urlcolor,
      linkcolor=linkcolor,
      citecolor=citecolor,
      }
    % Slightly bigger margins than the latex defaults
    
    \geometry{verbose,tmargin=1in,bmargin=1in,lmargin=1in,rmargin=1in}
    
    

    \begin{document}
    
    
    \maketitle
    
    

    
    \section{Implementing a Multilayer Artificial Neural Network from
Scratch}\label{implementing-a-multilayer-artificial-neural-network-from-scratch}

    As you may know, deep learning is getting a lot of attention from the
press and is without any doubt the hottest topic in machine learning
field. Deep learning can be understood as a set of algorithms that were
developed to train artificial neural networks with many layers most
efficiently. In this chapter, you will learn the basic concepts of
artificial neural networks so that you will be well-equipped for the
following chapters, which will introduce advanced Python-based deep
learning libraries and \textbf{Deep Neural Networks (DNN)} architectures
that are particularly well-suited for image and text-analyses.

The topics that we will cover in this chapter are as follows:

\begin{itemize}
\tightlist
\item
  Getting a conceptual understanding of multilayer neural networks
\item
  Implementing the fundamental backpropagation algorithm for neural
  network training from scratch
\item
  Training a basic multilayer neural network for image classfication
\end{itemize}

    \section{Modeling complex functions with artificial neural
networks}\label{modeling-complex-functions-with-artificial-neural-networks}

    At the beginning of this book, we started our journey through machine
learning algorithms with artificial neurons. Artificial neurons
represent the building blocks of the multilayer artificial neural
networks that we will discuss in this chapter. The basic concept behind
artificial neural networks was built upon hypothesis and models of how
the human brain works to solve complex problem tasks. Although
artificial neural networks have gained a lot of popularity in recent
years, early studies of neural networks go back to the 1940s when Warren
McCulloch and Walter Pitt first described how neurons could work.

However, in the decades that followed the first implementation of the
\textbf{McCulloch-Pitt neuron} model - Rosenblatt's perceptron in the
1950s, many researchers and machine learning practitioners slowly began
to lose interest in neural networks since no one had a good solution for
training a neural network with multiple layers. Eventually, interest in
neural networks was rekindled in 1986 when D.E. Rumelhart, G.E. Hinton,
and R.J. Williams were involved in the (red)discovery and popularization
of the backpropagation algorithm to train neural networks more
efficiently, which we will discuss in more detail later in this chapter.
Readers who are interested in the history of \textbf{Artificial
Intelligence (AI)}, machine learning, and neural networks are also
encouraged to read the Wikipedia article on \emph{AI winter}, which are
periods of time where a large portion of the research community lost
interest in the study of neural networks.

However, neural networks have never been as popular as they are today,
thanks to the many major breakthroughs that have been made in the
previous decade, which resulted in what we now call deep learning
algorithms and architectures - neural networks that are composed of many
layers. Neural networks are a hot topic not only in academic research
but also in big technology companies such as Facebook, Microsoft, and
Google, who invest heavily in artificial neural networks and deep
learning research. As of today, complex neural networks powered by deep
learning algorithms are considered state of the art when it comes to
complex problem solving such as image and voice recognition. Popular
examples of the products in our everyday life that are powered by deep
learning are Google's image search and Google Translate - an application
for smartphones than can automatically recognize text in images for
real-time translation into more than 20 languages.

Many exciting applications of DNNs have been developed at major tech
companies and the pharmaceutical industry as listed in the following,
non-comprehensive list of examples:

\begin{itemize}
\tightlist
\item
  Facebook's DeepFace for tagging images
\item
  Baidu's DeepSpeech, which is able to handle voice queries in Mandarin
\item
  Google's new language translation service
\item
  Novel techniques for drug discovery and toxicity prediction
\item
  A mobile application that can detect skin cancer with an accuracy
  similar to professionally trained dermatologists
\end{itemize}

    \subsection{Single-layer neural network
recap}\label{single-layer-neural-network-recap}

    This chapter is all about multilayer neural networks, how they work, and
how to train them to solve complex problems. However, before we dig
deeper into a particular multilayer neural network architecture, let's
briefly reiterate some of the concepts of single-layer neural networks
that we introduced in the beginning of this book, namely, the
\textbf{ADAptive LInear NEuron (Adaline)} algorithm, which is shown is
the following figure:

Previously, we implemented the Adaline algorithm to perform binary
classification, and we used the gradient descent optimization algorithm
to learn the weight coefficients of the model. In every epoch (pass over
the training set), we updated the weight vector \(w\) using the
following update rule:

\[w := w + \Delta w, \text{where} \Delta w = \eta J(w)\]

In other words, we computed the gradient based on the whole training set
and updated the weights of the model by taking a step into the opposite
direction of the gradient \(J(w)\). In order to find the optimal weights
of the model, we optimized an objective function that we defined as the
\textbf{Sum of Squared Errors (SSE)} cost function \(J(w)\).
Furthermore, we multiplied the gradient by a factor, the learning rate
\(\eta\), which we had to choose carefully to balance the speed of
learning against the risk of overshooting the global minimum of the cost
function.

In gradient descent optimization, we updated all weights simultaneously
after each epoch, and we defined the partial derivative for each weight
\(w_j\) in the weight vector \(w\) as follows:

\[\frac{\partial}{\partial w_j}J(w) = -\sum_i{y^{(i)} - a^{(i)}}x^{(i)}_j\]

Here, \(y^{(i)}\) is the target class label of a particular sample
\(x^{(i)}\), and \(a^{(i)}\) is the activation of the neuron, which is a
linear function in the special case of Adaline. Furthermore, we defined
the activation function \(\phi\) as follows:

\[\phi(z) = z = a\]

Here, the net input \(z\) is a linear combination of the weights that
are connecting the input to the output layer:

\[z = \sum_j w_jx_j = w^Tx\]

While we used the activation \(\phi(z)\) to compute the gradient update,
we implemented a threshold function to squash the continuous valued
output into binary class labels for prediction:

\[ŷ = 1 \, \text{if} \, g(z) \ge 0; -1 \, \text{otherwise}\]

Note that although Adaline consists of two layers, one input layer and
one output layer, it is called single-layer network because of its
single link between the input and output layers.

Also, we learned about a certain trick to accelerate the model training,
the so-called \textbf{stochastic gradient descent} optimization.
Stochastic gradient descent approximates the cost from a single training
sample (online learning) or a small subset of training samples
(mini-batch learning). We will make use of this concept later in this
chapter when we implement and train a multilayer perceptron. Apart from
faster learning - due to the more frequent weight updates compared to
gradient descent - its noisy nature is also regarded as beneficial when
training multilayer neural networks with non-linear activation
functions, which do not have a convex cost function. Here, the added
noise can help to escape local cost minima, but we will discuss this
topic in more detail later in this chapter.

    \subsection{Introducing the multilayer neural network
architecture}\label{introducing-the-multilayer-neural-network-architecture}

    In this section, you will learn how to connect multiple single neurons
to a multilayer feedforward neural network; this special type of
\emph{fully connected} network is also called \textbf{Multilayer
Perceptron (MLP)}. The following picture illustrates the concept of an
MLP consisting of three layers:

The MLP depicted in the preceding figure has one input layer, one hidden
layer, and one output layer. The units in the hidden layear are fully
connected to the input layer. If such a network has more than one hidden
layer, we also call it a \textbf{deep artificial neural network}.

We can add an arbitrary number of hidden layers to the MLP to create
deeper network architectures. Practically, we can think of the number of
layers and units in a neural network as additional hyperparameters that
we want to optimize for a given problem task using cross-validation
techniques.

However, the error gradients that we will calculate later via
backpropagation will become increasingly small as more layers are added
to a network. This vanishing gradient problem makes the model learning
more challenging. Therefore, special algorithms have been developed to
help train such deep neural network structures; this is known as
\textbf{deep learning}.

As shown in the preceding figure, we denote the \emph{i}th activation
unit in the \emph{l}th layer as \(a_i^{(l)}\). To make the math and code
implementations a bit more intuitive, we will not use numerical indices
to refer to layers, but we will use the \emph{in} superscript for the
input layer, the \emph{h} superscript for the hidden layer, and the
\emph{out} superscript for the output layer. For instance,
\(a_i^{(in)}\) refers to the \emph{i}th value in the input layer,
\(a_i^{(h)}\) refers to the \emph{i}th unit in the hidden layer, adn
\(a_i^{(out)}\) refers to the \emph{i}th unit in the output layer. Here,
the activation unit \(a_0^{(in)}\) and \(a_0^{(h)}\) are the
\textbf{bias units}, which we set equal to 1. The activation of the
units in the input layer is just its input plus the unit bias.

Later in this chapter, we will implement the multilayer perceptron using
separate vectors for the bias unit, which makes the code implementation
more efficient and easier to read. This concept is also used by
TensorFlow, a deep learning library that we will introduce later.
However, the mathematical equations that will follow, would appear more
complex or convoluted if we had to work with additional variables for
the bias. However, note that the computation via appending 1s to the
input vector (as shown previously) and using a weight variable as bias
is exactly the same as operating with separate bias vectors; it is
merely a different convention.

Each unit in layer \emph{l} is connected to all units in layer \(l+1\)
vi a weight coefficient. For example, the connection between the \(k\)th
unit in the layer \(l\) to the \(j\)th unit in layer \(l+1\) will be
written as \(w_{k,j}^{l}\). Referring back to the previous figure, we
denote the weight matrix that connects the input to the hidden layer as
\(W^{(h)}\), and write the matrix that connects the hidden layer to the
output layer as \(W^{(out)}\).

While one unit in the output layer would suffice for a binary
classification task, we saw a more general form of neural network in the
preceding figure, which allows us to perform multiclass classification
via a generalization of the \textbf{One-versus-all (OvA)} technique. To
better understand how this works, remember the one-hot representation of
categorical variables. For example, we can encode the three class labels
in the familiar Iris dataset (0=setosa, 1=versicolor, 2=virginica) as
follows:

\[0 = [1, 0, 0], 1 = [0, 1, 0], 2 = [0, 0, 1]\]

This one-hot vector representation allows us to tackle classification
tasks with an arbitrary number of unique class labels present in the
training set.

If you are new to neural network representations, the indexing notation
(subscripts and superscripts) may look a little bit confusing at first.
What may seem overly complicated at first will make much more sense in
later section when we vectorize the neural network representation. As
introduced earlier, we summarize the weights that connect the input and
hidden layers by a matrix \(w^{(h)} \in \mathbb{R}^{m \times d}\), where
\(d\) is te number of hidden units and \(m\) is the number of input
units including the bias unit. Since it is important to internalize this
notation to follow the concepts later in this chapter, let's summarize
what we have just learned in a descriptive illustration of a simplified
3-4-3 multilayer perceptron:

    \subsection{Activating a neural network via forward
propagation}\label{activating-a-neural-network-via-forward-propagation}

    In this section, we will describe the process of \textbf{forward
propagation} to calculate the output of an MLP model. To understand how
it fits into the context of learning an MLP model, let's summarize the
MLP learning procedure in three simple steps:

\begin{enumerate}
\def\labelenumi{\arabic{enumi}.}
\tightlist
\item
  Starting at the input layer, we forward propagate the patterns of the
  training data through the network to generate an output.
\item
  Based on the network's output, we calculate the error that we want to
  minimize using a cost function that we will describe later.
\item
  We backpropagate the error, find its derivative with respect to each
  weight in the network, and update the model.
\end{enumerate}

Finally, after we repeat these three steps for multiple epochs and learn
the weights of the MLP, we use forward backpropagation to calculate the
network output and apply a threshold function to obtain the predicted
class labels in the one-hot representation, which we described in the
previous section.

Now, let's walk through the individual steps of forward propagation to
generate an output from the patters in the training data. Since each
unit in the hidden layer is connected to all units in the input layers,
we first calculate the activation unit of the hidden layer \(a_l^{(h)}\)
as follows:

\[z_1^{(h)} = a_0^{(in)}w_{0,1}^{(h)} + a_1^{(in)}w_{1,1}^{(h)} + \ldots + a_m^{(in)}w_{m,1}^{(h)}\]

\[a_1^{(h)} = \phi(z_1^{(h)})\]

Here, \(z_1^{(h)}\) is the net input and \(\phi\) is the activation
function, which has to be differentiable to learn the weights that
connect the neurons using a gradient-based approach. To be able to solve
complex problem such as image classification, we need non-linear
activation functions in our MLP model, for example, the sigmoid
(logistic) activation function that we remember from the section about
logistic regression:

\[\phi(z) = \frac{1}{1 + e^{-z}}\]

As we can remember, the sigmoid function is an S-shaped curve that maps
the net input \(z\) onto a logistic distribution in the range 0 to 1,
which cuts the \(y\)-axis at \(z=0\), as shown in the following graph:

MLP is a typical example of a feedforward artificial neural network. The
term \textbf{feedforward} refers to the fact that each layer serves as
the input to the next layer without loops, in contrast to recurrent
neural network - an architecture that we will discuss later in this
chapter and discuss in more detail in the last chapter of this book. The
term \emph{multilayer perceptron} may sound a little bit confusing since
the artificial neurons in this network architecture are typically
sigmoid units, not perceptrons. Intuitively, we can think of the neurons
in the MLP as logistic regression units that return values in the
continuous range between 0 and 1.

For purposes of code efficiency and readability, we will now write the
activation in a more compact form using the concepts of basic linear
algebra, which will allow us to vectorize our code implementation via
NumPy rather than writing multiple nested and computationally expensive
Python \emph{for} loops:

\[z^{(h)} = a^{(in)}W^{(h)}\]

\[a^{(h)} = \phi(z^{(h)})\]

Here, \(a^{(in)}\) is our \(l \times m\) dimensional feature vector of a
sample \(x^{(in)}\) plus a bias unit. \(W^{(h)}\) is an \(m \times d\)
dimensional weight matrix where \(d\) is the number of units in the
hidden layer. After matrix-vector multiplication, we obtain the
\(l \times d\) dimensional net input \(z^{(h)}\) to calculate the
activation \(a^{(h)}\) (where \(a^{(h)} \in \mathbb{R}^{l \times d}\)).
Furthermore, we can generalize this computation to all \(n\) samples in
the training set:

\[Z^{(h)} = A^{(in)}W^{(h)}\]

Here, \$ A\^{}\{(in)\}\$ is not an \(n \times m\) matrix, and the
matrix-matrix multiplication will result in an \(n \times d\)
dimensional net input matrix \(Z^{(h)}\). Finally, we apply the
activation function \(\phi\) to each value in the net input matrix to
get the \(n \times d\) activation matrix \(A^{(h)}\) for the next layer
(here, the output layer):

\[a^{(h)} = \phi(z^{(h)})\]

Similarly, we can write the activation of the output layer in vectorized
form for multiple samples:

\[Z^{(out)} = A^{(h)}W^{(out)}\]

Here, we multiply the \(d \times t\) matrix \(W^{(out)}\) (\(t\) is the
number of output units) by the \(n \times d\) dimensional matrix
\(A^{(h)}\) to obtain the \(n \times t\) dimensional matrix
\(Z^{(out)}\) (the columns in this matrix represent the output for each
sample).

Lastly, we apply the sigmoid activation function to obtain the
continuous valued output of our network:

\[a^{(out)} = \phi(z^{(out)})\]

    \subsection{Classifying handwritten
digits}\label{classifying-handwritten-digits}

    In the previous section, we covered a lot of the theory around neural
networks, which can be a little bit overwhelming if you are new to this
topic. Before we continue with the discussion of the algorithm for
learning the weights of the MLP model, backpropagation, let's take a
short break from the theory and see a neural network in action.

In this section, we will implement and train our first multilayer neural
network to classify handwritten digits from the popular \textbf{Mixed
National Institute of Standards and Technology (MNIST)} dataset that has
been constructed by Yann LeCun and others, and servers as a popular
benchmark dataset for machine learning algorithms.

    \subsubsection{Obtaining the MNIST
dataset}\label{obtaining-the-mnist-dataset}

    The MNIST dataset is publicly available at
http://yann.lecun.com/exdb/mnist/ and consists of the following four
parts:

\begin{itemize}
\tightlist
\item
  Training set images: \emph{train-images-idx3-ubyte.gz} (60,000
  samples)
\item
  Training set labels: \emph{train-labels-idx1-ubyte.gz} (60,000 labels)
\item
  Test set images: \emph{t10k-images-idx3-ubyte.gz} (10,000 samples)
\item
  Test set labels: \emph{t10k-labels-idx1-ubyte.gz} (10,000 labels)
\end{itemize}

The MNIST dataset was constructed from two datasets from the US
\textbf{National Institute of Standards and Technology (NIST)}. The
training dataset consists of handwritten digits from 250 different
people, 50 percent high school students, and 50 percent employees from
the Census Bureau. Note that the test set contains handwritten digits
from different people following the same split.

The images are stored in byte format, and we will read them into NumPy
arrays that we will use to train and test our MLP implementation. In
order to do that, we will define the following helper function:

    \begin{Verbatim}[commandchars=\\\{\}]
{\color{incolor}In [{\color{incolor}1}]:} \PY{k+kn}{import} \PY{n+nn}{os} 
        \PY{k+kn}{import} \PY{n+nn}{struct}
        \PY{k+kn}{import} \PY{n+nn}{numpy} \PY{k}{as} \PY{n+nn}{np}
        
        \PY{k}{def} \PY{n+nf}{load\PYZus{}mnist}\PY{p}{(}\PY{n}{path}\PY{p}{,} \PY{n}{kind}\PY{o}{=}\PY{l+s+s1}{\PYZsq{}}\PY{l+s+s1}{train}\PY{l+s+s1}{\PYZsq{}}\PY{p}{)}\PY{p}{:}
            \PY{l+s+sd}{\PYZdq{}\PYZdq{}\PYZdq{}Load MNIST data from \PYZsq{}path\PYZsq{}\PYZdq{}\PYZdq{}\PYZdq{}}
            \PY{n}{labels\PYZus{}path} \PY{o}{=} \PY{n}{os}\PY{o}{.}\PY{n}{path}\PY{o}{.}\PY{n}{join}\PY{p}{(}\PY{n}{path}\PY{p}{,} \PY{l+s+s1}{\PYZsq{}}\PY{l+s+si}{\PYZpc{}s}\PY{l+s+s1}{\PYZhy{}labels\PYZhy{}idx1\PYZhy{}ubyte}\PY{l+s+s1}{\PYZsq{}} \PY{o}{\PYZpc{}} \PY{n}{kind}\PY{p}{)}
            \PY{n}{images\PYZus{}path} \PY{o}{=} \PY{n}{os}\PY{o}{.}\PY{n}{path}\PY{o}{.}\PY{n}{join}\PY{p}{(}\PY{n}{path}\PY{p}{,} \PY{l+s+s1}{\PYZsq{}}\PY{l+s+si}{\PYZpc{}s}\PY{l+s+s1}{\PYZhy{}images\PYZhy{}idx3\PYZhy{}ubyte}\PY{l+s+s1}{\PYZsq{}} \PY{o}{\PYZpc{}} \PY{n}{kind}\PY{p}{)}
            
            \PY{k}{with} \PY{n+nb}{open}\PY{p}{(}\PY{n}{labels\PYZus{}path}\PY{p}{,} \PY{l+s+s1}{\PYZsq{}}\PY{l+s+s1}{rb}\PY{l+s+s1}{\PYZsq{}}\PY{p}{)} \PY{k}{as} \PY{n}{lbpath}\PY{p}{:}
                \PY{n}{magic}\PY{p}{,} \PY{n}{n} \PY{o}{=} \PY{n}{struct}\PY{o}{.}\PY{n}{unpack}\PY{p}{(}\PY{l+s+s1}{\PYZsq{}}\PY{l+s+s1}{\PYZgt{}II}\PY{l+s+s1}{\PYZsq{}}\PY{p}{,} \PY{n}{lbpath}\PY{o}{.}\PY{n}{read}\PY{p}{(}\PY{l+m+mi}{8}\PY{p}{)}\PY{p}{)}
                \PY{n}{labels} \PY{o}{=} \PY{n}{np}\PY{o}{.}\PY{n}{fromfile}\PY{p}{(}\PY{n}{lbpath}\PY{p}{,} \PY{n}{dtype}\PY{o}{=}\PY{n}{np}\PY{o}{.}\PY{n}{uint8}\PY{p}{)}
                
            \PY{k}{with} \PY{n+nb}{open}\PY{p}{(}\PY{n}{images\PYZus{}path}\PY{p}{,} \PY{l+s+s1}{\PYZsq{}}\PY{l+s+s1}{rb}\PY{l+s+s1}{\PYZsq{}}\PY{p}{)} \PY{k}{as} \PY{n}{imgpath}\PY{p}{:}
                \PY{n}{magic}\PY{p}{,} \PY{n}{num}\PY{p}{,} \PY{n}{rows}\PY{p}{,} \PY{n}{cols} \PY{o}{=} \PY{n}{struct}\PY{o}{.}\PY{n}{unpack}\PY{p}{(}\PY{l+s+s2}{\PYZdq{}}\PY{l+s+s2}{\PYZgt{}IIII}\PY{l+s+s2}{\PYZdq{}}\PY{p}{,} \PY{n}{imgpath}\PY{o}{.}\PY{n}{read}\PY{p}{(}\PY{l+m+mi}{16}\PY{p}{)}\PY{p}{)}
                \PY{n}{images} \PY{o}{=} \PY{n}{np}\PY{o}{.}\PY{n}{fromfile}\PY{p}{(}\PY{n}{imgpath}\PY{p}{,} \PY{n}{dtype}\PY{o}{=}\PY{n}{np}\PY{o}{.}\PY{n}{uint8}\PY{p}{)}\PY{o}{.}\PY{n}{reshape}\PY{p}{(}\PY{n+nb}{len}\PY{p}{(}\PY{n}{labels}\PY{p}{)}\PY{p}{,} \PY{l+m+mi}{784}\PY{p}{)}
                \PY{n}{images} \PY{o}{=} \PY{p}{(}\PY{p}{(}\PY{n}{images} \PY{o}{/} \PY{l+m+mf}{255.}\PY{p}{)} \PY{o}{\PYZhy{}} \PY{o}{.}\PY{l+m+mi}{5}\PY{p}{)} \PY{o}{*} \PY{l+m+mi}{2}
            
            \PY{k}{return} \PY{n}{images}\PY{p}{,} \PY{n}{labels}
\end{Verbatim}


    The \emph{load\_mnist} function returns two arrays, the first being an
\(n \times m\) dimensional NumPy array (\emph{images}), where \(n\) is
the number of samples and \(m\) is the number of features (here,
pixels). The training dataset consists of 60,000 training digits and the
test set contains 10,000 samples, respectively. The images in the MNIST
dataset consist of 28 x 28 pixels, and each pixel is represented by a
gray scale intensity value. Here, we unroll the 28 x 28 pixels into
one-dimensional row vectors, which represent the rows in our
\emph{images} array (784 per row or image). The second array
(\emph{labels}) returned by the \emph{load\_mnist} function contains the
corresponding target variable, the class labels (integers 0-9) of the
handwritten digits.

We normalized the pixels values in the MNIST to the range -1 to 1
(originally 0 to 255). The reason behind this is that gradient-based
optimization is much more stable under these conditions. Note that we
scaled the images on a pixel-by-pixel basis, which is different from the
feature scaling approach that we took in previous chapters. Previously,
we derived scaling paramters from the training set and used these to
scale each column in the training set and test set. However, when
working with image pixels, centering them at zero and rescaling them to
a {[}-1, 1{]} range is also common and usually works in practice.

Another recently developed trick to improve convergence in
gradient-based optimization through input scaling is batch
normalization, which is an advanced topic that we will not cover in this
book.

By executing the following code, we will now load the 60,000 training
instances as well as the 10,000 test samples from the local directories
where we unzipped the MNIST dataset (in the following code snippet, it
is assumed that the downloaded MNIST files were unzipped to the same
directory in which this code was executed):

    \begin{Verbatim}[commandchars=\\\{\}]
{\color{incolor}In [{\color{incolor}2}]:} \PY{n}{X\PYZus{}train}\PY{p}{,} \PY{n}{y\PYZus{}train} \PY{o}{=} \PY{n}{load\PYZus{}mnist}\PY{p}{(}\PY{l+s+s1}{\PYZsq{}}\PY{l+s+s1}{\PYZsq{}}\PY{p}{,} \PY{n}{kind}\PY{o}{=}\PY{l+s+s1}{\PYZsq{}}\PY{l+s+s1}{train}\PY{l+s+s1}{\PYZsq{}}\PY{p}{)}
        \PY{n+nb}{print}\PY{p}{(}\PY{l+s+s1}{\PYZsq{}}\PY{l+s+s1}{Rows: }\PY{l+s+si}{\PYZpc{}d}\PY{l+s+s1}{, columns: }\PY{l+s+si}{\PYZpc{}d}\PY{l+s+s1}{\PYZsq{}} \PY{o}{\PYZpc{}} \PY{p}{(}\PY{n}{X\PYZus{}train}\PY{o}{.}\PY{n}{shape}\PY{p}{[}\PY{l+m+mi}{0}\PY{p}{]}\PY{p}{,} \PY{n}{X\PYZus{}train}\PY{o}{.}\PY{n}{shape}\PY{p}{[}\PY{l+m+mi}{1}\PY{p}{]}\PY{p}{)}\PY{p}{)}
        
        \PY{n}{X\PYZus{}test}\PY{p}{,} \PY{n}{y\PYZus{}test} \PY{o}{=} \PY{n}{load\PYZus{}mnist}\PY{p}{(}\PY{l+s+s1}{\PYZsq{}}\PY{l+s+s1}{\PYZsq{}}\PY{p}{,} \PY{n}{kind}\PY{o}{=}\PY{l+s+s1}{\PYZsq{}}\PY{l+s+s1}{t10k}\PY{l+s+s1}{\PYZsq{}}\PY{p}{)}
        \PY{n+nb}{print}\PY{p}{(}\PY{l+s+s1}{\PYZsq{}}\PY{l+s+s1}{Rows: }\PY{l+s+si}{\PYZpc{}d}\PY{l+s+s1}{, columns: }\PY{l+s+si}{\PYZpc{}d}\PY{l+s+s1}{\PYZsq{}} \PY{o}{\PYZpc{}} \PY{p}{(}\PY{n}{X\PYZus{}test}\PY{o}{.}\PY{n}{shape}\PY{p}{[}\PY{l+m+mi}{0}\PY{p}{]}\PY{p}{,} \PY{n}{X\PYZus{}test}\PY{o}{.}\PY{n}{shape}\PY{p}{[}\PY{l+m+mi}{1}\PY{p}{]}\PY{p}{)}\PY{p}{)}
\end{Verbatim}


    \begin{Verbatim}[commandchars=\\\{\}]
Rows: 60000, columns: 784
Rows: 10000, columns: 784

    \end{Verbatim}

    To get an idea of how those images in MNIST look, let's visualize
examples of the digits 0-9 after reshaping the 784-pixel vectors from
our feature matrix into the original 28 x 28 image that we can plot via
Matplotlib's \emph{imshow} function:

    \begin{Verbatim}[commandchars=\\\{\}]
{\color{incolor}In [{\color{incolor}3}]:} \PY{k+kn}{import} \PY{n+nn}{matplotlib}\PY{n+nn}{.}\PY{n+nn}{pyplot} \PY{k}{as} \PY{n+nn}{plt}
        
        \PY{n}{fig}\PY{p}{,} \PY{n}{ax} \PY{o}{=} \PY{n}{plt}\PY{o}{.}\PY{n}{subplots}\PY{p}{(}\PY{n}{nrows}\PY{o}{=}\PY{l+m+mi}{2}\PY{p}{,} \PY{n}{ncols}\PY{o}{=}\PY{l+m+mi}{5}\PY{p}{,} \PY{n}{sharex}\PY{o}{=}\PY{k+kc}{True}\PY{p}{,} \PY{n}{sharey}\PY{o}{=}\PY{k+kc}{True}\PY{p}{)}
        
        \PY{n}{ax} \PY{o}{=} \PY{n}{ax}\PY{o}{.}\PY{n}{flatten}\PY{p}{(}\PY{p}{)}
        \PY{k}{for} \PY{n}{i} \PY{o+ow}{in} \PY{n+nb}{range}\PY{p}{(}\PY{l+m+mi}{10}\PY{p}{)}\PY{p}{:}
            \PY{n}{img} \PY{o}{=} \PY{n}{X\PYZus{}train}\PY{p}{[}\PY{n}{y\PYZus{}train}\PY{o}{==}\PY{n}{i}\PY{p}{]}\PY{p}{[}\PY{l+m+mi}{0}\PY{p}{]}\PY{o}{.}\PY{n}{reshape}\PY{p}{(}\PY{l+m+mi}{28}\PY{p}{,} \PY{l+m+mi}{28}\PY{p}{)}
            \PY{n}{ax}\PY{p}{[}\PY{n}{i}\PY{p}{]}\PY{o}{.}\PY{n}{imshow}\PY{p}{(}\PY{n}{img}\PY{p}{,} \PY{n}{cmap}\PY{o}{=}\PY{l+s+s1}{\PYZsq{}}\PY{l+s+s1}{Greys}\PY{l+s+s1}{\PYZsq{}}\PY{p}{)}
        
        \PY{n}{ax}\PY{p}{[}\PY{l+m+mi}{0}\PY{p}{]}\PY{o}{.}\PY{n}{set\PYZus{}xticks}\PY{p}{(}\PY{p}{[}\PY{p}{]}\PY{p}{)}
        \PY{n}{ax}\PY{p}{[}\PY{l+m+mi}{0}\PY{p}{]}\PY{o}{.}\PY{n}{set\PYZus{}yticks}\PY{p}{(}\PY{p}{[}\PY{p}{]}\PY{p}{)}
        \PY{n}{plt}\PY{o}{.}\PY{n}{tight\PYZus{}layout}\PY{p}{(}\PY{p}{)}
        \PY{n}{plt}\PY{o}{.}\PY{n}{show}\PY{p}{(}\PY{p}{)}
\end{Verbatim}


    \begin{center}
    \adjustimage{max size={0.9\linewidth}{0.9\paperheight}}{output_18_0.png}
    \end{center}
    { \hspace*{\fill} \\}
    
    We should now see a plot of the 2 x 5 subfigures showing a
representative image of each unique digit.

In addition, let's also plot multiple examples of the same digit to see
how different the handwritting really is:

    \begin{Verbatim}[commandchars=\\\{\}]
{\color{incolor}In [{\color{incolor}4}]:} \PY{n}{fig}\PY{p}{,} \PY{n}{ax} \PY{o}{=} \PY{n}{plt}\PY{o}{.}\PY{n}{subplots}\PY{p}{(}\PY{n}{nrows}\PY{o}{=}\PY{l+m+mi}{5}\PY{p}{,} \PY{n}{ncols}\PY{o}{=}\PY{l+m+mi}{5}\PY{p}{,} \PY{n}{sharex}\PY{o}{=}\PY{k+kc}{True}\PY{p}{,} \PY{n}{sharey}\PY{o}{=}\PY{k+kc}{True}\PY{p}{)}
        
        \PY{n}{ax} \PY{o}{=} \PY{n}{ax}\PY{o}{.}\PY{n}{flatten}\PY{p}{(}\PY{p}{)}
        \PY{k}{for} \PY{n}{i} \PY{o+ow}{in} \PY{n+nb}{range}\PY{p}{(}\PY{l+m+mi}{25}\PY{p}{)}\PY{p}{:}
            \PY{n}{img} \PY{o}{=} \PY{n}{X\PYZus{}train}\PY{p}{[}\PY{n}{y\PYZus{}train}\PY{o}{==}\PY{l+m+mi}{7}\PY{p}{]}\PY{p}{[}\PY{n}{i}\PY{p}{]}\PY{o}{.}\PY{n}{reshape}\PY{p}{(}\PY{l+m+mi}{28}\PY{p}{,} \PY{l+m+mi}{28}\PY{p}{)}
            \PY{n}{ax}\PY{p}{[}\PY{n}{i}\PY{p}{]}\PY{o}{.}\PY{n}{imshow}\PY{p}{(}\PY{n}{img}\PY{p}{,} \PY{n}{cmap}\PY{o}{=}\PY{l+s+s1}{\PYZsq{}}\PY{l+s+s1}{Greys}\PY{l+s+s1}{\PYZsq{}}\PY{p}{)}
        
        \PY{n}{ax}\PY{p}{[}\PY{l+m+mi}{0}\PY{p}{]}\PY{o}{.}\PY{n}{set\PYZus{}xticks}\PY{p}{(}\PY{p}{[}\PY{p}{]}\PY{p}{)}
        \PY{n}{ax}\PY{p}{[}\PY{l+m+mi}{0}\PY{p}{]}\PY{o}{.}\PY{n}{set\PYZus{}yticks}\PY{p}{(}\PY{p}{[}\PY{p}{]}\PY{p}{)}
        \PY{n}{plt}\PY{o}{.}\PY{n}{tight\PYZus{}layout}\PY{p}{(}\PY{p}{)}
        \PY{n}{plt}\PY{o}{.}\PY{n}{show}\PY{p}{(}\PY{p}{)}
\end{Verbatim}


    \begin{center}
    \adjustimage{max size={0.9\linewidth}{0.9\paperheight}}{output_20_0.png}
    \end{center}
    { \hspace*{\fill} \\}
    
    After executing the code, we should now see the first 25 variantes of
the digit 7.

After we have gone through all the previous steps, it is a good idea to
save the scaled images in a format that we can load more quickly into a
new Python session to avoid the overhead of reading in and processing
the data again. When we are working with NumPy arrays, an efficient yet
most convenient method to save multidimensional arrays to disk is
NumPy's \emph{savez} function.

In short, the \emph{savez} function is analogous to Python's
\emph{pickle} module that we used before, but optimized for storing
NumPy arrays. The \emph{savez} function produces zipped archives of our
data, producing \emph{.npz} files that contain files in the \emph{.npz}
format. Further, instead of using \emph{savez}, we will use
\emph{savez\_compressed}, which uses the same syntax as \emph{savez},
but further compresses the output file down to substantially smaller
file sizes. The following code snippet will save both the training and
test datasets to the archive file \emph{'mnist\_scaled.npz'}:

    \begin{Verbatim}[commandchars=\\\{\}]
{\color{incolor}In [{\color{incolor}5}]:} \PY{k+kn}{import} \PY{n+nn}{numpy} \PY{k}{as} \PY{n+nn}{np}
        
        \PY{n}{np}\PY{o}{.}\PY{n}{savez\PYZus{}compressed}\PY{p}{(}\PY{l+s+s1}{\PYZsq{}}\PY{l+s+s1}{mnist\PYZus{}scaled.npz}\PY{l+s+s1}{\PYZsq{}}\PY{p}{,} 
                            \PY{n}{X\PYZus{}train}\PY{o}{=}\PY{n}{X\PYZus{}train}\PY{p}{,} 
                            \PY{n}{y\PYZus{}train}\PY{o}{=}\PY{n}{y\PYZus{}train}\PY{p}{,} 
                            \PY{n}{X\PYZus{}test}\PY{o}{=}\PY{n}{X\PYZus{}test}\PY{p}{,} 
                            \PY{n}{y\PYZus{}test}\PY{o}{=}\PY{n}{y\PYZus{}test}\PY{p}{)}
\end{Verbatim}


    After we created the \emph{.npz} files, we can load the preprocessed
MNIST images arrays using NumPy's \emph{load} function as follows:

    \begin{Verbatim}[commandchars=\\\{\}]
{\color{incolor}In [{\color{incolor}6}]:} \PY{n}{mnist} \PY{o}{=} \PY{n}{np}\PY{o}{.}\PY{n}{load}\PY{p}{(}\PY{l+s+s1}{\PYZsq{}}\PY{l+s+s1}{mnist\PYZus{}scaled.npz}\PY{l+s+s1}{\PYZsq{}}\PY{p}{)}
\end{Verbatim}


    The \emph{mnist} variable now references to an object that can access
the four data arrays as we provided them keywork arguments to the
\emph{savez\_compressed} function, which are listed under the files
attribute list of the \emph{mnist} object:

    \begin{Verbatim}[commandchars=\\\{\}]
{\color{incolor}In [{\color{incolor}7}]:} \PY{n}{mnist}\PY{o}{.}\PY{n}{files}
\end{Verbatim}


\begin{Verbatim}[commandchars=\\\{\}]
{\color{outcolor}Out[{\color{outcolor}7}]:} ['X\_train', 'y\_train', 'X\_test', 'y\_test']
\end{Verbatim}
            
    For instance, to load the training data into our current Python session,
we will access the \emph{'X\_train'} array as follows (similar to a
Python dictionary):

    \begin{Verbatim}[commandchars=\\\{\}]
{\color{incolor}In [{\color{incolor}8}]:} \PY{n}{X\PYZus{}train} \PY{o}{=} \PY{n}{mnist}\PY{p}{[}\PY{l+s+s1}{\PYZsq{}}\PY{l+s+s1}{X\PYZus{}train}\PY{l+s+s1}{\PYZsq{}}\PY{p}{]}
\end{Verbatim}


    Using a list comprehension, we can retrieve all four data arrays as
follows:

    \begin{Verbatim}[commandchars=\\\{\}]
{\color{incolor}In [{\color{incolor}9}]:} \PY{n}{X\PYZus{}train}\PY{p}{,} \PY{n}{y\PYZus{}train}\PY{p}{,} \PY{n}{X\PYZus{}test}\PY{p}{,} \PY{n}{y\PYZus{}test} \PY{o}{=} \PY{p}{[}\PY{n}{mnist}\PY{p}{[}\PY{n}{f}\PY{p}{]} \PY{k}{for} \PY{n}{f} \PY{o+ow}{in} \PY{n}{mnist}\PY{o}{.}\PY{n}{files}\PY{p}{]}
\end{Verbatim}


    Note that while the preceding \emph{np.savez\_compressed} and
\emph{np.load} examples are not essential for executing the code at this
chapter, it serves as a demonstration of how to save and load NumPy
arrays conveniently and efficiently.

    \subsection{Implementing a multilayer
perceptron}\label{implementing-a-multilayer-perceptron}

    In subsection, we will now implement the code of an MLP with one input,
one hidden, and one output layers to classify the images in the MNIST
dataset. I have tried to keep the code as simple as possible. However,
it may seem a little bit complicated at first.

The code will contain parts that we have not talked about yet, such as
the backpropagation algorithm, but most of the code should look familiar
to you based on the Adaline implementation, and the discussion of
forward propagation in earlier sections.

DO not worry if not all the code makes immediate sense to you; we will
follow up on certain parts later in this chapter. However, going over
the code at this stage can make it easier to follow the theory later.

The following is the implementation of a multilayer perceptron:

    \begin{Verbatim}[commandchars=\\\{\}]
{\color{incolor}In [{\color{incolor}15}]:} \PY{k+kn}{import} \PY{n+nn}{numpy} \PY{k}{as} \PY{n+nn}{np}
         \PY{k+kn}{import} \PY{n+nn}{sys}
         
         \PY{k}{class} \PY{n+nc}{NeuralNetMLP}\PY{p}{(}\PY{n+nb}{object}\PY{p}{)}\PY{p}{:}
             \PY{l+s+sd}{\PYZdq{}\PYZdq{}\PYZdq{}Feedforward neural network / Multi\PYZhy{}layer perceptron classifier}
         \PY{l+s+sd}{    }
         \PY{l+s+sd}{    Parameters}
         \PY{l+s+sd}{    \PYZhy{}\PYZhy{}\PYZhy{}\PYZhy{}\PYZhy{}\PYZhy{}\PYZhy{}\PYZhy{}\PYZhy{}\PYZhy{}\PYZhy{}\PYZhy{}\PYZhy{}\PYZhy{}\PYZhy{}}
         \PY{l+s+sd}{    n\PYZus{}hidden : int (default: 30)}
         \PY{l+s+sd}{        Number of hidden units}
         \PY{l+s+sd}{    l2 : float (default: 0.)}
         \PY{l+s+sd}{        Lambda value for L2\PYZhy{}regularization.}
         \PY{l+s+sd}{        No regularization if l2=0. (default)}
         \PY{l+s+sd}{    epochs : int (default: 100)}
         \PY{l+s+sd}{        Number of passes over the training set}
         \PY{l+s+sd}{    eta : float (default: 0.001)}
         \PY{l+s+sd}{        Learning rate}
         \PY{l+s+sd}{    shuffle : bool (default: True)}
         \PY{l+s+sd}{        Shuffles training data every epoch}
         \PY{l+s+sd}{        if True to prevent cycles}
         \PY{l+s+sd}{    minibatch\PYZus{}size : int (default: 1)}
         \PY{l+s+sd}{        Number of training samples per minibatch}
         \PY{l+s+sd}{    seed : int (default: None)}
         \PY{l+s+sd}{        Random seed for initializing weights and shuffling}
         \PY{l+s+sd}{        }
         \PY{l+s+sd}{    Attributes}
         \PY{l+s+sd}{    \PYZhy{}\PYZhy{}\PYZhy{}\PYZhy{}\PYZhy{}\PYZhy{}\PYZhy{}\PYZhy{}\PYZhy{}\PYZhy{}\PYZhy{}\PYZhy{}\PYZhy{}\PYZhy{}\PYZhy{}}
         \PY{l+s+sd}{    eval\PYZus{} : dict}
         \PY{l+s+sd}{        Dictionary collecting the cost, training accuracy, }
         \PY{l+s+sd}{        and validation accuracy for each epoch during training}
         \PY{l+s+sd}{    \PYZdq{}\PYZdq{}\PYZdq{}}
             \PY{k}{def} \PY{n+nf}{\PYZus{}\PYZus{}init\PYZus{}\PYZus{}}\PY{p}{(}\PY{n+nb+bp}{self}\PY{p}{,} \PY{n}{n\PYZus{}hidden}\PY{o}{=}\PY{l+m+mi}{30}\PY{p}{,} \PY{n}{l2}\PY{o}{=}\PY{l+m+mf}{0.}\PY{p}{,} \PY{n}{epochs}\PY{o}{=}\PY{l+m+mi}{100}\PY{p}{,} 
                          \PY{n}{eta}\PY{o}{=}\PY{l+m+mf}{0.001}\PY{p}{,} \PY{n}{shuffle}\PY{o}{=}\PY{k+kc}{True}\PY{p}{,} \PY{n}{minibatch\PYZus{}size}\PY{o}{=}\PY{l+m+mi}{1}\PY{p}{,} 
                          \PY{n}{seed}\PY{o}{=}\PY{k+kc}{None}\PY{p}{)}\PY{p}{:}
                 \PY{n+nb+bp}{self}\PY{o}{.}\PY{n}{random} \PY{o}{=} \PY{n}{np}\PY{o}{.}\PY{n}{random}\PY{o}{.}\PY{n}{RandomState}\PY{p}{(}\PY{n}{seed}\PY{p}{)}
                 \PY{n+nb+bp}{self}\PY{o}{.}\PY{n}{n\PYZus{}hidden} \PY{o}{=} \PY{n}{n\PYZus{}hidden}
                 \PY{n+nb+bp}{self}\PY{o}{.}\PY{n}{l2} \PY{o}{=} \PY{n}{l2}
                 \PY{n+nb+bp}{self}\PY{o}{.}\PY{n}{epochs} \PY{o}{=} \PY{n}{epochs}
                 \PY{n+nb+bp}{self}\PY{o}{.}\PY{n}{eta} \PY{o}{=} \PY{n}{eta}
                 \PY{n+nb+bp}{self}\PY{o}{.}\PY{n}{shuffle} \PY{o}{=} \PY{n}{shuffle}
                 \PY{n+nb+bp}{self}\PY{o}{.}\PY{n}{minibatch\PYZus{}size} \PY{o}{=} \PY{n}{minibatch\PYZus{}size}
                 
             \PY{k}{def} \PY{n+nf}{\PYZus{}onehot}\PY{p}{(}\PY{n+nb+bp}{self}\PY{p}{,} \PY{n}{y}\PY{p}{,} \PY{n}{n\PYZus{}classes}\PY{p}{)}\PY{p}{:}
                 \PY{l+s+sd}{\PYZdq{}\PYZdq{}\PYZdq{}Encode labels into one\PYZhy{}hot representation}
         \PY{l+s+sd}{        }
         \PY{l+s+sd}{        Parameters}
         \PY{l+s+sd}{        \PYZhy{}\PYZhy{}\PYZhy{}\PYZhy{}\PYZhy{}\PYZhy{}\PYZhy{}\PYZhy{}\PYZhy{}\PYZhy{}\PYZhy{}\PYZhy{}\PYZhy{}\PYZhy{}\PYZhy{}}
         \PY{l+s+sd}{        y : array, shape = [n\PYZus{}samples]}
         \PY{l+s+sd}{            Target values}
         \PY{l+s+sd}{            }
         \PY{l+s+sd}{        Returns}
         \PY{l+s+sd}{        \PYZhy{}\PYZhy{}\PYZhy{}\PYZhy{}\PYZhy{}\PYZhy{}\PYZhy{}\PYZhy{}\PYZhy{}\PYZhy{}\PYZhy{}\PYZhy{}\PYZhy{}\PYZhy{}\PYZhy{}}
         \PY{l+s+sd}{        onehot : array = shape = (n\PYZus{}samples, n\PYZus{}labels)        }
         \PY{l+s+sd}{        \PYZdq{}\PYZdq{}\PYZdq{}}
                 \PY{n}{onehot} \PY{o}{=} \PY{n}{np}\PY{o}{.}\PY{n}{zeros}\PY{p}{(}\PY{p}{(}\PY{n}{n\PYZus{}classes}\PY{p}{,} \PY{n}{y}\PY{o}{.}\PY{n}{shape}\PY{p}{[}\PY{l+m+mi}{0}\PY{p}{]}\PY{p}{)}\PY{p}{)}
                 \PY{k}{for} \PY{n}{idx}\PY{p}{,} \PY{n}{val} \PY{o+ow}{in} \PY{n+nb}{enumerate}\PY{p}{(}\PY{n}{y}\PY{o}{.}\PY{n}{astype}\PY{p}{(}\PY{n+nb}{int}\PY{p}{)}\PY{p}{)}\PY{p}{:}
                     \PY{n}{onehot}\PY{p}{[}\PY{n}{val}\PY{p}{,} \PY{n}{idx}\PY{p}{]} \PY{o}{=} \PY{l+m+mf}{1.}
                 \PY{k}{return} \PY{n}{onehot}\PY{o}{.}\PY{n}{T}
             
             \PY{k}{def} \PY{n+nf}{\PYZus{}sigmoid}\PY{p}{(}\PY{n+nb+bp}{self}\PY{p}{,} \PY{n}{z}\PY{p}{)}\PY{p}{:}
                 \PY{l+s+sd}{\PYZdq{}\PYZdq{}\PYZdq{}Compute logistic function (sigmoid)\PYZdq{}\PYZdq{}\PYZdq{}}
                 \PY{k}{return} \PY{l+m+mf}{1.} \PY{o}{/} \PY{p}{(}\PY{l+m+mf}{1.} \PY{o}{+} \PY{n}{np}\PY{o}{.}\PY{n}{exp}\PY{p}{(}\PY{o}{\PYZhy{}}\PY{n}{np}\PY{o}{.}\PY{n}{clip}\PY{p}{(}\PY{n}{z}\PY{p}{,} \PY{o}{\PYZhy{}}\PY{l+m+mi}{250}\PY{p}{,} \PY{l+m+mi}{250}\PY{p}{)}\PY{p}{)}\PY{p}{)}
             
             \PY{k}{def} \PY{n+nf}{\PYZus{}forward}\PY{p}{(}\PY{n+nb+bp}{self}\PY{p}{,} \PY{n}{X}\PY{p}{)}\PY{p}{:}
                 \PY{l+s+sd}{\PYZdq{}\PYZdq{}\PYZdq{}Compute forward propagation step\PYZdq{}\PYZdq{}\PYZdq{}}
                 
                 \PY{c+c1}{\PYZsh{} step 1: net input of hidden layer}
                 \PY{c+c1}{\PYZsh{} [n\PYZus{}samples, n\PYZus{}features] dot [n\PYZus{}features, n\PYZus{}hidden]}
                 \PY{c+c1}{\PYZsh{} \PYZhy{}\PYZgt{} [n\PYZus{}samples, n\PYZus{}hidden]}
                 \PY{n}{z\PYZus{}h} \PY{o}{=} \PY{n}{np}\PY{o}{.}\PY{n}{dot}\PY{p}{(}\PY{n}{X}\PY{p}{,} \PY{n+nb+bp}{self}\PY{o}{.}\PY{n}{w\PYZus{}h}\PY{p}{)} \PY{o}{+} \PY{n+nb+bp}{self}\PY{o}{.}\PY{n}{b\PYZus{}h}
                 
                 \PY{c+c1}{\PYZsh{} step 2: activation of hidden layer}
                 \PY{n}{a\PYZus{}h} \PY{o}{=} \PY{n+nb+bp}{self}\PY{o}{.}\PY{n}{\PYZus{}sigmoid}\PY{p}{(}\PY{n}{z\PYZus{}h}\PY{p}{)}
                 
                 \PY{c+c1}{\PYZsh{} step 3: net input of output layer}
                 \PY{c+c1}{\PYZsh{} [n\PYZus{}samples, n\PYZus{}hidden] dot [n\PYZus{}hidden, n\PYZus{}classlabels]}
                 \PY{c+c1}{\PYZsh{} \PYZhy{}\PYZgt{} [n\PYZus{}samples, n\PYZus{}classlabels]}
                 \PY{n}{z\PYZus{}out} \PY{o}{=} \PY{n}{np}\PY{o}{.}\PY{n}{dot}\PY{p}{(}\PY{n}{a\PYZus{}h}\PY{p}{,} \PY{n+nb+bp}{self}\PY{o}{.}\PY{n}{w\PYZus{}out}\PY{p}{)} \PY{o}{+} \PY{n+nb+bp}{self}\PY{o}{.}\PY{n}{b\PYZus{}out}
                 
                 \PY{c+c1}{\PYZsh{} step 4: activation output layer}
                 \PY{n}{a\PYZus{}out} \PY{o}{=} \PY{n+nb+bp}{self}\PY{o}{.}\PY{n}{\PYZus{}sigmoid}\PY{p}{(}\PY{n}{z\PYZus{}out}\PY{p}{)}
                 
                 \PY{k}{return} \PY{n}{z\PYZus{}h}\PY{p}{,} \PY{n}{a\PYZus{}h}\PY{p}{,} \PY{n}{z\PYZus{}out}\PY{p}{,} \PY{n}{a\PYZus{}out}
             
             \PY{k}{def} \PY{n+nf}{\PYZus{}compute\PYZus{}cost}\PY{p}{(}\PY{n+nb+bp}{self}\PY{p}{,} \PY{n}{y\PYZus{}enc}\PY{p}{,} \PY{n}{output}\PY{p}{)}\PY{p}{:}
                 \PY{l+s+sd}{\PYZdq{}\PYZdq{}\PYZdq{}Compute cost function}
         \PY{l+s+sd}{        }
         \PY{l+s+sd}{        Parameters}
         \PY{l+s+sd}{        \PYZhy{}\PYZhy{}\PYZhy{}\PYZhy{}\PYZhy{}\PYZhy{}\PYZhy{}\PYZhy{}\PYZhy{}\PYZhy{}\PYZhy{}\PYZhy{}\PYZhy{}\PYZhy{}\PYZhy{}}
         \PY{l+s+sd}{        y\PYZus{}enc : array, shape = (n\PYZus{}samples, n\PYZus{}labels)}
         \PY{l+s+sd}{            one\PYZhy{}hot encoded class labels}
         \PY{l+s+sd}{        output : array, shape = [n\PYZus{}samples, n\PYZus{}output\PYZus{}units]}
         \PY{l+s+sd}{            Activation of the output layer (forward propagation)}
         \PY{l+s+sd}{        }
         \PY{l+s+sd}{        Returns }
         \PY{l+s+sd}{        \PYZhy{}\PYZhy{}\PYZhy{}\PYZhy{}\PYZhy{}\PYZhy{}\PYZhy{}\PYZhy{}\PYZhy{}\PYZhy{}\PYZhy{}\PYZhy{}\PYZhy{}\PYZhy{}\PYZhy{}}
         \PY{l+s+sd}{        cost : float}
         \PY{l+s+sd}{            Regularized cost}
         \PY{l+s+sd}{        \PYZdq{}\PYZdq{}\PYZdq{}}
                 \PY{n}{L2\PYZus{}term} \PY{o}{=} \PY{p}{(}\PY{n+nb+bp}{self}\PY{o}{.}\PY{n}{l2} \PY{o}{*} \PY{p}{(}\PY{n}{np}\PY{o}{.}\PY{n}{sum}\PY{p}{(}\PY{n+nb+bp}{self}\PY{o}{.}\PY{n}{w\PYZus{}h} \PY{o}{*}\PY{o}{*} \PY{l+m+mf}{2.}\PY{p}{)} \PY{o}{+} \PY{n}{np}\PY{o}{.}\PY{n}{sum}\PY{p}{(}\PY{n+nb+bp}{self}\PY{o}{.}\PY{n}{w\PYZus{}out} \PY{o}{*}\PY{o}{*} \PY{l+m+mf}{2.}\PY{p}{)}\PY{p}{)}\PY{p}{)}
                 \PY{n}{term1} \PY{o}{=} \PY{o}{\PYZhy{}}\PY{n}{y\PYZus{}enc} \PY{o}{*} \PY{p}{(}\PY{n}{np}\PY{o}{.}\PY{n}{log}\PY{p}{(}\PY{n}{output}\PY{p}{)}\PY{p}{)}
                 \PY{n}{term2} \PY{o}{=} \PY{p}{(}\PY{l+m+mf}{1.} \PY{o}{\PYZhy{}} \PY{n}{y\PYZus{}enc}\PY{p}{)} \PY{o}{*} \PY{n}{np}\PY{o}{.}\PY{n}{log}\PY{p}{(}\PY{l+m+mf}{1.} \PY{o}{\PYZhy{}} \PY{n}{output}\PY{p}{)}
                 \PY{n}{cost} \PY{o}{=} \PY{n}{np}\PY{o}{.}\PY{n}{sum}\PY{p}{(}\PY{n}{term1} \PY{o}{\PYZhy{}} \PY{n}{term2}\PY{p}{)} \PY{o}{+} \PY{n}{L2\PYZus{}term}
                 \PY{k}{return} \PY{n}{cost}
             
             \PY{k}{def} \PY{n+nf}{predict}\PY{p}{(}\PY{n+nb+bp}{self}\PY{p}{,} \PY{n}{X}\PY{p}{)}\PY{p}{:}
                 \PY{l+s+sd}{\PYZdq{}\PYZdq{}\PYZdq{}Predict class labels}
         \PY{l+s+sd}{        }
         \PY{l+s+sd}{        Parameters}
         \PY{l+s+sd}{        \PYZhy{}\PYZhy{}\PYZhy{}\PYZhy{}\PYZhy{}\PYZhy{}\PYZhy{}\PYZhy{}\PYZhy{}\PYZhy{}\PYZhy{}\PYZhy{}\PYZhy{}\PYZhy{}\PYZhy{}}
         \PY{l+s+sd}{        X : array, shape = [n\PYZus{}samples, n\PYZus{}features]}
         \PY{l+s+sd}{            Input layer with original features}
         \PY{l+s+sd}{        }
         \PY{l+s+sd}{        Returns }
         \PY{l+s+sd}{        \PYZhy{}\PYZhy{}\PYZhy{}\PYZhy{}\PYZhy{}\PYZhy{}\PYZhy{}\PYZhy{}\PYZhy{}\PYZhy{}\PYZhy{}\PYZhy{}\PYZhy{}\PYZhy{}\PYZhy{}}
         \PY{l+s+sd}{        y\PYZus{}pred : array, shape = [n\PYZus{}samples]}
         \PY{l+s+sd}{            Predicted class labels}
         \PY{l+s+sd}{        \PYZdq{}\PYZdq{}\PYZdq{}}
                 \PY{n}{z\PYZus{}h}\PY{p}{,} \PY{n}{a\PYZus{}h}\PY{p}{,} \PY{n}{z\PYZus{}out}\PY{p}{,} \PY{n}{a\PYZus{}out} \PY{o}{=} \PY{n+nb+bp}{self}\PY{o}{.}\PY{n}{\PYZus{}forward}\PY{p}{(}\PY{n}{X}\PY{p}{)}
                 \PY{n}{y\PYZus{}pred} \PY{o}{=} \PY{n}{np}\PY{o}{.}\PY{n}{argmax}\PY{p}{(}\PY{n}{z\PYZus{}out}\PY{p}{,} \PY{n}{axis}\PY{o}{=}\PY{l+m+mi}{1}\PY{p}{)}
                 \PY{k}{return} \PY{n}{y\PYZus{}pred}
             
             \PY{k}{def} \PY{n+nf}{fit}\PY{p}{(}\PY{n+nb+bp}{self}\PY{p}{,} \PY{n}{X\PYZus{}train}\PY{p}{,} \PY{n}{y\PYZus{}train}\PY{p}{,} \PY{n}{X\PYZus{}valid}\PY{p}{,} \PY{n}{y\PYZus{}valid}\PY{p}{)}\PY{p}{:}
                 \PY{l+s+sd}{\PYZdq{}\PYZdq{}\PYZdq{}Learn weights from training data}
         \PY{l+s+sd}{        }
         \PY{l+s+sd}{        Parameters}
         \PY{l+s+sd}{        \PYZhy{}\PYZhy{}\PYZhy{}\PYZhy{}\PYZhy{}\PYZhy{}\PYZhy{}\PYZhy{}\PYZhy{}\PYZhy{}\PYZhy{}\PYZhy{}\PYZhy{}\PYZhy{}\PYZhy{}}
         \PY{l+s+sd}{        X\PYZus{}train : array, shape = [n\PYZus{}samples, n\PYZus{}features]}
         \PY{l+s+sd}{            Input layer with original features}
         \PY{l+s+sd}{        y\PYZus{}train : array, shape = [n\PYZus{}samples]}
         \PY{l+s+sd}{            Target class labels}
         \PY{l+s+sd}{        X\PYZus{}valid : array, shape = [n\PYZus{}samples, n\PYZus{}features]}
         \PY{l+s+sd}{            Sample features for validation during training}
         \PY{l+s+sd}{        y\PYZus{}valid : array, shape = [n\PYZus{}samples]}
         \PY{l+s+sd}{            Simple labels for validation during training}
         \PY{l+s+sd}{            }
         \PY{l+s+sd}{        Returns}
         \PY{l+s+sd}{        \PYZhy{}\PYZhy{}\PYZhy{}\PYZhy{}\PYZhy{}\PYZhy{}\PYZhy{}\PYZhy{}\PYZhy{}\PYZhy{}\PYZhy{}\PYZhy{}\PYZhy{}\PYZhy{}\PYZhy{}}
         \PY{l+s+sd}{        self}
         \PY{l+s+sd}{        \PYZdq{}\PYZdq{}\PYZdq{}}
                 \PY{c+c1}{\PYZsh{} no. of class labels}
                 \PY{n}{n\PYZus{}output} \PY{o}{=} \PY{n}{np}\PY{o}{.}\PY{n}{unique}\PY{p}{(}\PY{n}{y\PYZus{}train}\PY{p}{)}\PY{o}{.}\PY{n}{shape}\PY{p}{[}\PY{l+m+mi}{0}\PY{p}{]}
                 \PY{n}{n\PYZus{}features} \PY{o}{=} \PY{n}{X\PYZus{}train}\PY{o}{.}\PY{n}{shape}\PY{p}{[}\PY{l+m+mi}{1}\PY{p}{]}
                 
                 \PY{c+c1}{\PYZsh{} Weight initialization}
                 
                 \PY{c+c1}{\PYZsh{} weights for input \PYZhy{}\PYZgt{} hidden}
                 \PY{n+nb+bp}{self}\PY{o}{.}\PY{n}{b\PYZus{}h} \PY{o}{=} \PY{n}{np}\PY{o}{.}\PY{n}{zeros}\PY{p}{(}\PY{n+nb+bp}{self}\PY{o}{.}\PY{n}{n\PYZus{}hidden}\PY{p}{)}
                 \PY{n+nb+bp}{self}\PY{o}{.}\PY{n}{w\PYZus{}h} \PY{o}{=} \PY{n}{np}\PY{o}{.}\PY{n}{random}\PY{o}{.}\PY{n}{normal}\PY{p}{(}\PY{n}{loc}\PY{o}{=}\PY{l+m+mf}{0.0}\PY{p}{,} \PY{n}{scale}\PY{o}{=}\PY{l+m+mf}{0.1}\PY{p}{,} 
                                             \PY{n}{size}\PY{o}{=}\PY{p}{(}\PY{n}{n\PYZus{}features}\PY{p}{,} 
                                                   \PY{n+nb+bp}{self}\PY{o}{.}\PY{n}{n\PYZus{}hidden}\PY{p}{)}\PY{p}{)}
                 
                 \PY{c+c1}{\PYZsh{} weights for hidden \PYZhy{}\PYZgt{} output}
                 \PY{n+nb+bp}{self}\PY{o}{.}\PY{n}{b\PYZus{}out} \PY{o}{=} \PY{n}{np}\PY{o}{.}\PY{n}{zeros}\PY{p}{(}\PY{n}{n\PYZus{}output}\PY{p}{)}
                 \PY{n+nb+bp}{self}\PY{o}{.}\PY{n}{w\PYZus{}out} \PY{o}{=} \PY{n+nb+bp}{self}\PY{o}{.}\PY{n}{random}\PY{o}{.}\PY{n}{normal}\PY{p}{(}\PY{n}{loc}\PY{o}{=}\PY{l+m+mf}{0.0}\PY{p}{,} \PY{n}{scale}\PY{o}{=}\PY{l+m+mf}{0.1}\PY{p}{,} 
                                                 \PY{n}{size}\PY{o}{=}\PY{p}{(}\PY{n+nb+bp}{self}\PY{o}{.}\PY{n}{n\PYZus{}hidden}\PY{p}{,} 
                                                       \PY{n}{n\PYZus{}output}\PY{p}{)}\PY{p}{)}
         
                 \PY{c+c1}{\PYZsh{} for progr. format.}
                 \PY{n}{epoch\PYZus{}strlen} \PY{o}{=} \PY{n+nb}{len}\PY{p}{(}\PY{n+nb}{str}\PY{p}{(}\PY{n+nb+bp}{self}\PY{o}{.}\PY{n}{epochs}\PY{p}{)}\PY{p}{)}
                 \PY{n+nb+bp}{self}\PY{o}{.}\PY{n}{eval\PYZus{}} \PY{o}{=} \PY{p}{\PYZob{}}\PY{l+s+s1}{\PYZsq{}}\PY{l+s+s1}{cost}\PY{l+s+s1}{\PYZsq{}}\PY{p}{:} \PY{p}{[}\PY{p}{]}\PY{p}{,} 
                               \PY{l+s+s1}{\PYZsq{}}\PY{l+s+s1}{train\PYZus{}acc}\PY{l+s+s1}{\PYZsq{}}\PY{p}{:} \PY{p}{[}\PY{p}{]}\PY{p}{,} 
                               \PY{l+s+s1}{\PYZsq{}}\PY{l+s+s1}{valid\PYZus{}acc}\PY{l+s+s1}{\PYZsq{}}\PY{p}{:} \PY{p}{[}\PY{p}{]}\PY{p}{\PYZcb{}}
                 
                 \PY{n}{y\PYZus{}train\PYZus{}enc} \PY{o}{=} \PY{n+nb+bp}{self}\PY{o}{.}\PY{n}{\PYZus{}onehot}\PY{p}{(}\PY{n}{y\PYZus{}train}\PY{p}{,} \PY{n}{n\PYZus{}output}\PY{p}{)}
                 
                 \PY{c+c1}{\PYZsh{} iterate over training epochs}
                 \PY{k}{for} \PY{n}{i} \PY{o+ow}{in} \PY{n+nb}{range}\PY{p}{(}\PY{n+nb+bp}{self}\PY{o}{.}\PY{n}{epochs}\PY{p}{)}\PY{p}{:}
                     \PY{c+c1}{\PYZsh{} iterate over minibatches}
                     \PY{n}{indices} \PY{o}{=} \PY{n}{np}\PY{o}{.}\PY{n}{arange}\PY{p}{(}\PY{n}{X\PYZus{}train}\PY{o}{.}\PY{n}{shape}\PY{p}{[}\PY{l+m+mi}{0}\PY{p}{]}\PY{p}{)}
                     
                     \PY{k}{if} \PY{n+nb+bp}{self}\PY{o}{.}\PY{n}{shuffle}\PY{p}{:}
                         \PY{n+nb+bp}{self}\PY{o}{.}\PY{n}{random}\PY{o}{.}\PY{n}{shuffle}\PY{p}{(}\PY{n}{indices}\PY{p}{)}
                         
                     \PY{k}{for} \PY{n}{start\PYZus{}idx} \PY{o+ow}{in} \PY{n+nb}{range}\PY{p}{(}\PY{l+m+mi}{0}\PY{p}{,} \PY{n}{indices}\PY{o}{.}\PY{n}{shape}\PY{p}{[}\PY{l+m+mi}{0}\PY{p}{]} \PY{o}{\PYZhy{}} \PY{n+nb+bp}{self}\PY{o}{.}\PY{n}{minibatch\PYZus{}size} \PY{o}{+} \PY{l+m+mi}{1}\PY{p}{,} \PY{n+nb+bp}{self}\PY{o}{.}\PY{n}{minibatch\PYZus{}size}\PY{p}{)}\PY{p}{:}
                         \PY{n}{batch\PYZus{}idx} \PY{o}{=} \PY{n}{indices}\PY{p}{[}\PY{n}{start\PYZus{}idx}\PY{p}{:}\PY{n}{start\PYZus{}idx} \PY{o}{+} \PY{n+nb+bp}{self}\PY{o}{.}\PY{n}{minibatch\PYZus{}size}\PY{p}{]}
                         
                         \PY{c+c1}{\PYZsh{} forward propagation}
                         \PY{n}{z\PYZus{}h}\PY{p}{,} \PY{n}{a\PYZus{}h}\PY{p}{,} \PY{n}{z\PYZus{}out}\PY{p}{,} \PY{n}{a\PYZus{}out} \PY{o}{=} \PY{n+nb+bp}{self}\PY{o}{.}\PY{n}{\PYZus{}forward}\PY{p}{(}\PY{n}{X\PYZus{}train}\PY{p}{[}\PY{n}{batch\PYZus{}idx}\PY{p}{]}\PY{p}{)}
                         
                         \PY{c+c1}{\PYZsh{} Backpropagation}
                         
                         \PY{c+c1}{\PYZsh{} [n\PYZus{}samples, n\PYZus{}classlabels]}
                         \PY{n}{sigma\PYZus{}out} \PY{o}{=} \PY{n}{a\PYZus{}out} \PY{o}{\PYZhy{}} \PY{n}{y\PYZus{}train\PYZus{}enc}\PY{p}{[}\PY{n}{batch\PYZus{}idx}\PY{p}{]}
                         
                         \PY{c+c1}{\PYZsh{} [n\PYZus{}samples, n\PYZus{}hidden]}
                         \PY{n}{sigmoid\PYZus{}derivative\PYZus{}h} \PY{o}{=} \PY{n}{a\PYZus{}h} \PY{o}{*} \PY{p}{(}\PY{l+m+mf}{1.} \PY{o}{\PYZhy{}} \PY{n}{a\PYZus{}h}\PY{p}{)}
                         
                         \PY{c+c1}{\PYZsh{} [n\PYZus{}samples, n\PYZus{}classlabels] dot [n\PYZus{}classlabels, n\PYZus{}hidden]}
                         \PY{c+c1}{\PYZsh{} \PYZhy{}\PYZgt{} [n\PYZus{}samples, n\PYZus{}hidden]}
                         \PY{n}{sigma\PYZus{}h} \PY{o}{=} \PY{p}{(}\PY{n}{np}\PY{o}{.}\PY{n}{dot}\PY{p}{(}\PY{n}{sigma\PYZus{}out}\PY{p}{,} \PY{n+nb+bp}{self}\PY{o}{.}\PY{n}{w\PYZus{}out}\PY{o}{.}\PY{n}{T}\PY{p}{)} \PY{o}{*} \PY{n}{sigmoid\PYZus{}derivative\PYZus{}h}\PY{p}{)}
                         
                         \PY{c+c1}{\PYZsh{} [n\PYZus{}features, n\PYZus{}samples] dot [n\PYZus{}samples, n\PYZus{}hidden]}
                         \PY{c+c1}{\PYZsh{} \PYZhy{}\PYZgt{} [n\PYZus{}features, n\PYZus{}hidden]}
                         \PY{n}{grad\PYZus{}w\PYZus{}h} \PY{o}{=} \PY{n}{np}\PY{o}{.}\PY{n}{dot}\PY{p}{(}\PY{n}{X\PYZus{}train}\PY{p}{[}\PY{n}{batch\PYZus{}idx}\PY{p}{]}\PY{o}{.}\PY{n}{T}\PY{p}{,} \PY{n}{sigma\PYZus{}h}\PY{p}{)}
                         \PY{n}{grad\PYZus{}b\PYZus{}h} \PY{o}{=} \PY{n}{np}\PY{o}{.}\PY{n}{sum}\PY{p}{(}\PY{n}{sigma\PYZus{}h}\PY{p}{,} \PY{n}{axis}\PY{o}{=}\PY{l+m+mi}{0}\PY{p}{)}
                         
                         \PY{c+c1}{\PYZsh{} [n\PYZus{}hidden, n\PYZus{}samples] dot [n\PYZus{}samples, n\PYZus{}classlabels]}
                         \PY{c+c1}{\PYZsh{} \PYZhy{}\PYZgt{} [n\PYZus{}hidden, n\PYZus{}classlabels]}
                         \PY{n}{grad\PYZus{}w\PYZus{}out} \PY{o}{=} \PY{n}{np}\PY{o}{.}\PY{n}{dot}\PY{p}{(}\PY{n}{a\PYZus{}h}\PY{o}{.}\PY{n}{T}\PY{p}{,} \PY{n}{sigma\PYZus{}out}\PY{p}{)}
                         \PY{n}{grad\PYZus{}b\PYZus{}out} \PY{o}{=} \PY{n}{np}\PY{o}{.}\PY{n}{sum}\PY{p}{(}\PY{n}{sigma\PYZus{}out}\PY{p}{,} \PY{n}{axis}\PY{o}{=}\PY{l+m+mi}{0}\PY{p}{)}
                         
                         \PY{c+c1}{\PYZsh{} Regularization and weight updates}
                         \PY{n}{delta\PYZus{}w\PYZus{}h} \PY{o}{=} \PY{p}{(}\PY{n}{grad\PYZus{}w\PYZus{}h} \PY{o}{+} \PY{n+nb+bp}{self}\PY{o}{.}\PY{n}{l2}\PY{o}{*}\PY{n+nb+bp}{self}\PY{o}{.}\PY{n}{w\PYZus{}h}\PY{p}{)}
                         \PY{n}{delta\PYZus{}b\PYZus{}h} \PY{o}{=} \PY{n}{grad\PYZus{}b\PYZus{}h} \PY{c+c1}{\PYZsh{} bias is not regularized}
                         \PY{n+nb+bp}{self}\PY{o}{.}\PY{n}{w\PYZus{}h} \PY{o}{\PYZhy{}}\PY{o}{=} \PY{n+nb+bp}{self}\PY{o}{.}\PY{n}{eta} \PY{o}{*} \PY{n}{delta\PYZus{}w\PYZus{}h}
                         \PY{n+nb+bp}{self}\PY{o}{.}\PY{n}{b\PYZus{}h} \PY{o}{\PYZhy{}}\PY{o}{=} \PY{n+nb+bp}{self}\PY{o}{.}\PY{n}{eta} \PY{o}{*} \PY{n}{delta\PYZus{}b\PYZus{}h}
                         
                         \PY{n}{delta\PYZus{}w\PYZus{}out} \PY{o}{=} \PY{p}{(}\PY{n}{grad\PYZus{}w\PYZus{}out} \PY{o}{+} \PY{n+nb+bp}{self}\PY{o}{.}\PY{n}{l2}\PY{o}{*}\PY{n+nb+bp}{self}\PY{o}{.}\PY{n}{w\PYZus{}out}\PY{p}{)}
                         \PY{n}{delta\PYZus{}b\PYZus{}out} \PY{o}{=} \PY{n}{grad\PYZus{}b\PYZus{}out} \PY{c+c1}{\PYZsh{} bias is not regularized}
                         \PY{n+nb+bp}{self}\PY{o}{.}\PY{n}{w\PYZus{}out} \PY{o}{\PYZhy{}}\PY{o}{=} \PY{n+nb+bp}{self}\PY{o}{.}\PY{n}{eta} \PY{o}{*} \PY{n}{delta\PYZus{}w\PYZus{}out}
                         \PY{n+nb+bp}{self}\PY{o}{.}\PY{n}{b\PYZus{}out} \PY{o}{\PYZhy{}}\PY{o}{=} \PY{n+nb+bp}{self}\PY{o}{.}\PY{n}{eta} \PY{o}{*} \PY{n}{delta\PYZus{}b\PYZus{}out}
                     
                     \PY{c+c1}{\PYZsh{} Evaluation}
                     
                     \PY{c+c1}{\PYZsh{} Evaluation after each epoch during training}
                     \PY{n}{z\PYZus{}h}\PY{p}{,} \PY{n}{a\PYZus{}h}\PY{p}{,} \PY{n}{z\PYZus{}out}\PY{p}{,} \PY{n}{a\PYZus{}out} \PY{o}{=} \PY{n+nb+bp}{self}\PY{o}{.}\PY{n}{\PYZus{}forward}\PY{p}{(}\PY{n}{X\PYZus{}train}\PY{p}{)}
                     
                     \PY{n}{cost} \PY{o}{=} \PY{n+nb+bp}{self}\PY{o}{.}\PY{n}{\PYZus{}compute\PYZus{}cost}\PY{p}{(}\PY{n}{y\PYZus{}enc}\PY{o}{=}\PY{n}{y\PYZus{}train\PYZus{}enc}\PY{p}{,} 
                                               \PY{n}{output}\PY{o}{=}\PY{n}{a\PYZus{}out}\PY{p}{)}
                     
                     \PY{n}{y\PYZus{}train\PYZus{}pred} \PY{o}{=} \PY{n+nb+bp}{self}\PY{o}{.}\PY{n}{predict}\PY{p}{(}\PY{n}{X\PYZus{}train}\PY{p}{)}
                     \PY{n}{y\PYZus{}valid\PYZus{}pred} \PY{o}{=} \PY{n+nb+bp}{self}\PY{o}{.}\PY{n}{predict}\PY{p}{(}\PY{n}{X\PYZus{}valid}\PY{p}{)}
                     
                     \PY{n}{train\PYZus{}acc} \PY{o}{=} \PY{p}{(}\PY{p}{(}\PY{n}{np}\PY{o}{.}\PY{n}{sum}\PY{p}{(}\PY{n}{y\PYZus{}train} \PY{o}{==} \PY{n}{y\PYZus{}train\PYZus{}pred}\PY{p}{)}\PY{p}{)}\PY{o}{.}\PY{n}{astype}\PY{p}{(}\PY{n}{np}\PY{o}{.}\PY{n}{float}\PY{p}{)} \PY{o}{/} \PY{n}{X\PYZus{}train}\PY{o}{.}\PY{n}{shape}\PY{p}{[}\PY{l+m+mi}{0}\PY{p}{]}\PY{p}{)}
                     \PY{n}{valid\PYZus{}acc} \PY{o}{=} \PY{p}{(}\PY{p}{(}\PY{n}{np}\PY{o}{.}\PY{n}{sum}\PY{p}{(}\PY{n}{y\PYZus{}valid} \PY{o}{==} \PY{n}{y\PYZus{}valid\PYZus{}pred}\PY{p}{)}\PY{p}{)}\PY{o}{.}\PY{n}{astype}\PY{p}{(}\PY{n}{np}\PY{o}{.}\PY{n}{float}\PY{p}{)} \PY{o}{/} \PY{n}{X\PYZus{}valid}\PY{o}{.}\PY{n}{shape}\PY{p}{[}\PY{l+m+mi}{0}\PY{p}{]}\PY{p}{)}
                     
                     \PY{n}{sys}\PY{o}{.}\PY{n}{stderr}\PY{o}{.}\PY{n}{write}\PY{p}{(}\PY{l+s+s1}{\PYZsq{}}\PY{l+s+se}{\PYZbs{}r}\PY{l+s+si}{\PYZpc{}0*d}\PY{l+s+s1}{/}\PY{l+s+si}{\PYZpc{}d}\PY{l+s+s1}{ | Cost: }\PY{l+s+si}{\PYZpc{}.2f}\PY{l+s+s1}{ }\PY{l+s+s1}{\PYZsq{}}
                                      \PY{l+s+s1}{\PYZsq{}}\PY{l+s+s1}{ | Train/Valid Acc.: }\PY{l+s+si}{\PYZpc{}.2f}\PY{l+s+si}{\PYZpc{}\PYZpc{}}\PY{l+s+s1}{/}\PY{l+s+si}{\PYZpc{}.2f}\PY{l+s+si}{\PYZpc{}\PYZpc{}}\PY{l+s+s1}{ }\PY{l+s+s1}{\PYZsq{}}
                                      \PY{o}{\PYZpc{}} \PY{p}{(}\PY{n}{epoch\PYZus{}strlen}\PY{p}{,} \PY{n}{i}\PY{o}{+}\PY{l+m+mi}{1}\PY{p}{,} \PY{n+nb+bp}{self}\PY{o}{.}\PY{n}{epochs}\PY{p}{,} 
                                         \PY{n}{cost}\PY{p}{,} \PY{n}{train\PYZus{}acc}\PY{o}{*}\PY{l+m+mi}{100}\PY{p}{,} \PY{n}{valid\PYZus{}acc}\PY{o}{*}\PY{l+m+mi}{100}\PY{p}{)}\PY{p}{)}
                     \PY{n}{sys}\PY{o}{.}\PY{n}{stderr}\PY{o}{.}\PY{n}{flush}\PY{p}{(}\PY{p}{)}
                     
                     \PY{n+nb+bp}{self}\PY{o}{.}\PY{n}{eval\PYZus{}}\PY{p}{[}\PY{l+s+s1}{\PYZsq{}}\PY{l+s+s1}{cost}\PY{l+s+s1}{\PYZsq{}}\PY{p}{]}\PY{o}{.}\PY{n}{append}\PY{p}{(}\PY{n}{cost}\PY{p}{)}
                     \PY{n+nb+bp}{self}\PY{o}{.}\PY{n}{eval\PYZus{}}\PY{p}{[}\PY{l+s+s1}{\PYZsq{}}\PY{l+s+s1}{train\PYZus{}acc}\PY{l+s+s1}{\PYZsq{}}\PY{p}{]}\PY{o}{.}\PY{n}{append}\PY{p}{(}\PY{n}{train\PYZus{}acc}\PY{p}{)}
                     \PY{n+nb+bp}{self}\PY{o}{.}\PY{n}{eval\PYZus{}}\PY{p}{[}\PY{l+s+s1}{\PYZsq{}}\PY{l+s+s1}{valid\PYZus{}acc}\PY{l+s+s1}{\PYZsq{}}\PY{p}{]}\PY{o}{.}\PY{n}{append}\PY{p}{(}\PY{n}{valid\PYZus{}acc}\PY{p}{)}
                 
                 \PY{k}{return} \PY{n+nb+bp}{self}
\end{Verbatim}


    Once you are done with executing this code, let's now initialize a new
784-100-10 MLP - a neural network with 784 input units
(\emph{n\_features}), 100 hidden units (\emph{n\_hidden}), and 10 output
units (\emph{n\_output}):

    \begin{Verbatim}[commandchars=\\\{\}]
{\color{incolor}In [{\color{incolor}16}]:} \PY{n}{nn} \PY{o}{=} \PY{n}{NeuralNetMLP}\PY{p}{(}\PY{n}{n\PYZus{}hidden}\PY{o}{=}\PY{l+m+mi}{100}\PY{p}{,} 
                           \PY{n}{l2}\PY{o}{=}\PY{l+m+mf}{0.01}\PY{p}{,} 
                           \PY{n}{epochs}\PY{o}{=}\PY{l+m+mi}{200}\PY{p}{,} 
                           \PY{n}{eta}\PY{o}{=}\PY{l+m+mf}{0.0005}\PY{p}{,} 
                           \PY{n}{minibatch\PYZus{}size}\PY{o}{=}\PY{l+m+mi}{100}\PY{p}{,} 
                           \PY{n}{shuffle}\PY{o}{=}\PY{k+kc}{True}\PY{p}{,} 
                           \PY{n}{seed}\PY{o}{=}\PY{l+m+mi}{1}\PY{p}{)}
\end{Verbatim}


    If you read through the \emph{NeuralNetMLP} code, you have probably
already guessed what these parameters are for. Here, you find a short
summary of these:

\begin{itemize}
\tightlist
\item
  \emph{l2}: This is the \(\lambda\) parameter for L2 regularization to
  decrease the degree of overfitting.
\item
  \emph{epochs}: This is the number of passes over the training set.
\item
  \emph{eta}: This is the learning rate \(\eta\).
\item
  \emph{shuffle}: This is for shuffling the training set prior to every
  epoch to prevent that the algorithm gets stuck in circles.
\item
  \emph{seed}: This is a random seed for shuffling and weigh
  initialization.
\item
  \emph{minibatch\_size}: This is the number of training samples in each
  mini-batch when splitting of the training data in each epoch for
  stochastic gradient descent. The gradient descent is computed for each
  mini-batch separately instead of the entire training data for faster
  learning.
\end{itemize}

Next, we train the MLP using 55,000 samples from the already shuffled
MNIST training dataset and used the remaining 5,000 samples for
validation during training. Note that training the neural network may
take up to 5 minutes on standard desktop computer hardware.

As you may have noticed from the preceding code implementation, we
implemented the \emph{fit} method so that it takes four input arguments:
training images, training labels, validation images, and validation
labels. In neural network training, it is really useful to already
compare training and validation accuracy during training, which helps us
judge whether the network model performs well, given the architecture
and hyperparameters.

In general, training (deep) neural networks is relatively expensive
compared with the other models we discussed so far. Thus, we want to
stop it early in certain circumstances and start over with different
hyperparameter setting. Alternatively, if we find that it increasingly
tends to overfit the training data (noticeable by an increasing gap
between training and validation set performance), we may want to stop
the training early as well.

Now, to start the training, we execute the following code:

    \begin{Verbatim}[commandchars=\\\{\}]
{\color{incolor}In [{\color{incolor}17}]:} \PY{n}{nn}\PY{o}{.}\PY{n}{fit}\PY{p}{(}\PY{n}{X\PYZus{}train}\PY{o}{=}\PY{n}{X\PYZus{}train}\PY{p}{[}\PY{p}{:}\PY{l+m+mi}{55000}\PY{p}{]}\PY{p}{,} \PY{n}{y\PYZus{}train}\PY{o}{=}\PY{n}{y\PYZus{}train}\PY{p}{[}\PY{p}{:}\PY{l+m+mi}{55000}\PY{p}{]}\PY{p}{,} 
                \PY{n}{X\PYZus{}valid}\PY{o}{=}\PY{n}{X\PYZus{}train}\PY{p}{[}\PY{l+m+mi}{55000}\PY{p}{:}\PY{p}{]}\PY{p}{,} \PY{n}{y\PYZus{}valid}\PY{o}{=}\PY{n}{y\PYZus{}train}\PY{p}{[}\PY{l+m+mi}{55000}\PY{p}{:}\PY{p}{]}\PY{p}{)}
\end{Verbatim}


    \begin{Verbatim}[commandchars=\\\{\}]
200/200 | Cost: 5047.08  | Train/Valid Acc.: 99.29\%/97.94\%  
    \end{Verbatim}

\begin{Verbatim}[commandchars=\\\{\}]
{\color{outcolor}Out[{\color{outcolor}17}]:} <\_\_main\_\_.NeuralNetMLP at 0x7f16158d4a20>
\end{Verbatim}
            
    In our \emph{NeuralNetMLP} implementation, we also defined an
\emph{eval\_} attribute that collects the cost, training, and validation
accuracy for each epoch so that we can visualize the results using
Matplotlib:

    \begin{Verbatim}[commandchars=\\\{\}]
{\color{incolor}In [{\color{incolor}18}]:} \PY{k+kn}{import} \PY{n+nn}{matplotlib}\PY{n+nn}{.}\PY{n+nn}{pyplot} \PY{k}{as} \PY{n+nn}{plt}
         
         \PY{n}{plt}\PY{o}{.}\PY{n}{plot}\PY{p}{(}\PY{n+nb}{range}\PY{p}{(}\PY{n}{nn}\PY{o}{.}\PY{n}{epochs}\PY{p}{)}\PY{p}{,} \PY{n}{nn}\PY{o}{.}\PY{n}{eval\PYZus{}}\PY{p}{[}\PY{l+s+s1}{\PYZsq{}}\PY{l+s+s1}{cost}\PY{l+s+s1}{\PYZsq{}}\PY{p}{]}\PY{p}{)}
         \PY{n}{plt}\PY{o}{.}\PY{n}{ylabel}\PY{p}{(}\PY{l+s+s1}{\PYZsq{}}\PY{l+s+s1}{Cost}\PY{l+s+s1}{\PYZsq{}}\PY{p}{)}
         \PY{n}{plt}\PY{o}{.}\PY{n}{xlabel}\PY{p}{(}\PY{l+s+s1}{\PYZsq{}}\PY{l+s+s1}{Epochs}\PY{l+s+s1}{\PYZsq{}}\PY{p}{)}
         \PY{n}{plt}\PY{o}{.}\PY{n}{show}\PY{p}{(}\PY{p}{)}
\end{Verbatim}


    \begin{center}
    \adjustimage{max size={0.9\linewidth}{0.9\paperheight}}{output_40_0.png}
    \end{center}
    { \hspace*{\fill} \\}
    
    As we can see, the cost decreased substantially during the first 100
epochs and seems to slowly converge in the last 100 epochs. However, the
small slope between epoch 175 and epoch 200 indicates that the cost
would further decrease with a training over additional epochs.

Next, let's take a look at the training and validation accuracy:

    \begin{Verbatim}[commandchars=\\\{\}]
{\color{incolor}In [{\color{incolor}19}]:} \PY{n}{plt}\PY{o}{.}\PY{n}{plot}\PY{p}{(}\PY{n+nb}{range}\PY{p}{(}\PY{n}{nn}\PY{o}{.}\PY{n}{epochs}\PY{p}{)}\PY{p}{,} \PY{n}{nn}\PY{o}{.}\PY{n}{eval\PYZus{}}\PY{p}{[}\PY{l+s+s1}{\PYZsq{}}\PY{l+s+s1}{train\PYZus{}acc}\PY{l+s+s1}{\PYZsq{}}\PY{p}{]}\PY{p}{,} 
                  \PY{n}{label}\PY{o}{=}\PY{l+s+s1}{\PYZsq{}}\PY{l+s+s1}{training}\PY{l+s+s1}{\PYZsq{}}\PY{p}{)}
         \PY{n}{plt}\PY{o}{.}\PY{n}{plot}\PY{p}{(}\PY{n+nb}{range}\PY{p}{(}\PY{n}{nn}\PY{o}{.}\PY{n}{epochs}\PY{p}{)}\PY{p}{,} \PY{n}{nn}\PY{o}{.}\PY{n}{eval\PYZus{}}\PY{p}{[}\PY{l+s+s1}{\PYZsq{}}\PY{l+s+s1}{valid\PYZus{}acc}\PY{l+s+s1}{\PYZsq{}}\PY{p}{]}\PY{p}{,} 
                  \PY{n}{label}\PY{o}{=}\PY{l+s+s1}{\PYZsq{}}\PY{l+s+s1}{validation}\PY{l+s+s1}{\PYZsq{}}\PY{p}{,} \PY{n}{linestyle}\PY{o}{=}\PY{l+s+s1}{\PYZsq{}}\PY{l+s+s1}{\PYZhy{}\PYZhy{}}\PY{l+s+s1}{\PYZsq{}}\PY{p}{)}
         \PY{n}{plt}\PY{o}{.}\PY{n}{ylabel}\PY{p}{(}\PY{l+s+s1}{\PYZsq{}}\PY{l+s+s1}{Accuracy}\PY{l+s+s1}{\PYZsq{}}\PY{p}{)}
         \PY{n}{plt}\PY{o}{.}\PY{n}{xlabel}\PY{p}{(}\PY{l+s+s1}{\PYZsq{}}\PY{l+s+s1}{Epochs}\PY{l+s+s1}{\PYZsq{}}\PY{p}{)}
         \PY{n}{plt}\PY{o}{.}\PY{n}{legend}\PY{p}{(}\PY{p}{)}
         \PY{n}{plt}\PY{o}{.}\PY{n}{show}\PY{p}{(}\PY{p}{)}
\end{Verbatim}


    \begin{center}
    \adjustimage{max size={0.9\linewidth}{0.9\paperheight}}{output_42_0.png}
    \end{center}
    { \hspace*{\fill} \\}
    
    The plot reveals that the gap between training and validation accuracy
increases the more epochs we train the network. At approximately the
50th epoch, the training and validation accuracy values are equal, and
then, the network starts overfitting the training data.

Note that this example was chosen deliberately to illustrate the effect
of overfitting and demonstrate why it is useful to compare the
validation and training accuracy values during training. One way to
decrease the effect of overfitting is to increase the regularization
strength - for example, by setting \emph{l2=0.1}. Another useful
technique to tackle overfitting in neural networks, dropout, will be
covered later.

Finally, let's evaluate the generalization performance of the model by
calculating the prediction accuracy on the test set:

    \begin{Verbatim}[commandchars=\\\{\}]
{\color{incolor}In [{\color{incolor}20}]:} \PY{n}{y\PYZus{}test\PYZus{}pred} \PY{o}{=} \PY{n}{nn}\PY{o}{.}\PY{n}{predict}\PY{p}{(}\PY{n}{X\PYZus{}test}\PY{p}{)}
         \PY{n}{acc} \PY{o}{=} \PY{p}{(}\PY{n}{np}\PY{o}{.}\PY{n}{sum}\PY{p}{(}\PY{n}{y\PYZus{}test} \PY{o}{==} \PY{n}{y\PYZus{}test\PYZus{}pred}\PY{p}{)}\PY{o}{.}\PY{n}{astype}\PY{p}{(}\PY{n}{np}\PY{o}{.}\PY{n}{float}\PY{p}{)} \PY{o}{/} \PY{n}{X\PYZus{}test}\PY{o}{.}\PY{n}{shape}\PY{p}{[}\PY{l+m+mi}{0}\PY{p}{]}\PY{p}{)}
         \PY{n+nb}{print}\PY{p}{(}\PY{l+s+s1}{\PYZsq{}}\PY{l+s+s1}{Training accuracy: }\PY{l+s+si}{\PYZpc{}.2f}\PY{l+s+si}{\PYZpc{}\PYZpc{}}\PY{l+s+s1}{\PYZsq{}} \PY{o}{\PYZpc{}} \PY{p}{(}\PY{n}{acc} \PY{o}{*} \PY{l+m+mi}{100}\PY{p}{)}\PY{p}{)}
\end{Verbatim}


    \begin{Verbatim}[commandchars=\\\{\}]
Training accuracy: 97.53\%

    \end{Verbatim}

    Despite the slight overfitting on the training data, our relatively
simple one-hidden layer neural network achieved a relatively good
performance on the test dataset, similar to the validation set accuracy
(97.98 percent).

To further fine-tune the model, we could change the number of hidden
units, values of the regularization parameters, and the learning rate or
use various other tricks that have been developed over the years but are
beyond the scope of this book. In a next chapter, you will learn about a
different neural network architecture that is known for its good
performance on image datasets. Also, the chapter will introduce
additional performance-enhancing tricks such as adaptive learning rates,
momentum learning, and dropout.

Lastly, let's take a look at some of the images that our MLP struggles
with:

    \begin{Verbatim}[commandchars=\\\{\}]
{\color{incolor}In [{\color{incolor}23}]:} \PY{n}{miscl\PYZus{}img} \PY{o}{=} \PY{n}{X\PYZus{}test}\PY{p}{[}\PY{n}{y\PYZus{}test} \PY{o}{!=} \PY{n}{y\PYZus{}test\PYZus{}pred}\PY{p}{]}\PY{p}{[}\PY{p}{:}\PY{l+m+mi}{25}\PY{p}{]}
         \PY{n}{correct\PYZus{}lab} \PY{o}{=} \PY{n}{y\PYZus{}test}\PY{p}{[}\PY{n}{y\PYZus{}test} \PY{o}{!=} \PY{n}{y\PYZus{}test\PYZus{}pred}\PY{p}{]}\PY{p}{[}\PY{p}{:}\PY{l+m+mi}{25}\PY{p}{]}
         \PY{n}{miscl\PYZus{}lab} \PY{o}{=} \PY{n}{y\PYZus{}test\PYZus{}pred}\PY{p}{[}\PY{n}{y\PYZus{}test} \PY{o}{!=} \PY{n}{y\PYZus{}test\PYZus{}pred}\PY{p}{]}\PY{p}{[}\PY{p}{:}\PY{l+m+mi}{25}\PY{p}{]}
         
         \PY{n}{fig}\PY{p}{,} \PY{n}{ax} \PY{o}{=} \PY{n}{plt}\PY{o}{.}\PY{n}{subplots}\PY{p}{(}\PY{n}{nrows}\PY{o}{=}\PY{l+m+mi}{5}\PY{p}{,} \PY{n}{ncols}\PY{o}{=}\PY{l+m+mi}{5}\PY{p}{,} \PY{n}{sharex}\PY{o}{=}\PY{k+kc}{True}\PY{p}{,} \PY{n}{sharey}\PY{o}{=}\PY{k+kc}{True}\PY{p}{)}
         \PY{n}{ax} \PY{o}{=} \PY{n}{ax}\PY{o}{.}\PY{n}{flatten}\PY{p}{(}\PY{p}{)}
         \PY{k}{for} \PY{n}{i} \PY{o+ow}{in} \PY{n+nb}{range}\PY{p}{(}\PY{l+m+mi}{25}\PY{p}{)}\PY{p}{:}
             \PY{n}{img} \PY{o}{=} \PY{n}{miscl\PYZus{}img}\PY{p}{[}\PY{n}{i}\PY{p}{]}\PY{o}{.}\PY{n}{reshape}\PY{p}{(}\PY{l+m+mi}{28}\PY{p}{,} \PY{l+m+mi}{28}\PY{p}{)}
             \PY{n}{ax}\PY{p}{[}\PY{n}{i}\PY{p}{]}\PY{o}{.}\PY{n}{imshow}\PY{p}{(}\PY{n}{img}\PY{p}{,} \PY{n}{cmap}\PY{o}{=}\PY{l+s+s1}{\PYZsq{}}\PY{l+s+s1}{Greys}\PY{l+s+s1}{\PYZsq{}}\PY{p}{,} \PY{n}{interpolation}\PY{o}{=}\PY{l+s+s1}{\PYZsq{}}\PY{l+s+s1}{nearest}\PY{l+s+s1}{\PYZsq{}}\PY{p}{)}
             \PY{n}{ax}\PY{p}{[}\PY{n}{i}\PY{p}{]}\PY{o}{.}\PY{n}{set\PYZus{}title}\PY{p}{(}\PY{l+s+s1}{\PYZsq{}}\PY{l+s+si}{\PYZpc{}d}\PY{l+s+s1}{) t: }\PY{l+s+si}{\PYZpc{}d}\PY{l+s+s1}{ p: }\PY{l+s+si}{\PYZpc{}d}\PY{l+s+s1}{\PYZsq{}}
                             \PY{o}{\PYZpc{}} \PY{p}{(}\PY{n}{i}\PY{o}{+}\PY{l+m+mi}{1}\PY{p}{,} \PY{n}{correct\PYZus{}lab}\PY{p}{[}\PY{n}{i}\PY{p}{]}\PY{p}{,} \PY{n}{miscl\PYZus{}lab}\PY{p}{[}\PY{n}{i}\PY{p}{]}\PY{p}{)}\PY{p}{)}
         \PY{n}{ax}\PY{p}{[}\PY{l+m+mi}{0}\PY{p}{]}\PY{o}{.}\PY{n}{set\PYZus{}xticks}\PY{p}{(}\PY{p}{[}\PY{p}{]}\PY{p}{)}
         \PY{n}{ax}\PY{p}{[}\PY{l+m+mi}{0}\PY{p}{]}\PY{o}{.}\PY{n}{set\PYZus{}yticks}\PY{p}{(}\PY{p}{[}\PY{p}{]}\PY{p}{)}
         \PY{n}{plt}\PY{o}{.}\PY{n}{tight\PYZus{}layout}\PY{p}{(}\PY{p}{)}
         \PY{n}{plt}\PY{o}{.}\PY{n}{show}\PY{p}{(}\PY{p}{)}
\end{Verbatim}


    \begin{center}
    \adjustimage{max size={0.9\linewidth}{0.9\paperheight}}{output_46_0.png}
    \end{center}
    { \hspace*{\fill} \\}
    
    We should now see a 5 x 5 subplot matrix where the first number of
subtitles indicates the plot index, the second number represents the
true class label (\emph{t}), and the third number stands for the
predicted class label (\emph{p}).

As we can see in the preceding figure, some of those images are even
challenging for us humans to classify correctly. For example, the 6 in
subplot 8 really looks like a carelessly drawn 0, and the 8 in subplot
23 could be a 9 due to the narrow lower part combined with the bold
line.

    \subsection{Training an artificial neural
network}\label{training-an-artificial-neural-network}

    Now that we have seen a neural network in action and have gained a basic
understanding of how it works by looking over the code, let's dig a
little bit deeper into some of the concepts, such as the logistic cost
function and the backpropagation algorithm that we implemented to learn
the weights.

    \subsubsection{Computing the logistic cost
function}\label{computing-the-logistic-cost-function}

    The logistic cost function that we implemented as the *\_compute\_cost*
method is actually pretty simple to follow since it is the same cost
function that we described in the logistic regression previously.

\[J(w) = -\sum_{i=1}^n y^{[i]}\log(a^{[i]}) + (1 - y^{[i]})\log(1-a^{[i]})\]

Here, \(a^{[i]}\) is the sigmoid activation of the \(i\)th sample in the
dataset, which we compute in the forward propagation step:

\[a^{[i]} = \phi(z^{[i]})\]

Again, note that in this context, the superscript \emph{{[}i{]}} is an
index for training samples, not layers.

Now, let's add a regularization term, which allows us to reduce the
degree of overfitting. As you recall from earlier chapters, the L2
regularization term is defined as follows (remember that we do not
regularize the bias units):

\[L2 = \lambda||w||_2^2 = \lambda\sum_{j=1}^m w_j^2\]

By adding the L2 regularization term to our logistic cost function, we
obtain the following equation:

\[J(w) = \left[-\sum_{i=1}^n y^{[i]}\log(a^{[i]}) + (1 - y^{[i]})\log(1-a^{[i]})\right] + \lambda||w||_2^2\]

Since we implemented an MLP for multiclass classification that returns
an output vector of \(t\) elements that we need to compare to the
\(t \times l\) dimensional target vector in the one-hot encoding
representation, for, example, the activation of the third layer and the
target class (here, class 2) for a particular sample may look like this:

\[
a^{out} = 
\begin{bmatrix}
0.1 \\
0.9 \\
\vdots \\
0.3
\end{bmatrix}, 
y = 
\begin{bmatrix}
0 \\
1 \\
\vdots \\
0
\end{bmatrix}
\]

Thus, we need to generalize the logistic cost function to all \(t\)
activation units in our network. Thus, the cost function (without the
regularization term) becomes the following:

\[J(w) = -\sum_{i=1}^n\sum_{j=1}^t y_j^{[i]}\log(a_j^{[i]}) + (1 - y_j^{[i]})\log(1-a_j^{[i]})\]

Here, again, the superscript \([i]\) is the index of a particular sample
in our training set.

The following generalized regularization term may look a little bit
complicated at first, but we are just calculating the sum of all weights
in an \(l\) layer (without the bias term) that we added to the first
column:

\[J(w) = -\left[\sum_{i=1}^n\sum_{j=1}^t y_j^{[i]}\log(a_j^{[i]}) + (1 - y_j^{[i]})\log(1-a_j^{[i]})\right] + \frac{\lambda}{2}\sum_{l=1}^{L-1}\sum_{i=1}^{u_l}\sum_{j=1}^{u_{l+1}}(w_{j,i}^{(t)})^2\]

Here, \(u_l\) refers to the number of units of a given layer l, and the
following expression represents the penalty term:

\[\frac{\lambda}{2}\sum_{l=1}^{L-1}\sum_{i=1}^{u_l}\sum_{j=1}^{u_{l+1}}(w_{j,i}^{(t)})^2\]

Remember that our goal is to minimize the cost function \(J(W)\); thus
we need to calculate the partial derivative of the parameters \(W\) with
respect to each weight for every layer in the network:

\[\frac{\partial}{\partial w_{j,i}^{(l)}} J(W)\]

In the next section, we will talk about the backpropagation algorithm,
which allows us to calculate those partial derivatives to minimize the
cost function.

Note that \(W\) consists of multiple matrices. In a multilayer
perceptron with one hidden unit, we have the weight matrix \(W^{(h)}\),
which connects the input to the hidden layer, and \(W^{out}\), which
connects the hidden layer to the output layer. An intuitive
visualization of the three-dimensional tensor \(W\) is provided in the
following figure:

In this simplified figure, it may seem that both \(W^{(h)}\) and
\(Ẁ^{(out)}\) have the same number of rows and columns, which is
typically not the case unless we initialize an MLP with the same number
of hidden layers, output units, and input features. If this sounds
confusing, stay tuned for the next section, where we will discuss the
dimensionality of \(W^{(h)}\) and \(W^{(out)}\) in more detail in the
context of the backpropagation algorithm. Also, I want to encourage you
to read through the code of the \emph{NeuralNetMLP} again, which I
annotated with helpful comments about the dimensionality with regard to
the different matrices and vector transformations.

    \subsubsection{Developing your intuition for
backpropagation}\label{developing-your-intuition-for-backpropagation}

    Although backpropagation was rediscovered and popularized more than 30
years ago, it still remains one of the most widely used algorithms to
train artificial neural networks very efficiently.

In this section, I intend to provide a short and intuitive summary and
the bigger picture of how this fascinating algorithm works before we
dive into more mathematical details. In essence, we can think of
backpropagation as a very computationally efficient approach to compute
the partial derivatives of a complex cost function in multilayer neural
networks. Here, our goal is to use those derivatives to learn the weight
coefficients for parametrizing such a multilayer artificial neural
network. The challenge in the parametrization of neural networks is that
we are typically dealing with a very large number of weight coefficients
in a high-dimensional feature space. In contrast to cost functions of
single-layer neural networks such as Adaline or logistic regression,
which we have seen in previous chapter, the error surface of a neural
network cost function is not convex or smooth with respect to the
parameters. There are many bumps in this high-dimensional cost surface
(local minima) that we have to overcome in order to find the global
minimum of the cost function.

You may recall the concept of the chain rule from your introductory
calculus. The chain rule is an approach to compute the derivative of a
complex, nested, function, such as \(f(g(x))\). In the context of
computer algebra, a set of techniques has been developed to solve such
problems very efficiently, which is also known as \textbf{automatic
differentiation}.

Automatic differentiation comes with two modes, the forward and reverse
modes; backpropagation is simply just a special case of reverse mode
automatic differentiation. The key point is that applying the chain rule
in the forward mode can be quite expensive since we would have to
multiply large matrices for each layer (Jacobians) that we eventually
multiply by a vector to obtain the output. The trick of reverse mode is
that we start from right to left: we multiply a matrix by a vector,
which yields another vector that is multiplied by the next matrix and so
on. Matrix-vector multiplication is computationally much cheaper than
matrix-matrix multiplication, which is why backpropagation is one of the
most popular algorithms used in neural network training.

    \subsubsection{Training neural networks via
backpropagation}\label{training-neural-networks-via-backpropagation}

    In this section, we will go through the math of backpropagation to
understand how you can learn the weights in a neural network very
efficiently. Depending on how comfortable you are with mathematical
representations, the following equations may seem relatively complicated
at first.

In a previous section, we saw how to calculate the cost as the
difference between the activation of the last layer and the target class
label. Now, we will see how the backpropagation algorithm works to
update the weights in our MLP model from a mathematical perspective,
which we implemented in the \emph{\# Backpropagation} section inside the
\emph{fit} method. As we recall from the beginning of this chapter, we
first need to apply forward propagation in order to obtain the
activation of the output layer, which we formulated as follows:

\[Z^{(h)} = A^{(in)}W^{(h)} \quad \text{net input of the hidden layer}\]

\[A^{(h)} = \phi(Z^{(h)}) \quad \text{activation of the hidden layer}\]

\[Z^{(out)} = A^{(h)}W^{(out)} \quad \text{net input of the output layer}\]

\[A^{(out)} = \phi(Z^{(out)}) \quad \text{activation of the output layer}\]

Concisely, we just forward-propagate the input features through the
connection in the network, as shown in the following illustration:

In backpropagation, we propagate the error from right to left. We start
by calculating the error vector of the output layer:

\[\delta^{(out)} = a^{(out)} - y\]

Here, \(y\) is the vector of the true class labels (the corresponding
variable in the \emph{NeuralNetMLP} code is \(sigma_out\)).

Next, we calculate the error term of the hidden layer:

\[\delta^{(h)} = \delta^{(out)}(W^{(out)})^T \odot \frac{\partial \phi(z^{(h)})}{\partial z^{(h)}}\]

Here, \(\frac{\partial \phi(z^{(h)})}{\partial z^{(h)}}\) is simply the
derivative of the sigmoid activation function, which we computed as
\emph{sigmoid\_derivative\_h = a\_h (1. - a\_h)} in the \emph{fit}
method of the \emph{NeuralNetMLP}:

\[\frac{\partial \phi(z^{(h)})}{\partial z^{(h)}} = (a^{(h)} \odot (1 - a^{(h)}))\]

Note that the \(\odot\) symbol means element-wise multiplication in this
context.

    Next, we compute the \(\delta^{(h)}\) layer error matrix
(\emph{sigma\_h}) as follows:

\[\delta^{(h)} = \delta^{(out)} (W^{(out)})^T \odot (a^{(h)} \odot (1 - a^{(h)}))\]

To better understand how we computed this \(\delta^{(h)}\) term, let's
walk through it in more detail. In the preceding equation, we used the
transpose \((W^{(out)})^T\) of the \(h \times t\)-dimensional matrix
\(W^{(out)}\). Here, \(t\) is the number of output class labels and
\(h\) is the number of hidden units. The matrix multiplication between
the \(n \times t\)-dimensional \(\delta^{(out)}\) matrix and the
\(t \times h\)-dimensional matrix \((W^{(out)})^T\), results in an
\(n \times t\)-dimensional matrix that we multiplied elementwise by the
sigmoid derivative of the same dimension to obtain the
\(n \times t\)-dimensional matrix \(\delta^{(h)}\). Eventually, after
obtaining the \(\delta\) terms, we can now write the derivation of the
cost function as follows:

\[\frac{\partial}{\partial w_{i,j}^{out}} J(W) = a_j^{(h)} \delta_i^{(out)}\]

\[\frac{\partial}{\partial w_{i,j}^{h}} J(W) = a_j^{(in)} \delta_i^{(h)}\]

Next, we need to accumulate the partial derivative of every node in each
layer and the error of the node in the next layer. However, remember
that we need to compute \(\Delta_{i,j}^{(l)}\) for every sample in the
training set. Thus, it is easier to implement it as a vectorized version
like in our \emph{NeuralNetMLP} code implementation:

\[\Delta^{(h)} = \Delta^{(h)} + (A^{(in)})^T \delta^{(h)}\]

\[\Delta^{(out)} = \Delta^{(out)} + (A^{(h)})^T \delta^{(out)}\]

And after we have accumulated the partial derivatives, we can add the
regularization term:

\[\Delta^{(l)} := \Delta^{(l)} + \lambda^{(l)} \quad \text{except for the bias term}\]

The two previous mathematical equations correspond to the code variables
\emph{delta\_w\_h}, \emph{delta\_b\_h}, \emph{delta\_w\_out}, and
\emph{delta\_b\_out} in \emph{NeuralNetMLP}.

Lastly, after we have computed the gradients, we can now update the
weights by taking an opposite step towards the gradient for each layer
\(l\):

\[W^{(l)} := W^{(l)} - \eta \delta^{(l)}\]

To bring everything together, let's summarize backpropagation in the
following figure:

    \subsection{About the convergence in neural
networks}\label{about-the-convergence-in-neural-networks}

    You might be wondering why we did not use regular gradient descent but
instead used mini-batch learning to train our neural network for the
handwrittten digit classification. You may recall our discussion on
stochastic gradient descent that we used to implement online learning.
In online learning, we compute the gradient based on a single training
sample (\(k=1\)) at a time to perform the weight update. Although this
is a stochastic approach, it often leads to very accurate solutions with
a much faster convergence than regular gradient descent. Mini-batch
learning is a special form of stochastic gradient descent where we
compute the gradient based on a subset \(k\) of the \(n\) training
samples with \(l < k < n\). Mini-batch learning has the advantage over
online learning that we can make use of our vectorized implementations
to improve computational efficiency. However, we can update the weights
much faster than in regular gradient descent. Intuitively, you can think
of mini-batch learning as predicting the voter turnout of a presidential
election from a poll asking only a representative subset of the
population rather than asking the entire population (which would be
equal to running to actual election).

Multilayer neural networks are much harder to train than simpler
algorithms such as Adaline, logistic regression, or support vector
machines. In multilayer neural networks, we typically have hundreds,
thousands, or even billions of weights that we need to optimize.
Unfortunately, the output function has a rough surface and the
optimization algorithm can easily become trapped in local minima, as
shown in the following figure:

    Note that this representation is extremely simplified since our neural
network has many dimensions; it makes it impossible to visualize the
actual cost surface for the human eye. Here, we only show the cost
surface for a single weight on the \(x\)-axis. However, the main message
is that we do not want our algorithm to get trapped in local minima. By
increasing the learning rate, we can more readily escape such local
minima. On the other hand, we also increase the chance of overshooting
the global optimum if the learning rate is too large. Since we
initialize the weights randomly, we start with a solution to the
optimization problem that is typically hopelessly wrong.

    \section{A few last words about the neural network
implementation}\label{a-few-last-words-about-the-neural-network-implementation}

    You may be wondering why we went through all of this theory just to
implement a simple multilayer network that can classify handwritten
digits instead of using an open source Python machine learning library.
In fact, we will introduce more complex neural network models in the
next chapters, which we will train using the open source TensorFlow
library. Although the from scratch implementation in this chapter seems
a bit tedious at first, it was a good exercise for understanding the
basics behind backpropagation and neural network training, and a basic
understanding of algorithms is crucial for applying machine learning
techniques appropriately and successfully.

Now that you have learned how feedforward neural network work, we are
ready to explore more sophisticated deep neural networks, such as
TensorFlow and Keras, which allow us to construct neural networks more
efficiently. Over the past two years, since its release in November
2015, Tensorflow has gained a lot of popularity among machine learning
researchers, who use it to construct deep neural networks because of its
ability to optimize mathematical expressions for computations on multi
dimensional arrays utilizing \textbf{Graphical Processing Units (GPUs)}.
While TensorFlow can be considered a low-level deep learning library,
simplifying API such as Keras have been developed that make the
construction of common deep learning models even more convenient.

    \section{Summary}\label{summary}

    In this chapter, you have learned the basic concepts behind multilayer
artificial neural networks, which are currently the hottest topics in
machine learning research. In a previous chapter, we started our journey
with simple-layer neural network structures and now we have connected
multiple neurons to a powerful neural network architecture to solve
complex problems such as handwritten digit recognition. We demystified
the popular backpropagation algorithm, which is one of the building
blocks of many neural networks models that are used in deep learning.
After learning about the backpropagation algorithm in this chapter, we
are well-equipped for exploring more complex deep neural network
architectures. In the remaining chapters, we will introduce TensorFlow,
an open source library geared towards deep learning, which allows us to
implement and train multilayer neural networks more efficiently.


    % Add a bibliography block to the postdoc
    
    
    
    \end{document}
